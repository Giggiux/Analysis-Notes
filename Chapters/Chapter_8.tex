\chapter{Differentialrechnung in $\mathbb{R}^n$}
\section{Partielle Ableitungen und Differential}
Wie kann man die Begriffe der \todo{Missing content?? page 113 top} Differentialrechnung auf Funktionen $f:\Omega \subset \mathbb{R}^n\rightarrow\mathbb{R}$ erweitern?\\

Funktion in mehreren variablen sind ein bisschen komplizierter als Funktionen in einer variable.
\subsubsection*{Beispiel}
\begin{enumerate}
\item $f(x)=x^2+5$ ist in ursprung stetig da $\lim\limits_{x\rightarrow 0}f(x)=f(0)$. Aber $f:\mathbb{R}^2\rightarrow\mathbb{R}$ \[f(x,y) = \left\{ {\begin{array}{*{20}{c}}
{\frac{{xy}}{{{x^2} + {y^2}}}}&{(x,y)\not  = (0,0)}\\
0&{(x,y) = (0,0)}
\end{array}} \right.\] ist im Ursprung nicht stetig. %page 114 top
\todo{Where is number 2 of the beispiel??}
\end{enumerate}
\todo{is this continuation of the Beispiel, or is it outside??}
\begin{figure}[ht]
\begin{minipage}[b]{0.45\linewidth}
\centering
\[\mathop {\lim }\limits_{\begin{array}{*{20}{c}}
{x \to 0}\\
{y = 0}
\end{array}} \frac{{x \cdot y}}{{{x^2} + {y^2}}} = 0 = f(0,0)\]
\end{minipage}
\hspace{0.5cm}
\begin{minipage}[b]{0.45\linewidth}
\centering
\[\mathop {\lim }\limits_{\begin{array}{*{20}{c}}
{y \to 0}\\
{x = 0}
\end{array}} \frac{{x \cdot y}}{{{x^2} + {y^2}}} = 0 = f(0,0)\]
\end{minipage}
\end{figure}

Aber der Limes entlang der Gerade $y=mx$ 
\[\mathop {\lim }\limits_{\begin{array}{*{20}{c}}
{x \to 0}\\
{y \to 0}\\
{y = mx}
\end{array}} f(x,mx) = \mathop {\lim }\limits_{x \to 0} \frac{{m{x^2}}}{{(1 + {m^2}){x^2}}} = \mathop {\frac{m}{{1 + {m^2}}}}\limits_{\begin{array}{*{20}{c}}
 \downarrow \\
{{\text{Hängt von }} m {\text{ ab}}}
\end{array}} \]
und $\frac{m}{1+m^2}\not=0$, falls $m\not=0$. Eine funktion $f(x,y)$ an der stelle $(x_0,y_0)$ ist stetig wenn der limes $\mathop {\lim }\limits_{(x,y) \to ({x_0},{y_0})} f(x,y)$ in jeder Richtung der gleichen wert haben. 
\begin{framed}
\centerline{\textbf{Definition 8.1}}
\noindent Sei $\Omega\subset\mathbb{R}^n$, $f:\Omega \rightarrow\mathbb{R}$, $a\in\Omega$
\begin{enumerate}
\item $f$ hat den Grenzwert $c\in\mathbb{R}$, d.h $$\lim\limits_{x\rightarrow a} f(x)=c$$ ween es zu jeder (Beliebig kleinen) Schranke $\varepsilon>0$, eine $\delta$-umgebung \[{B_\delta }(a): = \left\{ {x \in \mathbb{R}^n}\mid\left| {x - a} \right| < \delta  \right\}\] gibt, so dass $\left| {f(x) - a} \right| < \varepsilon$ für alle $x\in\Omega\cap B_\delta (a), x\not=a$ gilt
\item $f$ heisst in $a\in\Omega$ stetig, wenn $\mathop {\lim }\limits_{x \to a} f'(x) = f(a)$ gilt.
\item $f$ heisst in $\Omega$ stetig, wenn $f$ in allen $a\in\Omega$ stetig ist. 
\end{enumerate}
Die Summe, das Produkt, der Quotient (Nenner ungleich Null) stetiger Funktion sind stetig.\\

$f$ besitzt keinen Grenzwert in $x_0$ wenn sich bei Annäherungen an $x_0$ auf verschiedenen Kurven (z.b. Geraden) verschiedene oder keine Grenzwert ergeben.
\end{framed}

\subsection*{Sandwichlemma}
Sei $f,g,h$ funktionen wobei $g<f<h$. Wenn $\mathop {\lim }\limits_{x \to a} g = L = \mathop {\lim }\limits_{x \to a} h$ gilt, dann ergibt $\lim\limits_{x\rightarrow a}f=L$.\\

\noindent Da $\mathop {\lim }\limits_{(x,y) \to (0,0)} \left| y \right| = 0$ gilt, $\mathop {\lim }\limits_{(x,y) \to (0,0)} f(x,y) = 0 \Rightarrow f$ ist in (0,0) stetig.\\

\noindent \textbf{\underline{Oder}}\\

\noindent Für Grenzwertbestimmungen (also auch für Stetigkeitsuntersuchungen) ist es oft nützlich, die Funktionen mittels Polarkoordinaten umzuschreiben. Vor allem bei Rationalen Funktionen. \\

Hierbei gilt $x=r\cos\theta$, $y=r\sin\theta$, wobei $r=$ länge des Vektors $(x,y)$ und $\varphi$ der Winkel. Nun lass wir die Länge $r$ gegen 0 gehen. 

\subsubsection*{Beispiel}
\begin{enumerate}
\item Die Funktionen 
\begin{itemize}
\item $f(x,y)=x^2+y^2$
\item $f(x,y,z)=x^3+\frac{x^2}{y^2+1}+z$
\item $f(x,y)=4x^2 y^3+3xy$
\item $f(x,y)=\cos xy$
\end{itemize}
sind stetig, da sie aus Steigen Funktionen zusammengesetzt.

\item \[f(x,y) = \left\{ {\begin{array}{*{20}{c}}
{\frac{{{x^2}y}}{{{x^2} + {y^2}}}}&{{\text{für}}}&{{\text{(x,y)}}\not  = (0,0)}\\
0&{{\text{für}}}&{{\text{(x,y)}}= (0,0)}
\end{array}} \right.\]
Für $(x,y)\not=(0,0)$ ist $f$ als Quotient von steiger Funktionen stetig. Es verbleibt $f$ im Punkt $(0,0)$ zu untersuchen. Da \[\left| {\frac{{{x^2}}}{{{x^2} + {y^2}}}} \right| \le 1\] $$ 0<\left| f(x,y)\right| <\left| y\right|$$ \[f(x,y) = \frac{{{x^2}y}}{{{x^2} + {y^2}}} = \frac{{\left( {{r^2}{{\cos }^2}\theta } \right)\left( {r\sin \theta } \right)}}{{{r^2}\left( {{{\cos }^2}\theta  + {{\sin }^2}\theta } \right)}} = r{\cos ^2}\theta \sin \theta \] 
\[\mathop {\lim }\limits_{r \to 0} f(r,\theta ) = \mathop {\lim }\limits_{r \to 0} r{\cos ^2}\theta \sin \theta  = 0\]
\item Wir können nochmals die Stetigkeit der Funktion 
\[f(x,y) = \left\{ {\begin{array}{*{20}{c}}
{\frac{{{x^2}y}}{{{x^2} + {y^2}}}}&{{\text{für}}}&{{\text{(x,y)}}\not  = (0,0)}\\
0&{{\text{für}}}&{{\text{(x,y)}}= (0,0)}
\end{array}} \right.\] mittels Polarkoordinaten untersuchen $$f(x,y)=\frac{r^2\cos\theta\sin\theta}{r^2}=\cos\theta\sin\theta$$ \[\mathop {\lim }\limits_{r \to 0} f(x,y) = \cos \theta \sin \theta \] hängt von $\theta$ ab. $$\Rightarrow f\text{ in (0,0) nicht stetig}$$

\subsubsection*{Bemerkung}\todo{is this supposed to be inside the list or out??}
Eine trickreiche Variante Grenzwerte zu berechnen, ergibt sich durch substitution, d.h. man berechnet den Grenzwert \[\mathop {\lim }\limits_{(x,y) \to ({x_0},{y_0})} f\left( {g(x,y)} \right)\] indem man zunächst $t=g(x,y)$ setzt und den Grenzwert \[{t_0} = \mathop {\lim }\limits_{(x,y) \to ({x_0},{y_0})} g(x,y)\] bestimmt. Dann ist \[\mathop {\lim }\limits_{(x,y) \to ({x_0},{y_0})} f\left( {g(x,y)} \right) = \mathop {\lim }\limits_{t \to {t_0}} f(t)\] 
\end{enumerate}

\subsubsection*{Beispiel}
\[\mathop {\lim }\limits_{(x,y) \to (4,0)} \frac{{\sin xy}}{{xy}}\] Hier ist $g(x,y)=xy$, $\mathop {\lim }\limits_{(x,y) \to (4,0)} g(x,y) = 0$. Somit \[\mathop {\lim }\limits_{(x,y) \to (4,0)} \frac{{\sin xy}}{{xy}} = \mathop {\lim }\limits_{t \to 0} \frac{{\sin t}}{t} = 1\] Wir werden auch sehen das die Existenz der Ableitungen in einigen Richtungen ungenügend für die Differenzierbarkeit der Funktion ist. \\

\noindent\textbf{\underline{Was bedeutet die Ableitung in einiger Richtung?}}
\subsubsection*{Beispiel}
Sei $$f:\mathbb{R}^2\rightarrow \mathbb{R}$$$$(x,y)\rightarrow \left(x^2+xy\right)\cos(xy)$$Man kann für jedes $y$, die Funktion $$\mathbb{R}\rightarrow\mathbb{R}$$$$x\rightarrow \left( x^2+xy\right)\left(\cos xy\right)$$als Funktion einer Variablen $x$ auflassen und die Ableitung davon berechnen. Das Resultat mit $\frac{\partial f}{\partial x}$ bezeichnet, ist die erste partielle Ableitung von $f$ nach $x$. In diesem fall ist es durch \[\frac{{\partial f}}{{\partial x}}(x,y) = (2x + y)(\cos xy) - ({x^2} + xy)y\sin (xy)\] gegeben. \\

\noindent Analog definiert man $\frac{\partial f}{\partial y}$\[\frac{{\partial f}}{{\partial y}}(x,y) = x(\cos xy) - ({x^2} + xy)x\sin (xy)\] Die allgemeine Definition nimmt folgende Gestallt ein. Sei $\Omega \subset\mathbb{R}^n$. In zukunft  bezeichnen wir die $i-$te Koordinate eines Vektors $x\in\mathbb{R}^n$ mit $x^i$; also ist $x=\left( x^1,x^2,\dots,x^n\right)$.\\

\noindent Sei $e_i:=\left( 0,\dots,0,1,0,\dots,0\right)$ der $i-$te Basisvektor von $\mathbb{R}^n$

\begin{framed}
\centerline{\textbf{Definition 8.2}}
\noindent Die Funktion $f:\Omega\subset\mathbb{R}^n\rightarrow\mathbb{R}$ heisst an der stelle $x_0\in\Omega$ in Richtung $e_i$ (oder nach $x^i$) partielle differenzierbar falls der limes \[\frac{{\partial f}}{{\partial {x^i}}}({x_0}) = {f_{{x^i}}}({x_0}): =  - \mathop {\lim }\limits_{\begin{array}{*{20}{c}}
{h \to 0}\\
{h\not  = 0}
\end{array}} \frac{{f({x_0} + h{e_i}) - f({x_0})}}{h}\]
\[ = \mathop {\lim }\limits_{\begin{array}{*{20}{c}}
{h \to 0}\\
{h\not  = 0}
\end{array}} \frac{{f\left( {{x_0}^1,{x_0}^2, \ldots ,{x_0}^i + h,{x_0}^{i + 1}, \ldots ,{x_0}^n} \right) - f\left( {{x_0}^1, \ldots ,{x_0}^n} \right)}}{h}\]
existiert %Page 121
\end{framed}

