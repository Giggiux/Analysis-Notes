\chapter{Differentialrechnung in $\mathbb{R}^{\lowercase{n}}$}
\section{Partielle Ableitungen und Differential}
Wie kann man die Begriffe der \todo{Missing content?? page 113 top} Differentialrechnung auf Funktionen $f:\Omega \subset \mathbb{R}^n\rightarrow\mathbb{R}$ erweitern?\\

Funktion in mehreren variablen sind ein bisschen komplizierter als Funktionen in einer variable.
\subsubsection*{Beispiel}
\begin{enumerate}
\item $f(x)=x^2+5$ ist in ursprung stetig da $\lim\limits_{x\rightarrow 0}f(x)=f(0)$. Aber $f:\mathbb{R}^2\rightarrow\mathbb{R}$ \[f(x,y) = \left\{ {\begin{array}{*{20}{c}}
{\frac{{xy}}{{{x^2} + {y^2}}}}&{(x,y)\not  = (0,0)}\\
0&{(x,y) = (0,0)}
\end{array}} \right.\] ist im Ursprung nicht stetig. %page 114 top
\todo{Where is number 2 of the beispiel??}
\end{enumerate}
\todo{is this continuation of the Beispiel, or is it outside??}
\begin{figure}[ht]
\begin{minipage}[b]{0.45\linewidth}
\centering
\[\mathop {\lim }\limits_{\begin{array}{*{20}{c}}
{x \to 0}\\
{y = 0}
\end{array}} \frac{{x \cdot y}}{{{x^2} + {y^2}}} = 0 = f(0,0)\]
\end{minipage}
\hspace{0.5cm}
\begin{minipage}[b]{0.45\linewidth}
\centering
\[\mathop {\lim }\limits_{\begin{array}{*{20}{c}}
{y \to 0}\\
{x = 0}
\end{array}} \frac{{x \cdot y}}{{{x^2} + {y^2}}} = 0 = f(0,0)\]
\end{minipage}
\end{figure}

Aber der Limes entlang der Gerade $y=mx$ 
\[\mathop {\lim }\limits_{\begin{array}{*{20}{c}}
{x \to 0}\\
{y \to 0}\\
{y = mx}
\end{array}} f(x,mx) = \mathop {\lim }\limits_{x \to 0} \frac{{m{x^2}}}{{(1 + {m^2}){x^2}}} = \mathop {\frac{m}{{1 + {m^2}}}}\limits_{\begin{array}{*{20}{c}}
 \downarrow \\
{{\text{Hängt von }} m {\text{ ab}}}
\end{array}} \]
und $\frac{m}{1+m^2}\not=0$, falls $m\not=0$. Eine funktion $f(x,y)$ an der stelle $(x_0,y_0)$ ist stetig wenn der limes $\mathop {\lim }\limits_{(x,y) \to ({x_0},{y_0})} f(x,y)$ in jeder Richtung der gleichen wert haben. 
\begin{definition}{8.1}
Sei $\Omega\subset\mathbb{R}^n$, $f:\Omega \rightarrow\mathbb{R}$, $a\in\Omega$
\begin{enumerate}
\item $f$ hat den Grenzwert $c\in\mathbb{R}$, d.h $$\lim\limits_{x\rightarrow a} f(x)=c$$ ween es zu jeder (Beliebig kleinen) Schranke $\varepsilon>0$, eine $\delta$-umgebung \[{B_\delta }(a): = \left\{ {x \in \mathbb{R}^n}\mid\left| {x - a} \right| < \delta  \right\}\] gibt, so dass $\left| {f(x) - a} \right| < \varepsilon$ für alle $x\in\Omega\cap B_\delta (a), x\not=a$ gilt
\item $f$ heisst in $a\in\Omega$ stetig, wenn $\mathop {\lim }\limits_{x \to a} f'(x) = f(a)$ gilt.
\item $f$ heisst in $\Omega$ stetig, wenn $f$ in allen $a\in\Omega$ stetig ist. 
\end{enumerate}
Die Summe, das Produkt, der Quotient (Nenner ungleich Null) stetiger Funktion sind stetig.\\

$f$ besitzt keinen Grenzwert in $x_0$ wenn sich bei Annäherungen an $x_0$ auf verschiedenen Kurven (z.b. Geraden) verschiedene oder keine Grenzwert ergeben.
\end{definition}

\subsection*{Sandwichlemma}
Sei $f,g,h$ funktionen wobei $g<f<h$. Wenn $\mathop {\lim }\limits_{x \to a} g = L = \mathop {\lim }\limits_{x \to a} h$ gilt, dann ergibt $\lim\limits_{x\rightarrow a}f=L$.\\

\noindent Da $\mathop {\lim }\limits_{(x,y) \to (0,0)} \left| y \right| = 0$ gilt, $\mathop {\lim }\limits_{(x,y) \to (0,0)} f(x,y) = 0 \Rightarrow f$ ist in (0,0) stetig.\\

\noindent \textbf{\underline{Oder}}\\

\noindent Für Grenzwertbestimmungen (also auch für Stetigkeitsuntersuchungen) ist es oft nützlich, die Funktionen mittels Polarkoordinaten umzuschreiben. Vor allem bei Rationalen Funktionen. \\

Hierbei gilt $x=r\cos\theta$, $y=r\sin\theta$, wobei $r=$ länge des Vektors $(x,y)$ und $\varphi$ der Winkel. Nun lass wir die Länge $r$ gegen 0 gehen. 

\subsubsection*{Beispiel}
\begin{enumerate}
\item Die Funktionen 
\begin{itemize}
\item $f(x,y)=x^2+y^2$
\item $f(x,y,z)=x^3+\frac{x^2}{y^2+1}+z$
\item $f(x,y)=4x^2 y^3+3xy$
\item $f(x,y)=\cos xy$
\end{itemize}
sind stetig, da sie aus Steigen Funktionen zusammengesetzt.

\item \[f(x,y) = \left\{ {\begin{array}{*{20}{c}}
{\frac{{{x^2}y}}{{{x^2} + {y^2}}}}&{{\text{für}}}&{{\text{(x,y)}}\not  = (0,0)}\\
0&{{\text{für}}}&{{\text{(x,y)}}= (0,0)}
\end{array}} \right.\]
Für $(x,y)\not=(0,0)$ ist $f$ als Quotient von steiger Funktionen stetig. Es verbleibt $f$ im Punkt $(0,0)$ zu untersuchen. Da \[\left| {\frac{{{x^2}}}{{{x^2} + {y^2}}}} \right| \le 1\] $$ 0<\left| f(x,y)\right| <\left| y\right|$$ \[f(x,y) = \frac{{{x^2}y}}{{{x^2} + {y^2}}} = \frac{{\left( {{r^2}{{\cos }^2}\theta } \right)\left( {r\sin \theta } \right)}}{{{r^2}\left( {{{\cos }^2}\theta  + {{\sin }^2}\theta } \right)}} = r{\cos ^2}\theta \sin \theta \] 
\[\mathop {\lim }\limits_{r \to 0} f(r,\theta ) = \mathop {\lim }\limits_{r \to 0} r{\cos ^2}\theta \sin \theta  = 0\]
\item Wir können nochmals die Stetigkeit der Funktion 
\[f(x,y) = \left\{ {\begin{array}{*{20}{c}}
{\frac{{{x^2}y}}{{{x^2} + {y^2}}}}&{{\text{für}}}&{{\text{(x,y)}}\not  = (0,0)}\\
0&{{\text{für}}}&{{\text{(x,y)}}= (0,0)}
\end{array}} \right.\] mittels Polarkoordinaten untersuchen $$f(x,y)=\frac{r^2\cos\theta\sin\theta}{r^2}=\cos\theta\sin\theta$$ \[\mathop {\lim }\limits_{r \to 0} f(x,y) = \cos \theta \sin \theta \] hängt von $\theta$ ab. $$\Rightarrow f\text{ in (0,0) nicht stetig}$$

\subsubsection*{Bemerkung}\todo{is this supposed to be inside the list or out??}
Eine trickreiche Variante Grenzwerte zu berechnen, ergibt sich durch substitution, d.h. man berechnet den Grenzwert \[\mathop {\lim }\limits_{(x,y) \to ({x_0},{y_0})} f\left( {g(x,y)} \right)\] indem man zunächst $t=g(x,y)$ setzt und den Grenzwert \[{t_0} = \mathop {\lim }\limits_{(x,y) \to ({x_0},{y_0})} g(x,y)\] bestimmt. Dann ist \[\mathop {\lim }\limits_{(x,y) \to ({x_0},{y_0})} f\left( {g(x,y)} \right) = \mathop {\lim }\limits_{t \to {t_0}} f(t)\] 
\end{enumerate}

\subsubsection*{Beispiel}
\[\mathop {\lim }\limits_{(x,y) \to (4,0)} \frac{{\sin xy}}{{xy}}\] Hier ist $g(x,y)=xy$, $\mathop {\lim }\limits_{(x,y) \to (4,0)} g(x,y) = 0$. Somit \[\mathop {\lim }\limits_{(x,y) \to (4,0)} \frac{{\sin xy}}{{xy}} = \mathop {\lim }\limits_{t \to 0} \frac{{\sin t}}{t} = 1\] Wir werden auch sehen das die Existenz der Ableitungen in einigen Richtungen ungenügend für die Differenzierbarkeit der Funktion ist. \\

\noindent\textbf{\underline{Was bedeutet die Ableitung in einiger Richtung?}}
\subsubsection*{Beispiel}
Sei $$f:\mathbb{R}^2\rightarrow \mathbb{R}$$$$(x,y)\rightarrow \left(x^2+xy\right)\cos(xy)$$Man kann für jedes $y$, die Funktion $$\mathbb{R}\rightarrow\mathbb{R}$$$$x\rightarrow \left( x^2+xy\right)\left(\cos xy\right)$$als Funktion einer Variablen $x$ auflassen und die Ableitung davon berechnen. Das Resultat mit $\frac{\partial f}{\partial x}$ bezeichnet, ist die erste partielle Ableitung von $f$ nach $x$. In diesem fall ist es durch \[\frac{{\partial f}}{{\partial x}}(x,y) = (2x + y)(\cos xy) - ({x^2} + xy)y\sin (xy)\] gegeben. \\

\noindent Analog definiert man $\frac{\partial f}{\partial y}$\[\frac{{\partial f}}{{\partial y}}(x,y) = x(\cos xy) - ({x^2} + xy)x\sin (xy)\] Die allgemeine Definition nimmt folgende Gestallt ein. Sei $\Omega \subset\mathbb{R}^n$. In zukunft  bezeichnen wir die $i-$te Koordinate eines Vektors $x\in\mathbb{R}^n$ mit $x^i$; also ist $x=\left( x^1,x^2,\dots,x^n\right)$.\\

\noindent Sei $e_i:=\left( 0,\dots,0,1,0,\dots,0\right)$ der $i-$te Basisvektor von $\mathbb{R}^n$

\begin{definition}{8.2}
Die Funktion $f:\Omega\subset\mathbb{R}^n\rightarrow\mathbb{R}$ heisst an der stelle $x_0\in\Omega$ in Richtung $e_i$ (oder nach $x^i$) partielle differenzierbar falls der limes \[\frac{{\partial f}}{{\partial {x^i}}}({x_0}) = {f_{{x^i}}}({x_0}): =  - \mathop {\lim }\limits_{\begin{array}{*{20}{c}}
{h \to 0}\\
{h\not  = 0}
\end{array}} \frac{{f({x_0} + h{e_i}) - f({x_0})}}{h}\]
\[ = \mathop {\lim }\limits_{\begin{array}{*{20}{c}}
{h \to 0}\\
{h\not  = 0}
\end{array}} \frac{{f\left( {{x_0}^1,{x_0}^2, \ldots ,{x_0}^i + h,{x_0}^{i + 1}, \ldots ,{x_0}^n} \right) - f\left( {{x_0}^1, \ldots ,{x_0}^n} \right)}}{h}\]
existiert
\end{definition}
\subsubsection*{Bemerkung 8.3}
\missingfigure{page 121, middle}
Sei $f:\mathbb{R}^2\rightarrow\mathbb{R}, \left(x_0^1,x_0^2\right)\in\mathbb{R}^2$. Wir betrachten die scharen von $f$ $$f(\cdot ,x_0^2):\mathbb{R}\rightarrow \mathbb{R}$$ und $$f(x_0^1,\cdot ):\mathbb{R}\rightarrow\mathbb{R}$$ $\frac{\partial f}{\partial x^1}$, $\frac{\partial f}{\partial x^2}$ sind die Ansteig der Tangente zur entsprechende schrittkurven

\subsubsection*{Beispiel}
\begin{enumerate}
\item $f(x,y,z)=\cos yz+\sin xy$
\begin{itemize}
\item $\frac{\partial f}{\partial x}=y\cos xy$
\item $\frac{\partial f}{\partial y}=-\sin(yz)\cdot z+\cos(xy)\cdot x$
\item $\frac{\partial f}{\partial z}=-\sin(yz)\cdot y$
\end{itemize}
\item \[f(x,y) = \left\{ {\begin{array}{*{20}{c}}
{\frac{{{x^3}y}}{{{x^2} + {y^2}}}}&{(x,y)\not  = (0,0)}\\
0&{(x,y)\not  = (0,0)}
\end{array}} \right.\]
$$\frac{{\partial f}}{{\partial x}}(0,0) = \mathop {\lim }\limits_{h \to 0} \frac{{f(h,0) - f(0,0)}}{h} = \lim \frac{{\frac{{{h^3} \cdot 0}}{{{h^2}}} - 0}}{h} = 0$$
\[\frac{{\partial f}}{{\partial y}}(0,0) = \mathop {\lim }\limits_{h \to 0} \frac{{f(0,h) - f(0,0)}}{h} = \lim \frac{{\frac{{h \cdot {0^3}}}{{0 + {h^2}}} - 0}}{h} = 0\]
\end{enumerate}
\subsubsection*{Bemerkung}
Für Funktionen $f:\mathbb{R}\rightarrow\mathbb{R}$ einer variable impliziert die differenzierbarkeit in $x_0$, die Stetigkeit in $x_0$ und zudem eine gute Approximation von $f$ durch eine affine Funktion in einer Umgebung von $x_0$. Folgendes Beispiel zeigt, dass in $\mathbb{R}^n$ $(n\geq 2)$ Partielle Differenzierbarkeit keine analoges  Approximationseigenschaften oder stetigkeit impliziert:
\[f: \mathbb{R}^2 \to\mathbb{R} ,{\text{ }}f(x,y) = \left\{ {\begin{array}{*{20}{c}}
{\frac{{xy}}{{{x^2} + {y^2}}}}&{(x,y)\not  = (0,0)}\\
0&{(x,y)\not  = (0,0)}
\end{array}} \right.\]
Für alle $\left( x_0,y_0\right) \in\mathbb{R}^2$ ist $f$ in beiden Richtungen partiel differenzierbar:
\begin{itemize}
\item Für $\left( x_0,y_0\right)\not=\left( 0,0\right)$ \[\frac{{\partial f}}{{\partial x}}\left( {{x_0},{y_0}} \right) = {\left. {\frac{{y\left( {{x^2} + {y^2}} \right) - 2{x^2}y}}{{{{\left( {{x^2} + {y^2}} \right)}^2}}}} \right|_{(x,y) = \left( {{x_0},{y_0}} \right)}} = \frac{{y_0^3 - x_0^2{y_0}}}{{{{\left( {{x_0}^2 + {y_0}^2} \right)}^2}}}\]
\[\frac{{\partial f}}{{\partial y}}\left( {{x_0},{y_0}} \right) = {\left. {\frac{{x\left( {{x^2} + {y^2}} \right) - 2x{y^2}}}{{{{\left( {{x^2} + {y^2}} \right)}^2}}}} \right|_{(x,y)\not  = \left( {{x_0},{y_0}} \right)}} = \frac{{{x^2} - x{y^2}}}{{{{\left( {{x^2} + {y^2}} \right)}^2}}}\]
\item Für $\left(x_0,y_0\right)=\left( 0,0\right)$\[\frac{{\partial f}}{{\partial x}}(0,0) = \mathop {\lim }\limits_{h \to 0} \frac{{\overbrace {f\left( {0 + h,0} \right) - f\left( {0,0} \right)}^{f({x_0} + h{e_1}) - f(x_0)}}}{h} = \lim \frac{0}{h} = 0\]\[\frac{{\partial f}}{{\partial y}}(0,0) = \mathop {\lim }\limits_{h \to 0} \frac{{\overbrace {f\left( {0,0 + h} \right) - f\left( {0,0} \right)}^{f({x_0} + h{e_2}) - f({x_0})}}}{h} = \lim \frac{0}{h} = 0\]
\end{itemize}
Im Ursprung besitzt $f$ beide partielle Ableitungen, sie ist aber nicht stetig. Der Grund ist, dass die partielle Ableitungen nur partielle Informationen geben. Wir müssen die Differenzierbarkeit irgend eine andere weise verallgemeinen.\\

Die Lösung dieses Problem ist, dass man eine Approximations-Eigenschaft durch eine Lineare Abbildung postuliert. \\

Sei $f:\mathbb{R}\rightarrow\mathbb{R}$ differenzierbar in $x_0;f'(x_0)$ existiert. In diesem Fall kann $f$ für alle $x$ nähe $x_0$ durch die Funktion $f(x_0)+f'(x_0)(x-x_0)$ gut approximiert werden. Dass heisst dass $$f(x)=f(x_0)+f'(x_0)(x-x_0)+R(x,x_0)\text{ \hspace{2mm}mit  }\mathop {\lim }\limits_{x \to {x_0}} \frac{{R(x,{x_0})}}{{x - {x_0}}} = 0$$

\subsubsection*{Bemerkung}
$f'(x):\mathbb{R}\rightarrow\mathbb{R}'$ sollt als lineare Abbildung interpretiert werden
\subsection*{Lineare Abbildungen}
Eine Abbildung $A:\mathbb{R}^n\rightarrow\mathbb{R}$ ist linear falls für alle $x,y\in\mathbb{R}^n$ und $\alpha,\beta\in\mathbb{R}$ $$A\left( \alpha x+\beta y\right) =\alpha A(x)+\beta A(y)$$
Eine solche Abbildung ist durch ihre Werte $$A(e_i):=A_1,A(e_2):=A_2,\dots,A(e_n):=A_n$$ auf der Standardbasis $e_1,\dots,e_n$ eindeutig bestimmt. Aus $x = \sum\limits_{i = 1}^n {{x^i}{e_i}} $ und linearität folgt nämlich

\[
A(x) = \sum\limits_{i = 1}^n {{x^i}A({e_i}) = \sum\limits_{i = 1}^n {{A_i}{x^i}} } \tag{\textasteriskcentered}
\]
Umgekehrt bestimmt ein Vektor $\left( A_1,\dots,A_n\right)$ vermöge der Formel (\textasteriskcentered) eine Lineare Abbildung.\\

Schreiben wir $x = \left( {\begin{array}{*{20}{c}}
{{x^1}}\\
 \vdots \\
{{x^n}}
\end{array}} \right)$ für einen Vektor $x = {({x^1})_{1 \le i \le n}}$ und \\ %break needed to have everything on the same line
$A=\left( A_1,\dots,A_n\right)$ für die Darstellung einer Lineare Abbildung $A:\mathbb{R}^n\rightarrow\mathbb{R}$ bezüglich die Standard Basis $\left\{e_1,\dots,e_n\right\}$ so ist \[A(x) = \left( {{A_1}, \ldots ,{A_n}} \right)\left( {\begin{array}{*{20}{c}}
{{x^1}}\\
 \vdots \\
{{x^n}}
\end{array}} \right) = \sum {{A_i}{x^i}} \]

\begin{definition}{8.4}
Die Funktion $f:\Omega\rightarrow\mathbb{R}$ heisst an der Stelle $x_0\in\Omega\subset\mathbb{R}^n$ differenzierbar falls eine lineare Abbildung $A:\mathbb{R}^n\rightarrow\mathbb{R}$ gibt so dass \[f(x) = f\left( {{x_0}} \right) + A\left( {x - {x_0}} \right) + R\left( {{x_0},x} \right)\] wobei $\mathop {\lim }\limits_{x \to {x_0}} \frac{{R\left( {x,{x_0}} \right)}}{{\left| {x - {x_0}} \right|}} = 0$
\end{definition}
In diesem fall heisst $A$ der Differential an der Stelle $x_0$ und wird mit $\mathop {df}\limits_{{x_0}} $ bezeichnet, d.h. $f$ ist total differenzierbar in $x_0=\left( x_0^1,\dots,x_0^n\right)$ falls reelle Zahlen $A_1,\dots,A_n$ existieren so dass gilt \[f(x) = f\left( {{x_0}} \right) + {A_1}\left( {{x^1} - x_0^1} \right) + {A_2}\left( {{x^2} - x_0^2} \right) +  \ldots  + {A_n}\left( {{x^n} - x_0^n} \right) + R\left( {x,{x_0}} \right)\] mit $\mathop {\lim }\limits_{x \to {x_0}} \frac{{R\left( {x,{x_0}} \right)}}{{\left| {x - {x_0}} \right|}} = 0$

\subsubsection*{Bemerkung: Geometrische Interpretation}
Sei $f:\Omega \rightarrow\mathbb{R}$, $\Omega\in\mathbb{R}^2$. Wir können die differenzierbare Funktion nähe dem Punkt $x_0=\left( x_0^1,x_0^2\right)$ mit hilfe der Lineare Funktion \[P\left( x \right) = P\left( {{x^1},{x^2}} \right) = f\left( {x_0^1,x_0^2} \right) + \underbrace {{A_1}\left( {{x^1} - x_0^1} \right) + {A_2}\left( {{x^2} - x_0^2} \right)}_{{d_x}_{_0}f\left( {x - {x_0}} \right)}\] approximieren. \\

Die Differenz $\underbrace {f(x) - P(x)}_{{d_x}_{_0}f\left( {x - {x_0}} \right)}\mathop  \to \limits_{x \to {x_0}} 0$\todo{can't understand what comes after the formula, page 126.1 middle} $P(x)$ ist eine Ebene. Die ist die Tangenteebene zur $f$ an der Stelle $x_0$ und spielt die Rolle des Tangente für Funktionen in einer Variable. \missingfigure{page 126.1 bottom}

\subsubsection*{Beispiel 8.5}
\begin{enumerate}[\indent a)]
\item Jede affin Lineare Funktion $f(x)=Ax+b$, $x\in\mathbb{R}^n$, wobei $a:\mathbb{R}^n\rightarrow\mathbb{R}$ linear, b$\in\mathbb{R}$ ist an jeder stelle $x_0\in\mathbb{R}^n$ differenzierbar, mit $\mathop {df}\limits_{{x_0}} =A$ unabhängig von $x_0$ da 
\[f\left( x \right) - f\left( {{x_0}} \right) - A\left( {x - {x_0}} \right) = 0\hspace{10mm}\forall x,{x_0} \in\mathbb{R}^n\]
\item Koordinaten funktionen $x^i:\mathbb{R}^n\rightarrow\mathbb{R}$, $\left( x^1,x^2,\dots,x^n\right)\rightarrow x^i$, $x^i(x)=x^i$. Dann ist $x^i$ differenzierbar an jeder Stelle $x_0\in\mathbb{R}^n$ mit $${\left. {d{x^i}} \right|_{x = {x_0}}} = \left( {0, \ldots ,0,1,0, \ldots ,0} \right)$$ die Differenziale $dx^1,dx^2,\dots,dx^n$ bilden also an jeder Stelle $x_0\in\mathbb{R}^n$ eine Basis des Raumes $L\left( {\mathbb{R}^n:\mathbb{R}} \right):=\left\{ A:\mathbb{R}^n\rightarrow\mathbb{R}; A\text{ linear}\right\}$, wobei wir $A\in L\left( \mathbb{R}^n:\mathbb{R}\right)$ mit der darstellung $A=\left( A_1,\dots,A_n\right)$ bzg. der Standardbasis $\{ e_1,\dots,e_n\}$ der $\mathbb{R}^n$ identifizieren, und mit $A_i=A\left( e_i\right)$ \[d{x^i} = \left( {0, \ldots ,0,1,0, \ldots ,0} \right)\]\[\left( {d{x^i}\left( {{e_1}} \right),d{x^i}\left( {{e_2}} \right), \ldots ,d{x^i}\left( {{e_n}} \right)} \right)\]
Da gilt $d{x^i}\left( {{e_j}} \right) = \left\{ {\begin{array}{*{20}{c}}
1&{i = j}\\
0&{i\not  = j}
\end{array}} \right.$ ist ${\left( {d{x^i}} \right)_{1 \le i \le n}}$ die duale Basis von $L\left( \mathbb{R}^n:\mathbb{R}\right)$ zur Standardbasis ${\left( {e_i} \right)_{1 \le i \le n}}$ des $\mathbb{R}^n$.
\item Jedes $f:\mathbb{R}\rightarrow\mathbb{R}\in\subset '\left(\mathbb{R}\right)$ besitzt das Differential \[df\left( {{x_0}} \right) = \frac{{df}}{{dx}}\left( {{x_0}} \right)dx = f'\left( {{x_0}} \right)dx\] d.h. $f'\left(x_0\right)$ ist die Darstellung von $df\left(x_0\right)$ bezüglich der Basis $dx$ von $L\left(\mathbb{R}:\mathbb{R}\right)$ 
\item $f\left(x,y\right)=xe^y$, $\mathbb{R}^2\rightarrow\mathbb{R}$ ist an jeder Stelle $\left(x_0,y_0\right)\in\mathbb{R}^2$ differenzierbar und es gilt \[df\left( {{x_0},{y_0}} \right) = \left( {\frac{{\partial f}}{{\partial x}}\left( {{x_0},{y_0}} \right),\frac{{\partial f}}{{\partial y}}\left( {{x_0},{y_0}} \right)} \right) = \left( {{e^{{y_0}}},x{e^{{y_0}}}} \right)\] \[f\left( {x,y} \right) - f\left( {{x_0},{y_0}} \right) = \underbrace {f\left( {x,y} \right) - f\left( {{x_0},y} \right)}_ \swarrow  + f\left( {{x_0},y} \right) - f\left( {{x_0},{y_0}} \right)\] \[ = \frac{{\partial f}}{{\partial x}}\left( {\xi ,y} \right)\left( {x - {x_0}} \right) + \frac{{\partial f}}{{\partial y}}\left( {{x_0},\eta } \right)\left( {y - {y_0}} \right)\]
Nach der MWS der DR, mit geeigneten Zwischenstellen $\xi=\xi (y)$ und $\eta$ \[ = \frac{{\partial f}}{{\partial x}}\left( {{x_0},{y_0}} \right)\left( {x - {x_0}} \right) + \frac{{\partial f}}{{\partial y}}\left( {{x_0},{y_0}} \right)\left( {y - {y_0}} \right) + R\left( {x,y} \right)\] 
mit
\begin{align*}
 R\left( {x,y} \right) = &\left[ {\frac{{\partial f}}{{\partial x}}\left( {\xi ,y} \right) - \frac{{\partial f}}{{\partial x}}\left( {{x_0},{y_0}} \right)} \right]\left( {x - {x_0}} \right) \\
 + &\left[ {\frac{{\partial f}}{{\partial y}}\left( {{x_0},\eta } \right) - \frac{{\partial f}}{{\partial y}}\left( {{x_0},{y_0}} \right)} \right]\left( {y - {y_0}} \right)
\end{align*}
Wegen die Stetigkeit der Funktionen \[\frac{{\partial f}}{{\partial x}}\left( {x,y} \right) = {e^y}\hspace{5mm}\text{und}\hspace{5mm}\frac{{\partial f}}{{\partial y}}\left( {x,y} \right) = x{e^y}\] können wir den ``Fehler'' $R\left( x,y\right)$ leicht abschätzen \[\frac{{\left| {R\left( {x,y} \right)} \right|}}{{\left| {\left( {x,y} \right) - \left( {{x_0},{y_0}} \right)} \right|}} \le \mathop {\sup }\limits_{\begin{array}{*{20}{c}}
{\left| {\xi  - {x_0}} \right| < \left| {x - {x_0}} \right|}\\
{\left| {\eta  - {y_0}} \right| < \left| {y - {y_0}} \right|}
\end{array}} \left( {\left| {{e^y} - {e^{{y_0}}}} \right| + \left| {{x_0}} \right|\left| {{e^\eta } - {e^{{y_0}}}} \right|} \right)\]
Für $\left( x,y\right)\rightarrow\left( x_0,y_0\right)$, $\left( x,y\right)\not=\left( x_0,y_0\right)$: d.h. es gilt \[\frac{{R\left( {x,y} \right)}}{{\left| {\left( {x,y} \right) - \left( {{x_0},{y_0}} \right)} \right|}} \to 0\] 
d.h. es gilt \[\frac{{f\left( {x,y} \right) - f\left( {{x_0},{y_0}} \right) - \frac{{\partial f}}{{\partial x}}\left( {{x_0},{y_0}} \right)\left( {x - {x_0}} \right) - \frac{{\partial f}}{{\partial y}}\left( {{x_0},{y_0}} \right)\left( {y - {y_0}} \right)}}{{\left| {\left( {x,y} \right) - \left( {{x_0},{y_0}} \right)} \right|}}\mathop  \to \limits_{\left( {x,y} \right) \to \left( {{x_0},{y_0}} \right)} 0\]
 d.h. $f\left( x,y\right)$ ist \todo{can't read, page 130 bottom} differenzierbar und 
\[df\left( {{x_0},{y_0}} \right) = \left( {\frac{{\partial f}}{{\partial x}}\left( {{x_0},{y_0}} \right),\frac{{\partial f}}{{\partial y}}\left( {{x_0},{y_0}} \right)} \right)\]
\item Die Funktion \[f\left( {x,y} \right) = \left\{ {\begin{array}{*{20}{c}}
{\frac{{{x^3}y}}{{{x^2} + {y^2}}}}&{\left( {x,y} \right)\not  = \left( {0,0} \right)}\\
0&{\left( {x,y} \right) = \left( {0,0} \right)}
\end{array}} \right.\] ist in $\left( 0,0\right)$ differenzierbar. \\

Wir haben schon gesehen dass $\frac{\partial f}{\partial x}\left( 0,0\right)=0$ und $\frac{\partial f}{\partial y}\left( 0,0\right)=0$. Dann gilt \[\frac{{\left| R \right|}}{{\left| {\left( {x,y} \right)} \right|}} = \frac{{\left| {f\left( {x,y} \right) - f\left( {0,0} \right) - \frac{{\partial f}}{{\partial x}}\left( {0,0} \right)\left( {x - 0} \right) - \frac{{\partial f}}{{\partial y}}\left( {0,0} \right)\left( {y - 0} \right)} \right|}}{{\left| {\left( {x - 0,y - 0} \right)} \right|}}\] \[ = \frac{{\left| {f\left( {x,y} \right) - 0 - 0 - 0} \right|}}{{\left| {\left( {x,y} \right)} \right|}} = \frac{{\left| {f\left( {x,y} \right)} \right|}}{{\left| {\left( {x,y} \right)} \right|}}\] 
Zum untersuchen ist 
\[\mathop {\lim }\limits_{\left( {x,y} \right) \to \left( {0,0} \right)}  \frac{{\left| {R\left( {\left( {x,y} \right),\left( {0,0} \right)} \right)} \right|}}{{\left( {x,y} \right) - \left( {0,0} \right)}} = \mathop {\lim }\limits_{\left( {x,y} \right) \to \left( {0,0} \right)} \frac{{\left| {f\left( {x,y} \right)} \right|}}{{\left| {\left( {x,y} \right)} \right|}}\]
Mittels Polarkoordinaten ist dies noch einsichtiger 
\[\mathop {\lim }\limits_{\left( {x,y} \right) \to \left( {0,0} \right)} \frac{{\left| {f\left( {x,y} \right)} \right|}}{{{x^2} + {y^2}}} = \mathop {\lim }\limits_{r \to 0} \frac{{{r^4}{{\cos }^3}\theta \sin \theta }}{{{r^2}}} = \mathop {\lim }\limits_{r \to 0} {r^2}{\cos ^3}\theta \sin \theta  = 0\] $$\Rightarrow f\text{ in }\left( 0,0\right)\text{ differenzierbar}$$
\end{enumerate}

\noindent Gibt es eine Beziehung zwischen des Differential und der partielle Ableitungen?

\subsubsection*{Bemerkung 8.6}

Sei $f:\Omega\rightarrow\mathbb{R}$, $\Omega\subset\mathbb{R}^n$ differenzierbar an der Stelle $x_0\in\Omega$. Dann existieren die partiellen Ableitungen $\frac{\partial f}{\partial x^i}\left( x_0\right)$, $i=1,\dots,n$ und dass Differential kann \[{d_{{y_0}}}f = \left( {\frac{{\partial f}}{{\partial {x^1}}}\left( {{x_0}} \right), \ldots ,\frac{{\partial f}}{{\partial {x^n}}}\left( {{x_0}} \right)} \right)\] dargestellt werden. 

\subsubsection*{Beweis}
$f$ an der Stelle $x_0$ differenzierbar \[\Rightarrow f\left( {{x_0} + h{e_i}} \right) = f\left( {{x_0}} \right) + \left( {{d_{{x_0}}}f} \right)\left( {h{e_i}} \right) + R\left( {{x_0} + h{e_i},{x_0}} \right)\] wobei \[\mathop {\lim }\limits_{h \to 0} \frac{{R\left( {{x_0} + h{e_i},{x_0}} \right)}}{h} = \lim \frac{{f\left( {{x_0} + h{e_i}} \right) - f\left( {{x_0}} \right)\left( {{d_{{x_0}}}f\left( {h{e_i}} \right)} \right)}}{h} = 0\] \[ \Rightarrow \lim \frac{{f\left( {{x_0} + h{e_i}} \right) - f\left( {{x_0}} \right)}}{h} = \lim \frac{{h{d_{{x_0}}}f\left( {{e_i}} \right)}}{h} = {d_{{x_0}}}f\left( {{e_i}} \right)\] d.h. $\frac{\partial f}{\partial x^i}\left( x_0\right)$ existiert und $=d_{x_0}f\left( e_i\right)$. \\

Da $\left( dx^i\right)_{i=1,\dots,n}$ die zur $\left( e_j\right)_{1\leq j\leq n}$ duale Basis ist \[{d_{{x_0}}}f = \sum\limits_{i = 1}^n {\frac{{\partial f}}{{\partial {x^i}}}\left( {{x_0}} \right)d{x^i} = \left( {\frac{{\partial f}}{{\partial {x^1}}}\left( {{x_0}} \right),\frac{{\partial f}}{{\partial {x^2}}}\left( {{x_0}} \right), \ldots ,\frac{{\partial f}}{{\partial {x^n}}}\left( {{x_0}} \right)} \right)} \]

\subsubsection*{Beispiel}
Die Funktion \[f\left( {x,y} \right) = \left\{ {\begin{array}{*{20}{c}}
{\frac{{xy}}{{{x^2} + {y^2}}}}&{\left( {x,y} \right)\not  = \left( {0,0} \right)}\\
0&{\left( {x,y} \right) = \left( {0,0} \right)}
\end{array}} \right.\] ist in $\left( 0,0\right)$ nicht differenzierbar ($f$ ist  in $\left( 0,0\right)$ nicht stetig)

\subsubsection*{Satz 8.7}
Falls $f:\Omega\rightarrow\mathbb{R}$ in $x_0\in\Omega\subset\mathbb{R}$ differenzierbar, ist sie in $x_0$ auch stetig.

\subsubsection*{Beweis}
Folgt aus der Definition
\begin{definition}{8.8}
$f:\Omega\rightarrow\mathbb{R}$ heisst von der Klasse $C'$, $\left( f\in C'\left(\Omega\right)\right)$ falls $f$ an jeder Stelle $x_0\in\Omega$ und in jede Richtung $e_i$ partielle differenzierbar ist und die Funktionen $x\rightarrow\frac{\partial f}{\partial x^i}\left(x\right)$ für jedes $1\leq i\leq n$ auf $\Omega$ stetig sind
\end{definition}
\subsubsection*{Satz 8.9}
Sei $f\in C'\left(\Omega\right)$. Dann ist $f$ an jeder Stelle $x_0\in\Omega$ differenzierbar. 

\subsubsection*{Beweis}
Für $n=3$ seien $x=\left( x^1,x^2,x^3\right)$, $x_0=\left( x_0^1,x_0^2,x_0^3\right)$. Dann ist 
\begin{align*}
f\left( x \right) - f\left( {{x_0}} \right) = &\left\{ {f\left( {{x^1},{x^2},{x^3}} \right) - f\left( {{x^1},{x^2},x_0^3} \right)} \right\}\\
+ &\left\{ {f\left( {{x^1},{x^2},x_0^3} \right) - f\left( {{x^1},x_0^2,x_0^3} \right)} \right\}\\
+ &\left\{ {f\left( {{x^1},x_0^2,x_0^3} \right) - f\left( {x_0^1,x_0^2,x_0^3} \right)} \right\}
\end{align*}
Nach dem MWS der DR gilt: \[f\left( {{x^1},x_0^2,x_0^3} \right) - f\left( {x_0^1,x_0^2,x_0^3} \right) = \frac{{\partial f}}{{\partial {x^1}}}\left( {{\xi ^1},x_0^2,x_0^3} \right)\left( x^1-x_0^1\right)\] 
wobei $\xi^1$ zwischen $x_0^1$ und $x^1$. Analog: 
\[f\left( {{x^1},{x^2},x_0^3} \right) - f\left( {{x^1},x_0^2,x_0^3} \right) = \frac{{\partial f}}{{\partial {x^2}}}\left( {{x^1},{\xi ^2},x_0^3} \right)\left( {{x^2} - x_0^2} \right)\] wobei $\xi^2\in\left(x_0^2,x^2\right)$ 
und \[f\left( {{x^1},{x^2},{x^3}} \right) - f\left( {{x^1},{x^2},x_0^3} \right) = \frac{{\partial f}}{{\partial {x^3}}}\left( {{x^1},{x^2},{\xi ^3}} \right)\left( {{x^3} - x_0^3} \right)\]
Eingesetz in dem Ausdrucke für $f(x)-f(x_0)$ ergibt: 
\begin{align*}
f\left( x \right) - f\left( {{x_0}} \right) = &\frac{{\partial f}}{{\partial {x^1}}}\left( {{\xi ^1},x_0^2,x_0^3} \right)\left( {{x^1} - x_0^1} \right)\\
+ &\frac{{\partial f}}{{\partial {x^2}}}\left( {{x^1},{\xi ^2},x_0^3} \right)\left( {{x^2} - x_0^2} \right)\\ + &\frac{{\partial f}}{{\partial {x^3}}}\left( {{x^1},{x^2},{\xi ^3}} \right)\left( {{x^3} - x_0^3} \right)
\end{align*}
Also
\[f\left( x \right) - f\left( {{x_0}} \right) = \sum\limits_{i = 1}^3 {\frac{{\partial f}}{{\partial {x^1}}}\left( {x_0^1,x_0^2,x_0^3} \right)\left( {{x^i} - x_0^i} \right)}  + R\left( {{x_0},x} \right)\]
Wobei 
\begin{align*}
R\left( {{x_0},x} \right) = &\left( {\frac{{\partial f}}{{\partial {x^1}}}\left( {{\xi ^1},x_0^2,x_0^3} \right) - \frac{{\partial f}}{{\partial {x^1}}}\left( {x_0^1,x_0^2,x_0^3} \right)} \right)\left( {{x^1} - x_0^1} \right)\\
 + &\left( {\frac{{\partial f}}{{\partial {x^2}}}\left( {{x^1},{\xi ^2},x_0^3} \right) - \frac{{\partial f}}{{\partial {x^2}}}\left( {x_0^1,x_0^2,x_0^3} \right)} \right)\left( {{x^2} - x_0^2} \right)\\
 + &\left( {\frac{{\partial f}}{{\partial {x^3}}}\left( {{x^1},{x^2},{\xi ^3}} \right) - \frac{{\partial f}}{{\partial {x^3}}}\left( {x_0^1,x_0^2,x_0^3} \right)} \right)\left( {{x^3} - x_0^3} \right)
\end{align*}
Also
\[\left| {R\left( {x,{x_0}} \right)} \right| < \left| {x - {x_0}} \right|\underbrace {\left\{ {\left| {\left(  \ldots  \right)} \right| + \left| {\left(  \ldots  \right)} \right| + \left| {\left(  \ldots  \right)} \right|} \right\}}_{\scriptstyle \to 0{\text{ mit }}x \to {x_0}\hfill\atop
\scriptstyle{\text{weil }}\frac{{\partial f}}{{\partial {x^i}}}{\text{ stetig sind}}\hfill}\]
Also $\lim\frac{R\left( x,x_0\right)}{\left( x-x_0\right)}=0$ und $f(x)$ ist differenzierbar.

\subsubsection*{Beispiel 8.10}
Polynome auf $\mathbb{R}^n$ sind von Klasse $C!$. Für jedes Multindex $\alpha=\left(\alpha_0,\dots , \alpha_n\right)\in\mathbb{N}^n$ definieren wir die Monomialfunktion \[{x^\alpha }: = {\left( {{x^0}} \right)^{{\alpha _0}}}{\left( {{x^1}} \right)^{{\alpha _1}}} \ldots {\left( {{x^n}} \right)^{{\alpha _n}}}\] Ein polynom von Grad $\leq N$ ist dann gegeben durch \[p(x) = \sum\limits_{\left| \alpha  \right| \le N} {{a_\alpha }{x^\alpha }} \] wobei $\left| \alpha  \right| = {\alpha _0} +  \ldots  + {\alpha _n}$
\todo[inline]{Pages 135.1 - 135.2 are a zusammenfassung, not sure if needed to be included}
\section{Differentiationsregeln}
Ganz analog zum eindimensionalen Fall gelten folgende Differentiationsregeln

\subsubsection*{Satz 8.11}
Sei $\Omega\subset\mathbb{R}^n$, sowie $f,g:\Omega\rightarrow\mathbb{R}$ an der Stelle $x_0\in\Omega$ differenzierbar. Dann gilt \begin{enumerate}
\item $d\left( {f + g} \right)\left( {{x_0}} \right) = df\left( {{x_0}} \right) + dg\left( {{x_0}} \right)$
\item $d\left( {fg} \right)\left( {{x_0}} \right) = g\left( {{x_0}} \right)df\left( {{x_0}} \right) + f\left( {{x_0}} \right)dg\left( {{x_0}} \right)$
\item Falls $g\left( x_0\right)\not=0$ \[d\left( {\frac{f}{g}} \right)\left( {{x_0}} \right) = \frac{{g\left( {{x_0}} \right)df\left( {{x_0}} \right) - f\left( {{x_0}} \right)dg\left( {{x_0}} \right)}}{{{{\left( {g\left( {{x_0}} \right)} \right)}^2}}}\]
\end{enumerate}
Beweis ist der selbe wie in Dim=1. Für die Kettenregel gibt es mehrere variationen 

\subsubsection*{Satz 8.12 (Kettenregel, 1. Version)}
Sei $g:\Omega\rightarrow\mathbb{R}$ in $x_0\in\Omega\subset\mathbb{R}^n$ differenzierbar, sowie $f\mathbb{R}\rightarrow\mathbb{R}$ an der stelle $g\left( x_0\right)\in\mathbb{R}$ differenzierbar. Dann gilt \[d\left( {f \circ g} \right)\left( {{x_0}} \right) = f'\left( {g\left( {{x_0}} \right)} \right) \cdot dg\left( {{x_0}} \right)\]

\missingfigure{page 137, middle}

\subsubsection*{Beweis}
$g$ an der Stelle $x_0$ differenzierbar 
\[ \Rightarrow g\left( x \right) - g\left( {{x_0}} \right)\mathop  = \limits^A dg\left( {{x_0}} \right)\left( {x - {x_0}} \right) + {R_g}\left( {x - {x_0}} \right)\]
mit \[\frac{{{R_g}\left( {x - {x_0}} \right)}}{{\left( {x - {x_0}} \right)}}\mathop  \to \limits_{x \to {x_0}} 0 \Rightarrow \frac{{g\left( x \right) - g\left( {{x_0}} \right)}}{{\left| {x - {x_0}} \right|}}\mathop  \le \limits^B C = \max \left[ {\frac{{\partial g}}{{\partial {x^i}}}\left( {{x_0}} \right)} \right]\] $f$ in $g\left( x_0\right)$ differenzierbar 
\[f\left( {g\left( x \right)} \right) - f\left( {g\left( {{x_0}} \right)} \right)\mathop  = \limits^C f'\left( {g\left( {{x_0}} \right)} \right)\left[ {g(x) - g\left( {{x_0}} \right)} \right] + {R_f}\left( {g\left( x \right),g\left( {{x_0}} \right)} \right)\]
Woraus folgt:
\begin{align*}
f\left( {g\left( x \right)} \right) - f\left( {g\left( {{x_0}} \right)} \right) = & f'\left( {g\left( {{x_0}} \right)} \right)\left[ {dg\left( {{x_0}} \right)\left( {x - {x_0}} \right) + {R_g}\left( {x,{x_0}} \right)} \right]\\ + & {R_f}\left( {g\left( {{x_0}} \right),g\left( x \right)} \right)
\end{align*}

Aus B folgt:
\[\frac{{{R_f}\left( {g\left( {{x_0}} \right),g\left( x \right)} \right)}}{{x - {x_0}}} = \underbrace {\underbrace {\frac{{{R_f}\left( {g\left( {{x_0}} \right) - g\left( x \right)} \right)}}{{\left| {g\left( x \right) - g\left( {{x_0}} \right)} \right|}}}_{\begin{array}{*{20}{c}}
 \downarrow \\
0
\end{array}} \cdot \underbrace {\frac{{\left| {g\left( x \right) - g\left( {{x_0}} \right)} \right|}}{{\left| {x - {x_0}} \right|}}}_{\mathop  \le \limits^B C}}_{\begin{array}{*{20}{c}}
 \downarrow \\
0
\end{array}}\]
d.h.
\[f\left( {g\left( x \right)} \right) - f\left( {g\left( {{x_0}} \right)} \right) = \left( {f'\left( {g\left( {{x_0}} \right)} \right) \cdot dg\left( {{x_0}} \right)} \right)\left( {x - {x_0}} \right) + {R_{f \circ g}}\left( {x,{x_0}} \right)\]
wobei
\[{R_{f \circ g}}\left( {x,{x_0}} \right) = f'\left( {g\left( {{x_0}} \right)} \right){R_g}\left( {x,{x_0}} \right) + {R_f}\left( {g\left( {{x_0}} \right),g\left( x \right)} \right)\]
und 
\[\frac{{{R_{f \circ g}}\left( {x,{x_0}} \right)}}{{x - {x_0}}} = \underbrace {f'\left( {g\left( {{x_0}} \right)} \right)\frac{{{R_g}\left( {x,{x_0}} \right)}}{{\left( {x - {x_0}} \right)}}}_{ \downarrow 0} + \underbrace {\frac{{{R_f}\left( {g\left( {{x_0}} \right),g\left( x \right)} \right)}}{{x - {x_0}}}}_{ \downarrow 0}\]

\subsubsection*{Beispiel 8.13}
Sei $h:\mathbb{R}^2\rightarrow\mathbb{R}$ $$h(x,y)=e^{xy}$$ $h= f\circ g$ wobei $g(x,y)=xy$, $f(t)=e^t$. Dann ist einerseits \[dh\left( {x,y} \right) = \left( {\frac{{\partial h}}{{\partial x}},\frac{{\partial h}}{{\partial y}}} \right) = \left( {y{e^{xy}},x{e^{xy}}} \right)\] anderseits nach Kettenregel 
\[dh\left( {x,y} \right) = d\left( {f \circ g} \right)' = f'\left( {g\left( {x,y} \right)} \right) \cdot dg\left( {x,y} \right) = {e^{xy}} \cdot \left( {y,x} \right) = \left( {y{e^{xy}},x{e^{xy}}} \right)\]
Für die nächste Kettenregel führen wir folgende Definition ein:

\begin{definition}{8.14}
Sei $\Omega \subset\mathbb{R}$ und $f=\left( f_1,\dots,f_n\right):\Omega\rightarrow\mathbb{R}^n$ eine Abbildung. Dann ist $f$ an der Stelle $x_0\in\mathbb{R}$ differenzierbar, falls jede Komponentenfunktion $f_i$ an der Stelle $x_0$ differenzierbar ist. Wir definieren in diesem Fall \[f'\left( {{x_0}} \right): = \left( {{f_1}'\left( {{x_0}} \right),{f_2}'\left( {{x_0}} \right), \ldots ,{f_n}'\left( {{x_0}} \right)} \right)\]
\end{definition}

\subsubsection*{Bemerkung 8.15}
$f'\left( x_0\right)$ kann als Geschwindigkeitsvektor im Punkt $f\left( x_0\right)$ aufgefasst werden.

\subsubsection*{Satz 8.16 (Kettenregel, 2. Version)}
Sei $\Omega\subset\mathbb{R}^n$,$I\subset\mathbb{R}$. Sei $g:I\rightarrow\Omega$, $t\rightarrow\left( g_1(t), g_2(t),\dots, g_n(t)\right)$, an der Stelle $t_0\in I$ differenzierbar sowie $f:\Omega\rightarrow\mathbb{R}$ an der Stelle $g\left( t_0\right)$ differenzierbar. Dann gilt:
\[\frac{d}{{dt}}\left( {f \circ g} \right)\left( {{t_0}} \right) = df\left( {g\left( {{t_0}} \right)} \right) \cdot g'\left( {{t_0}} \right)\]
\begin{align*}
d\left( {f \circ g} \right)\left( {{t_0}} \right) = &df\left( {g\left( {{t_0}} \right)} \right) \cdot dg\left( {{t_0}} \right)\\
 = &\frac{{\partial f}}{{\partial {x^1}}}\left( {g\left( {{t_0}} \right)} \right) \cdot \frac{{d{g_1}}}{{dt}}\left( {{t_0}} \right) + \frac{{\partial f}}{{\partial {x^2}}}\left( {g\left( {{t_0}} \right)} \right) \cdot \frac{{d{g_2}}}{{dt}}\left( {{t_0}} \right)\\ & + \ldots  + \frac{{\partial f}}{{\partial {x^n}}}\left( {g\left( {{t_0}} \right)} \right) \cdot \frac{{d{g_n}}}{{dt}}\left( {{t_0}} \right)
\end{align*}
\missingfigure{page 141, middle}

\subsubsection*{Beispiel 8.17}
Die vier Grundrechenarten sind differenzierbare Funktionen von zwei variablen. Insbesondere gilt: 
\begin{itemize}
\item $a:\mathbb{R}^2\rightarrow\mathbb{R}$, $\left( x,y\right) \rightarrow x+y$ \[da\left( {x,y} \right) = \left( {\frac{{\partial a}}{{\partial x}},\frac{{\partial a}}{{\partial y}}} \right) = \left( {1,1} \right)\]
\item $m:\mathbb{R}^2\rightarrow \mathbb{R}$, $\left( x,y\right) \rightarrow x\cdot y$ \[dm\left( {x,y} \right) = \left( {y,x} \right)\]
\end{itemize}
Setzt man diese beiden Funktionen in die obige Kettenregel ein, so erhält man die aus der Analysis I bekannte Summen und Produktregel: 
$$g:\mathbb{R}\rightarrow\mathbb{R}^2, t\rightarrow \left( g_1(t),g_2(t)\right)$$
\[\frac{d}{{dt}}\left( {{g_1} + {g_2}} \right) = \frac{d}{{dt}}\left( {a \circ g} \right) = \left( {1,1} \right) \cdot \left( {\frac{{d{g_1}}}{{dt}},\frac{{d{g_2}}}{{dt}}} \right) = 1\cdot \frac{{d{g_1}}}{{dt}} + 1\cdot \frac{{d{g_2}}}{{dt}}\]
und 
\begin{align*}
\frac{d}{{dt}}\left( {{g_1} \cdot {g_2}} \right) = \frac{d}{{dt}}\left( {m \circ g} \right) = &\left( {\left( {dm} \right)\left( {g(t)} \right)} \right) \cdot \left( {\frac{{dg}}{{dt}}} \right)\\ 
= &\left( {{g_2}\left( t \right),{g_1}\left( t \right)} \right) \cdot \left( {\frac{{d{g_1}}}{{dt}},\frac{{d{g_2}}}{{dt}}} \right)\\ 
= &\frac{{d{g_1}}}{{dt}} \cdot {g_2}\left( t \right) + \frac{{d{g_2}}}{{dt}} \cdot {g_1}\left( t \right)
\end{align*}

\subsubsection*{Beispiel 8.18}
Sei $f:\Omega\rightarrow\mathbb{R}$ differenzierbar an der Stelle $x_0\in\Omega$ und sei $e\in\mathbb{R}^n\backslash\{ 0\}$; mit $\left| e\right|=1$. Betrachte die Gerade $g(t)=x_0+te$, $t\in\mathbb{R}$ durch $x_0$ mit Richtungsvektor 

\missingfigure{page 143, top. ALSO ADD THE FORMULAS ON THE RIGHT PAGE USING A MINIPAGE}

Dann ist die Funktion $f\circ g$ in einer Umgebung von $t_0=0$ definiert und nach Kettenregel $f\circ g$ an der Stelle $t_0=0$ differenzierbar mit 
\[\frac{d}{{dt}}\left( {f \circ g} \right)\left( 0 \right) = df\left( {g\left( 0 \right)} \right)\frac{{dg}}{{dt}}\left( 0 \right) = df\left( {{x_0}} \right)\left( e \right) = \sum\limits_{i = 1}^n {\frac{{\partial f}}{{\partial {x^i}}}\left( {{x_0}} \right) \cdot {e^i}} \]
$e=\left( e^1,\dots, e^n\right) $ und wird Richtungsableitung von $f$ in Richtung $e$ genannt; $\partial_ef\left( x_0\right)$ bezeichnet. Insbesondere gilt für $e=e_i$
\[{\partial _{{e_i}}}f\left( {{x_0}} \right) = \frac{{\partial f}}{{\partial {x^i}}}\left( {{x_0}} \right) = df\left( {{x_0}} \right)\left( {{e_i}} \right)\]
Geometrisch die Richtungsableitung von $f$ in Richtung $e$ ist genau die Steigung der Tangente zur Schnittkurve falls wir den Graph von $f$ mit einer zur Ebene $xy$ senkrecht Ebene durch $x_0+te$ scheiden.
\missingfigure{page 144, middle}
Für den Mittelwertsatz der DR - zu verallgemeinen benützen wir folgenden Begriffen:
\begin{definition}{8.19}
Eine Menge $K\subset \mathbb{R}^n$ ist genau dann Konvex falls für jede Paar von Punkten $x,y\in K$ die Menge $K$ auch das segment $$\left( 1-t\right) x+ty\hspace{10mm} t\in\lbrack 0,y\rbrack$$ mit endpunkt $x,y$ enthält
\missingfigure{Page 145, middle}
\end{definition}
\subsubsection*{Satz 8.20}
Sei $\Omega\subset\mathbb{R}^n$ konvex $f:\Omega\rightarrow\mathbb{R}$ differenzierbar, $x_0,x_1\in\Omega$ sowie $x_t:=\left( 1-t\right) x_0+tx_1$\todo{is it $tx_1$ or $tx$, ?? page 145 middle}. Dann gibt es $\vartheta\in\lbrack 0,1\rbrack$ mit \[f\left( {{x_1}} \right) - f\left( {{x_0}} \right) = df\left( {{x_{i\vartheta }}} \right)\left( {{x_1} - {x_0}} \right)\]

\subsubsection*{Beweis}
Sei $g\left( t\right) = \left( 1-t\right) x_0+tx_1=x_t$. Dann ist $t\rightarrow \left( f\circ g\right)(t)$ auf $\left[ 0,1\right]$ stetig und in $\left( 0,1\right)$ differenzierbar. Also gibt es $\vartheta\in\left( 0,1\right)$ mit (nach MWS der DS einer variable) \[f\left( {{x_i}} \right) - f\left( {{x_0}} \right) = \left( {f \circ g} \right)(1) - \left( {f \circ g} \right)(0) = \left( {f \circ g} \right)'\left( \vartheta  \right)\left( {1 - 0} \right)\] Nur ist \[\left( {f \circ g} \right)'\left( \vartheta  \right) = df\left( {g\left( \vartheta  \right) \cdot \frac{{dg}}{{dt}}\left( \vartheta  \right)} \right)\] \todo{Is the formula done or does it continue on a new line, page 146 top}
%\[\left( {df} \right)\left( {{x_\vartheta }} \right)\left( { - {x_0} + {x_1}} \right)\]
Die Kettenregel wird auch angewandelt um Integrale mit Parametern zu studieren. Ein Beispiel davon ist:

\subsubsection*{Beispiel}
Sei $h:\mathbb{R}^2\rightarrow\mathbb{R}$, $\left( s,t\right)\rightarrow h\left(s,t \right)$. Wir nehmen an, $h$ ist stetig, $\frac{\partial h}{\partial t}$ existiert und ist uf ganz $\mathbb{R}^2$ stetig. Sei 
\[u(t) = \int\limits_a^{b(t)} {h(s,t)ds} {\text{, }}b(t) \in\subset '\left( \mathbb{R} \right){\text{, }}a \in\mathbb{R}\]\todo{Is it $C'$ or $\subset '$?? page 146 middle}

\subsubsection*{Satz 8.21}
Sei $h(s,t)$ eine stetige differenzierbare Funktion von zwei variablen und $b(t)$ differenzierbare Funktion eine variable. Dann ist die Funktion \[u(t): = \int\limits_a^{b(t)} {h(s,t)ds} \] wo definiert, differenzierbar mit der Ableitung \[u'(t): = h\left( {b(t),t} \right) \cdot b'(t) + \int\limits_a^{b(t)} {\frac{{\partial h}}{{\partial t}}\left( {s,t} \right)ds} \]
\subsubsection*{Korollar 8.22}
Sei $h=h\left( s,t\right) :\mathbb{R}^2\rightarrow\mathbb{R}$ stetig, und $\frac{\partial h}{\partial t}$ existiert und auf ganz $\mathbb{R}^2$ stetig. Sei \[u(t) = \int\limits_0^t {h\left( {s,t} \right)ds} \] Dann \[u(t) \in  \subset '\left( \mathbb{R}\right)\text{ und }u'(t) = h\left( {t,t} \right) + \int\limits_0^t {\frac{{\partial h}}{{\partial t}}\left( {s,t} \right)ds} \]

\subsubsection*{Beweis}
Setze $b(t)=t$, $a=0$ in Satz 8.21.

\subsubsection*{Korollar 8.23}
Sei $h:\mathbb{R}^2\rightarrow\mathbb{R}$ eine stetige Funktion mit Stetiger partieller Ableitung $\frac{\partial h}{\partial t}$. Dann ist die Funktion \[u(t): = \int\limits_a^b {h(s,t)ds} \] differenzierbar mit Ableitung \[u'(t): = \int\limits_a^b {\frac{{\partial h}}{{\partial t}}(s,t)ds} \] 

\subsubsection*{Beweis}
Setze $b(t)=b$, in Satz 8.20

\subsubsection*{Bemerkung 8.24}
Mit Korollar 8.23 kann man bestimmte Integrale berechnen, auch wenn die zugehörige unbestimmten Integrale nicht elementar darstellbar sind
\subsubsection*{Beispiel 8.25}
Berechne das integral \[\int\limits_0^1 {\frac{{{x^5} - 1}}{{\log x}}} dx\] Sei \[u\left( \alpha  \right): = \int\limits_0^1 {\frac{{{x^\alpha } - 1}}{{\log x}}dx} \] Für $\alpha \geq 0$ erfüllt $u\left(\alpha\right)$ die Voraussetzungen des Satzes. Wir berechnen \[u'\left( \alpha  \right) = \int\limits_0^1 {\frac{\partial }{{\partial \alpha }}\left( {\frac{{{x^\alpha } - 1}}{{\log x}}} \right)dx}  = \int\limits_0^1 {\frac{{{x^\alpha }\log x}}{{\log x}}dx = \int\limits_0^1 {{x^\alpha }dx}  = \left. {\frac{{{x^{\alpha  + 1}}}}{{\alpha  + 1}}} \right|_0^{x = 1} = } \frac{1}{{\alpha  + 1}}\] 
Daraus folgt aus Fundamentales Satz der Integral Rechnung \[u\left( \alpha  \right) = \int {u'\left( \alpha  \right)d\alpha  = \int {\frac{{d\alpha }}{{\alpha  + 1}} = \log \left( {\alpha  + 1} \right)} }  + C\]
Für eine noch zu bestimmende Konstante $C$. Aber
\[u\left( 0 \right) = \int\limits_0^1 {0dx = 0 \Rightarrow C = 0} \]
\[ \Rightarrow \int\limits_0^1 {\frac{{{x^\alpha } - 1}}{{\log x}}dx = \log \left( {\alpha  + 1} \right)} \]
\[ \Rightarrow \int\limits_0^1 {\frac{{{x^5} - 1}}{{\log x}}dx = \log 6} \]

\subsubsection*{Beweis Satz 8.21 (Idee)}
Sei \[f\left( {x,y} \right) = \int\limits_a^x {h\left( {s,y} \right)ds}:\mathbb{R}^2\rightarrow\mathbb{R} \]
\[g\left( t \right) = \left( {\begin{array}{*{20}{c}}
{b\left( t \right)}\\
t
\end{array}} \right):\mathbb{R}\rightarrow\mathbb{R}^2\text{, }g'\left( t \right) = \left( {\begin{array}{*{20}{c}}
{b'\left( t \right)}\\
1
\end{array}} \right)\]
Dann \[u(t) = \left( {f \circ g} \right)(t) = f\left( {b(t),t} \right) = \int\limits_a^{b(t)} {h(s,t)ds} \]
Nach Hauptsatz der Integral Rechnung $f$ ist nach $x$ partielle differenzierbar und $\frac{\partial f}{\partial x}=h(x,y)$. Man muss zeigen das $f$ ist nach $y$ partielle differenzierbar mit \[\frac{{\partial f}}{{\partial y}}(x,y) = \int\limits_a^x {\frac{{\partial h(s,y)}}{{\partial x}}ds} \]
Dann ergibt die Kettenregel
\begin{align*}
u'(t) = &\frac{d}{{dt}}\left( {f \circ g} \right)(t) = \left( {\frac{{\partial f}}{{\partial x}}\left( {g(t)} \right),\frac{{\partial f}}{{\partial y}}\left( {g(t)} \right)} \right) \cdot \frac{{dg}}{{dt}}\\
= &\left( {h\left( {b(t),t} \right),\left( {\int\limits_a^x {\frac{{\partial h(s,y)}}{{\partial y}}ds} } \right)h\left( {b(t),t} \right)} \right)\left( {\begin{array}{*{20}{c}}
{b'(t)}\\
1
\end{array}} \right)\\
= &\left( {h\left( {b(t),t} \right),\int\limits_a^{b(t)} {\frac{{\partial h}}{{\partial y}}(s,t)ds} } \right)\left( {\begin{array}{*{20}{c}}
{b'(t)}\\
1
\end{array}} \right)\\
= & h\left( {b(t),t} \right) \cdot b'(t) + \int\limits_a^{b(t)} {\frac{{\partial h}}{{\partial t}}(s,t)ds}
\end{align*}
\section{Differentialformen und Vektorfelder}
Sei $L\left(\mathbb{R}^n,\mathbb{R}\right)$ der Raum der linearen Abbildungen von $\mathbb{R}^n$ nach $\mathbb{R}$. Falls $f:\Omega\subset\mathbb{R}^n\rightarrow\mathbb{R}$ eine Funktion ist, die in jedem Punkt differenzierbar ist, dann ist $df(x)\in L\left(\mathbb{R}^n,\mathbb{R}\right)$ und man erhält eine Abbildung
\begin{align*}
\Omega &\rightarrow L\left(\mathbb{R}^n,\mathbb{R}\right)\\
x_0 &\rightarrow df\left(x_0\right) = \left( {\frac{{\partial f}}{{\partial {x^1}}}\left( {{x_0}} \right), \ldots ,\frac{{\partial f}}{{\partial {x^n}}}\left( {{x_0}} \right)} \right)\\
& \hspace{16mm}=  \frac{{\partial f}}{{\partial {x^1}}}\left( {{x_0}} \right)d{x^1} +  \ldots  + \frac{{\partial f}}{{\partial {x^n}}}\left( {{x_0}} \right)d{x^n}
\end{align*}
Dies ist ein Beispiel von $1-$form

\begin{definition}{8.26}
Eine Differentialform vom Grad 1 (auch ``$1-$Form'') auf $\Omega$ ist eine Abbildung $$\lambda :\Omega\rightarrow L\left(\mathbb{R}^n,\mathbb{R}\right)$$
\end{definition}
\subsubsection*{Beispiel 8.27}
\begin{enumerate}
\item Seien $x^i:\mathbb{R}^n\rightarrow\mathbb{R}$ die Koordinatenfunktionen $1\leq i\leq n$. Für jedes $x_0\in\mathbb{R}^n$ ist $dx^i\left(x_0\right)\in L\left( \mathbb{R}^n,\mathbb{R}\right)$; dies führt zur $1-$Form
\begin{align*}
dx^i:\mathbb{R}^n&\rightarrow L\left( \mathbb{R}^n,\mathbb{R}\right)\\
x_0 &\rightarrow dx^i\left( x_0\right)
\end{align*}
Für jedes $x_0\in\mathbb{R}^n$, gilt $dx^i\left(e_j\right)=\delta_{ij}$ also bilden $dx^1\left( x_0\right),\dots,dx^n\left( x_0\right)$ eine Basis für $L\left(\mathbb{R}^n,\mathbb{R}\right)$, $\forall x\in\mathbb{R}^n$.\\

Eine beliebig $1-$Form $\lambda :\mathbb{R}^n\rightarrow L\left(\mathbb{R}^n,\mathbb{R}\right)$ lässt sich dann eindeutig wie folgt darstellen \[\lambda \left( {{x_0}} \right) = \sum\limits_{i = 1}^n {{\lambda _i}\left( {{x_0}} \right)d{x^i}\left( {{x_0}} \right)} \] wobei $\lambda_i:\mathbb{R}^n\rightarrow\mathbb{R}$ Funktionen sind.
\item Für jedes $f\in\subset'\left(\Omega\right)$\todo{Is it $C'$ or $\subset'$?? page 152.1 top} ist das differential $df$ eine $1-Form$ \[df = \frac{{\partial f}}{{\partial {x^1}}}d{x^1} + \frac{{\partial f}}{{\partial {x^2}}}d{x^2} +  \ldots  + \frac{{\partial f}}{{\partial {x^n}}}d{x^n}\]
\item Der Ausdrück $\lambda\left(x,y,z\right)=3dx+5zdy+xdz$ definiert ein $1-$Form auf $\mathbb{R}^3$ mit 
\begin{align*}
& \lambda_1\left(x,y,z\right)=3\\
& \lambda_2\left(x,y,z\right)=5z\\
& \lambda_3\left(x,y,z\right)=x
\end{align*}
\end{enumerate}
\begin{definition}{8.28}
Ein Vektorfeld auf $\Omega\subset\mathbb{R}^n$ ist eine Abbildung $v:\Omega\rightarrow\mathbb{R}^n$
\end{definition}\todo[inline]{Does the definition include the examples? page 153 top}

\subsubsection*{Beispiel}
\begin{enumerate}
\item \begin{align*}
v:\mathbb{R}^2 &\rightarrow\mathbb{R}^2\\
\left( x,y\right) &\rightarrow\left( 2xy,x^2\right)
\end{align*}
\missingfigure{page 153, middle}
\item $v\left( x,y\right) = \left( -y,x\right)$
\missingfigure{Page 153, bottom}
\end{enumerate}
\subsubsection*{Bemerkung 8.29}
Sei $<,>$ das übliche Skalarprodukt auf $\mathbb{R}^n$, d.h. \[ < x,y > : = \sum\limits_{i = 1}^n {{x^i}{y^i}} \]
Mittels $<,>$ kann man von $1-$Formen zu Vektorfelder und umgekehrt übergehen. Dies geht wie folgt:
\begin{enumerate}
\item Sei $v:\Omega\rightarrow\mathbb{R}^n$ ein Vektorfeld. Dann definieren wir $\forall x\in\Omega$, $\omega\in\mathbb{R}^n$
$$\lambda (x)(\omega):= < v(x),\omega >$$ Offensichtlich $\lambda (x)\in\left( \mathbb{R}^n, \mathbb{R}\right)$ und somit ist 
\begin{align*}
\lambda : \Omega &\rightarrow L\left(\mathbb{R}^n, \mathbb{R}\right) \\
x &\rightarrow\lambda (x): \mathbb{R}^n \rightarrow\mathbb{R}\\
&\hspace{17.5mm}\omega \rightarrow <v(x),\omega >
\end{align*}
eine $1-$Form auf $\Omega$
\end{enumerate}
\underline{Umgekehrt}
\begin{enumerate}[\indent 2.]
\item Sei $\lambda :\Omega\rightarrow L\left(\mathbb{R}^n,\mathbb{R}\right)$ $1-$Form und $\lambda (x): = \sum\limits_{i = 1}^n {{\lambda _i}(x)d{x^i}} $ wie oben.\\

Wir definieren 
\begin{align*}
v:\Omega &\rightarrow\mathbb{R}^n\\
x &\rightarrow \left( \lambda _1(x),\lambda _2(x),\dots,\lambda _n(x) \right)
\end{align*}
dann ist $v$ ein Vektorfeld und $$\lambda (x)(\omega)= < v(x),\omega>$$ Sei $\omega = \omega^1e_1+\omega^2 e_2 + \dots + \omega^n e_n$. Dann
\begin{align*}
\lambda (x)(\omega ) = & \sum\limits_{i = 1}^n {{\lambda _i}(x)d{x^i}(\omega )}\\
= & \sum\limits_{i = 1}^n {{\lambda _i}(x)d{x^i}\left( {{\omega ^1}{e_1} + {\omega ^2}{e_2} +  \ldots  + {\omega ^n}{e_n}} \right)}\\
 = & \sum\limits_{i = 1}^n {{\lambda _i}(x)\left( {{\omega ^1}d{x^i}\left( {{e_1}} \right) + {\omega ^i}d{x^i}\left( {{e_i}} \right) +  \ldots  + {\omega ^n}d{x^i}\left( {{e_n}} \right)} \right)} \\
\begin{array}{*{20}{c}}
{d{x^i}{{\left( {{e_j}} \right)}_{ij}}}& \leftarrow 
\end{array} = & \sum\limits_{i = 1}^n {{\lambda _i}(x){\omega ^i}}  = \left( {{\lambda _i}(x), \ldots ,{\lambda _n}(x)} \right) \cdot \left( {{\omega ^1}, \ldots ,{\omega ^n}} \right)\\
 = & < v(x),\omega  > 
\end{align*}
Diese Diskussion können wir auf das Differential einer Funktion anwenden
\end{enumerate}

\begin{definition}{8.30}
Sei $f\in\subset'\left(\Omega\right)$, das durch $$< v(x),\omega >:=df(x)(\omega ), \omega\in\mathbb{R}^n$$ definierte Vektorfeld heisst Gradientenfeld von $f$ und wird mit $v(x)=\nabla f(x)$ oder grad$f$ bezeichnet. \\

Bezüglich der Standardbasis $e_1,\dots, e_n$ der $\mathbb{R}^n$ folgt die Darstellung 
\[\nabla f(x) = \left( {\begin{array}{*{20}{c}}
{\frac{{\partial f}}{{\partial {x^1}}}(x)}\\
 \vdots \\
{\frac{{\partial f}}{{\partial {x^n}}}(x)}
\end{array}} \right),\forall x \in \Omega \]
\centerline{(Oben nehmen wir $\lambda (x): = df(x) = \sum {\frac{{\partial f}}{{\partial {x^i}}}{r^i}{x^i}} $, Bemerkung 8.29, 2.)}
\end{definition}
\subsubsection*{Satz 8.31}
Sei $f\in\subset '\left(\Omega\right)$ und $x_0\in\Omega$. Dann gibt $\nabla f\left(x_0\right)$ die Richtung und $\left| \nabla f\left(x_0\right)\right|$ den Betrag des Steilsten Anstiegen von $f$ an der Stelle $x_0$
\subsubsection*{Beweis}
Aus der Definition des Gradientenfeld folgt $\forall e\in\mathbb{R}^n$, unit Vektor $\mid\mid e\mid\mid =1$ $$df\left(x_0\right)\left( e\right)=< \nabla f\left( x_0\right) ,e >$$
\todo{Add arrow pointing down for description of function, page 157 middle}
Mit Cauchy-Schwarz folgt 
$$<\nabla f\left( x_0\right)>\leq\left|\left|\nabla f\left( x_0\right)\right|\right|\left|\left| e\right|\right| = \left|\left| \nabla f\left( x_0\right)\right|\right|$$
mit Gleichkeit genau dann wann $e$ ein positives vielfachen von $\nabla f\left( x_0\right)$ ist, nähmlich $$e=\frac{\nabla f\left( x_0\right)}{\left| \nabla f\left( x_0\right)\right|}$$ $$\Rightarrow df\left( x_0\right) e \leq \left| \nabla f\left( x_0\right)\right|$$
mit gleicheit für $e=\frac{\nabla f\left( x_0\right)}{\left| \nabla f\left( x_0\right)\right|}$
%END BEWEIS
$\nabla f\left( x_0\right)\not= 0\Rightarrow \nabla f\left( x_0\right)$ zeigt die Richtung an, in der $f$ am schnellsten wächst.\\

\noindent \underline{Geometrische Interpretation}
Sei $f:\mathbb{R}^n\rightarrow\mathbb{R}, C'$. Für jedes $s\in\mathbb{R}$ wird $f^{-1}(s)=\left\{ x\in\mathbb{R}^n\mid f(x)=s\right\}$ Niveaufläche genannt.

\subsubsection*{Beispiel}
\begin{enumerate}
    \item \begin{align*}
 f:\mathbb{R}^3 &\rightarrow\mathbb{R}\\
 \left( x,y,z\right) &\rightarrow x^2 +y^2 +z^2
 \end{align*}
dann ist $f^{-1}(s)=$ Sphäre mit Zenter $O$ und Radius $\sqrt{s}$
\item $f\left( x,y\right) = xy$ ist ein Hyperbolischer Parabolid mit Niveaulinien \missingfigure{page 158, bottom, use minipage} $f:\mathbb{R}^n \rightarrow\mathbb{R}$. Nun sei $x_0\in\Omega$ mit $f\left( x_0\right) = s$, i.e. $x_0 \in f^{-1} (s)$. Sei $\gamma :\lbrack -1,1\rbrack\rightarrow \mathbb{R}^n$ ein diff. kurve durch $x_0$ mit $\gamma\lbrack -1,1\rbrack\subset f^{-1}(s)$, $\gamma (0)=x_0$
\missingfigure{page 159, middle} 
Dann gilt $f\left( \gamma (t)\right)=s$, $\forall t\in\lbrack -1,1\rbrack$ und es folgt aus Kettenregel
\[\begin{array}{l}%Find if there is a better way to do this
\frac{d}{{dt}}\left( {f\left( {\gamma (t)} \right)} \right) = \frac{d}{{dt}}(s) = 0\\
\hspace{5mm} \Downarrow \\
df\left( {\gamma (t)} \right) \cdot \gamma '(t) = 0 =  < \nabla f\left( {\gamma (t)} \right),\gamma '(t) > 
\end{array}\]
Insbesondere $0=df\left(\gamma (0)\right)\cdot\gamma'(0)= < \nabla f\left( x_0\right),\gamma'(0) >$ d.h. $\nabla f\left( x_0\right)$ steht senkrecht zur Niveauflache von $f$ durch $x_0$
\subsubsection*{Beispiel}
Sei $f(x,y)=\frac{x^2-y^2}{2}$, $x,y\in\mathbb{R}^2$ $$\nabla f(x,y)=(x,-y)$$ Sei $\left( x_0,y_0\right) = (1,-1)$ $$\nabla f(1,-1)=(1,1)\hspace{10mm} \left( \nabla f(1,-1)\right) = \sqrt{2}$$ $$\frac{\nabla f}{\left| \nabla f\right|}(1,-1)=\frac{1}{\sqrt{2}}(1,1)$$
\item Im Punkt $P$ biegt der Bergweg ab; nach Südosten geht er mit 25\% steigung berg an, nach Süden mit 20\% Gefälle berg ab. Der wanderer im Nebel möchte über die Wiese möglichst rascht zum Gipfel. In welche Richtung muss er gehen und wie steil ist es dorthin?\\

Wir legen die Koordinatensystem so, dass die $x-$Achse nach Osten und die $y-$Achse nach Norden zeigt, und setzen voraus, dass die Höhenfunktion $h$ differenzierbar ist. Wir wollen ihren Gradienten in $P$ $\nabla h(P)$ bestimmen. Noch Voraussetzung hat $h$ die beiden Richtungsableitungen $$dh\left( P\right)\left( v_1\right)=0.25\hspace{10mm}dh\left( P\right)\left(  v_2\right)=-0.2$$ wobei $$v_1=\left( \frac{1}{\sqrt{2}},-\frac{1}{\sqrt{2}}\right)\hspace{10mm}v_2=(0,-1)$$\missingfigure{page 161, top}
\begin{align*}
dh \left( P\right) \left( v_1\right) = &\left( \frac{\partial h}{\partial x}\left(P\right), \frac{\partial h}{\partial y}\left(P\right)\right)\cdot v_1\\
= & \frac{\partial h}{\partial x}\left(P\right)\frac{1}{\sqrt{2}} +\frac{\partial h}{\partial y}\left( P\right) \left(-\frac{1}{\sqrt{2}}\right) = \frac{1}{4}\\
dh \left( P\right) \left( v_2\right) = &\left( \frac{\partial h}{\partial x}\left(P\right)\right) (0) - \frac{\partial h}{\partial y}\left(P\right)(-1)=-\frac{1}{5}
\end{align*}
Durch Lösen des linearen Gleichungssystem folgen wir: 
$$\frac{\partial h}{\partial x}\left( P\right) = \frac{\sqrt{2}}{4}+\frac{1}{5}, \frac{\partial h}{\partial y}\left( P\right) = \frac{1}{5}$$
Die Richtung des Gradients ist somit $$ \nabla h\left( P \right) = \arctan{\frac{\frac{1}{5}}{\frac{\sqrt{2}}{4}+ \frac{1}{5}}}=19.86 degrees$$\todo{add arg at the beginning of the equation using special command, as well as tilde on top of equal sign}
und die Steigung in diese Richtung ist 
$$\left| \nabla h\left( P\right) \right| = \sqrt{\left( \frac{\sqrt{2}}{4}+\frac{1}{5}\right)^2 + \left( \frac{1}{5}\right)^2} = 0.59 = 59\%$$ \todo{add tilde on top of second to last equal sign}
\end{enumerate}

\section{Wegintegrale}
Wir haben gesehen in Bemerkung 8.29 dass Mittels das übliche SKalarprodukt $<,>$ kann man von $1-$Formen zu Vektorfelder und umgekehrt übergehen. \\

In diesem Kapitel werden wir das ``Wegintegral'' von $1-$Formen oder \todo{can't read, page 162 middle} von Vektorfelder längs eine Kurve studieren. Dazu untersuchen wir zunächst Kurven in $\mathbb{R}^n$

\subsection*{Parameterdarstellung einer Kurve}
Sei $\gamma\subset\mathbb{R}^n$ eine Kurve. Eine Parameterdarstellung (PD) von $\gamma$ ist eine Funktion 
\begin{align*}
\gamma : I=\lbrack a,b\rbrack &\rightarrow\mathbb{R}^n\\
t &\rightarrow \gamma (t)
\end{align*}
wobei $\gamma\left( t\right)$ ein Punkt $\gamma$ ist und jeder Punkt auf $\gamma$ kann als $\gamma\left( t\right)$ dargestellt werden
$$\gamma\left( t\right) = \left( \gamma_1 (t),\dots,\gamma_n (t)\right)$$
Die positive Orientierung von $\gamma$ ist die Richtung mit der die Kurve durchgelaufen wird
\subsubsection*{Besipiel 8.32}
\begin{enumerate}
\item $$\gamma (t)=\left( a_1 +b_1 t,a_2 +b_2 t,a_3 +b_3 t\right)\text{, }t\in\mathbb{R}$$
ist die Parameterdarstellung einer Gerade durch den Punkt $a=\left( a_1,a_2,a_3\right)$ und parallel zum Vektor $\left( b_1,b_2,b_3\right)$
\missingfigure{page 163, middle}
\item $\gamma (t)=\left( a\cos t,b\sin t\right)$ ist eine Parameter Darstellung eine Ellipse
\missingfigure{Page 163, bottom}
\item $\gamma_1 (t)=\left( a\cos t,b\sin t, ct\right)$, $t\in\lbrack 0,2\pi\rbrack$ ist eine Parameterdarstellung einer elliptische Helix 
\missingfigure{Page 164 top}
$\gamma_2 (t)=\left( a\cos t,-b\sin t, c(2\pi -t)\right)$, $t\in\lbrack 0,2\pi\rbrack$ ist Parameterdarstellung der gleichen Kurve wobei die Orientierung umgekehrt ist 
\missingfigure{page 164, middle}
Der Tangentenvektor zur Kurve an der Stelle $\gamma (t)$ ist $\gamma' (t)$ $$\gamma'(t)=\left( \gamma_1'(t),\gamma_2'(t),\dots, \gamma_n'(t)\right)$$
\todo{Do I have to include the example?? page 164 bottom}
\end{enumerate}

\begin{definition}{8.33}
Das Wegintegral von $\vartheta$ langs $\gamma$ 
\[\int\limits_\gamma  {vd\vec s}  = \int\limits_\gamma  {v(\gamma )d\gamma : = \int\limits_a^b { < v\left( {\gamma \left( t \right)} \right),\gamma '(t) > dt} } \] ${vd\vec s}=\gamma'(t)dt$ heisst gerichtetes Längeelement
\end{definition}
\subsubsection*{Beispiel 8.34}
Ein einführendes Beispiel: Sei ein Massenpunkt, der sich unter den Einfluss eines Kraftfeldes $F:\mathbb{R}^2\rightarrow\mathbb{R}$ bewegt.\\

Wenn der Massenpunkt durch eine Konstante Kraft $\overrightarrow F $ längs einer Geraden um den Vektor $\vec s$ verschoben. \\

Die dabei verrichtete Arbeit ist definitionsgemäss das Skalar Produkt aus dem Kraftvektor $\overrightarrow F $ und dem Verschiebungsvektor $\vec s$.
\missingfigure{Page 166, top}

\noindent\underline{Allgemeinen Fall}\\
Verschiebungs längs einer Kurve $\gamma$ in einem Kraftfeld $F=\left( P(x,y),Q(x,y)\right)$ 
\missingfigure{Page 166, bottom}
$\Delta W= F\cdot \Delta (\gamma)=$ Kraftkomponente entlang des Weges mal züruckgelagter weg. \\

Da sich Betrag und Richtung der Kraft sowie der jeweilige Winkel zum Weg vom Punkt zu Punkt ändern, gilt das zur Berechnung notwendige Skalarprodukt näherungsweise jeweils nur für ein Wegelement $\overrightarrow\Delta r$. Die Berechnung der Arbeit erfolgt daher in folgender Weise. 
\begin{enumerate}[\indent a)]
\item Zerlegung des Weges in Teilabschnitte $$\Delta \gamma_1=\gamma (t_{i+1})-\gamma(t_i)=\frac{\Delta \gamma}{\Delta t}\cdot \Delta t$$
\item Ermittlung der Arbeit \todo{can't read, page 167 middle} Kraft: \[F\left( {\gamma \left( {{t_i}} \right)} \right) = F\left( {x\left( {{t_i}} \right),y\left( {{t_i}} \right)} \right)\]
\item Berechnen der Arbeit je Teilabschnitt-Skalarprodukt \[\Delta {W_i} = F\left( {x\left( {{t_i}} \right),y\left( {{t_i}} \right)} \right) \cdot \Delta {\gamma _i}\]
\item Aufsummeren der Teil-Arbeit \[W \approx \sum {\Delta {W_i} = \sum {F\left( {x\left( {{t_i}} \right),y\left( {{t_i}} \right)} \right)} }  \cdot \underbrace {\Delta \gamma }_{\frac{{\Delta \gamma }}{{\Delta t}} \cdot \Delta t}\]
\item Durch Verkleinerung des Wegelementes enthält man den exakten Wert der geleisteten Arbeit
\begin{align*}
W = &\int\limits_a^b {F\left( {\gamma \left( t \right)} \right)}  \cdot \gamma '\left( t \right)dt\\
 = &\int\limits_a^b { < F\left( {\gamma \left( t \right)} \right)} ,\gamma '\left( t \right) > dt
\end{align*}
\end{enumerate} 
\subsubsection*{Bemerkung 8.35}
Wir können das Wegintegral auch mit Differentialformen formulieren. Sei 
\begin{align*}
v:\Omega &\rightarrow\mathbb{R}^n\\
x &\rightarrow\left( v^i(x)\right)_{i=1}^n
\end{align*}
ein stetiges Vektorfeld ($v^i(x):\mathbb{R}^n\rightarrow\mathbb{R}$ stetig) dann ist durch $\lambda (x)(\omega):=< v(x),\omega >$ definierte $\lambda (x)\in L\left( \mathbb{R}^n,\mathbb{R}\right)$ eine $1-$Form
\begin{align*}
\int\limits_\gamma  {vd\vec s}  = &\int\limits_a^b { < v\left( {\gamma (t)} \right),\gamma '(t) > dt}\\
 = &\int\limits_a^b {\lambda \left( {\gamma \left( t \right)} \right)\left( {\gamma '\left( t \right)} \right)dt} 
\end{align*}

\noindent\underline{Umgekehrt}\\

\noindent Sei $\lambda :\Omega\rightarrow L\left( \mathbb{R}^n,\mathbb{R}\right)$ eine $1-$Form die in Folgende Sinne stetig ist: \\

Sei 
\begin{align*}
\gamma :\lbrack a,b\rbrack &\rightarrow\Omega\\
t &\rightarrow\left( \gamma^1 (t), \gamma^2 (t),\dots ,\gamma^n (t) \right)
\end{align*}
ein $C'-$weg. Dann ist 
\begin{align*}
\lbrack a,b\rbrack \rightarrow &\mathbb{R}\\
t\rightarrow &\lambda\left( \gamma (t)\right)\left( \gamma' (t)\right)\\
& = \sum{\lambda_i\left(\gamma(t)\right)\cdot\frac{d\gamma^i}{dt}(t)}
\end{align*}
eine Stetige Funktion somit ist das Integral $\int\limits_a^b {\lambda \left( {\gamma \left( t \right)} \right)\left( {\gamma '\left( t \right)} \right)dt} $ wohl definiert. 

\begin{definition}{8.36}
Das Wegintegral von $\gamma\in L\left( \mathbb{R}^n,\mathbb{R}\right)$ längs $\gamma$ ist \[\int\limits_\gamma  \lambda  : = \int\limits_a^b {\lambda \left( {\gamma (t)} \right)\left( {\gamma '(t)} \right)dt} \]
\end{definition}

\subsubsection*{Beispiel 8.37}
\begin{enumerate}
\item Sei $\gamma\in C'\left(\lbrack 0,2\pi\rbrack =\mathbb{R}^2\right)$ mit 
\[\begin{array}{*{20}{c}}
{\gamma (t) = \left( {\begin{array}{*{20}{c}}
{\cos t}\\
{\sin t}
\end{array}} \right)}&{}\\
{}&{0 \le t \le 2\pi }\\
{\gamma '(t) = \left( {\begin{array}{*{20}{c}}
{ - \sin t}\\
{\cos t}
\end{array}} \right)}&{}
\end{array}\]
eine Parametrisierung des Einheitskreises $\lambda = \lambda (x,y)$ die $1-$Form mit $$\lambda (x,y)=-ydx+xdy\hspace{10mm}(x,y)\in\mathbb{R}^2$$
Dann gilt 
\begin{align*}
\int\limits_\gamma  \lambda   = &\int\limits_0^{2\pi } {\underbrace {\left( { - \sin t,\cos t} \right)}_{\lambda (\gamma (t))} \cdot \underbrace {\left( {\begin{array}{*{20}{c}}
{ - \sin t}\\
{\cos t}
\end{array}} \right)}_{\gamma '(t)}dt} \\
= &\int\limits_0^{2\pi } {\left( {{{\sin }^2}t + {{\cos }^2}t} \right)dt}  = 2\pi 
\end{align*}
\item Sei $\gamma (x,y)=3x^2ydx + \left( x^3+1\right)dy$. Wir betrachten die Kurvenintegral längs verschiederer Wege
\missingfigure{page 170 top, add formulas as well using a minipage}
\begin{align*}
&\int\limits_{{\gamma _1}} \lambda   = \int {3{x^2}ydx + \left( {{x^3} + 1} \right)dy}  = \int\limits_0^1 {\left( {3{t^3} + \left( {{t^3} + 1} \right)} \right)dt = 2}\\
&\int\limits_\gamma  \lambda   = \int\limits_0^1 {\left( {3{t^4} + \left( {{t^3} + 1} \right) \cdot 2t} \right)dt = 2}  
\end{align*}
\subsubsection*{Bemerkung}\todo{is  this inside the enumerated list or out?? page 170 bottom}
Sei $f(x,y)=x^3y+y$. Dann ist  $$df(x+y)=3x^2ydx+\left( x^3+1\right) dy$$ und $$f(1,1)-f(0,0)=(1+1)-(0,0)=2$$
Wir können der Begriff des Wegintegrals auf Wege zu erweitern die Stückweise $C'$ sind. Ein Stückweise $C'-$Weg ist eine Stetige Abbildung $\gamma :\lbrack a,b\rbrack\rightarrow\mathbb{R}^n$ mit einer Unterteilung des Intervals 
$$a=a_0 < a_1 < a_2 < \dots <a_n = b$$ 
so dass \[{\left. \gamma  \right|_{\left[ {{a_i},{a_{i + 1}}} \right]}} = \left[ {{a_i},{a_{i + 1}}} \right] \to \mathbb{R}^n \] $C'$ ist.\\

\noindent d.h. $t\rightarrow\gamma'(t)$ ist auf $\left( a_i, a_{i+1}\right)$ stetig und erweitert sich stetig auf $\lbrack a_i, a_{i+1}\rbrack$
\subsubsection*{Beispiel}\todo{is  this inside the enumerated list or out?? page 171 middle}
\missingfigure{Page 171 bottom}
Dann definiert man
\[\int\limits_\gamma  \lambda  : = \sum\limits_{t = 0}^{n - 1} {\int\limits_{{{\left. \gamma  \right|}_{\left[ {{a_i},{a_{i + 1}}} \right]}}} \lambda  } \]
Jetzt werden wir einzige Grundlegenden Eigenschaften des Wegintegrals herleiten. 
\end{enumerate}

\subsubsection*{Satz 8.38}
Eigenschaften des Wegintegrals
\begin{enumerate}[\indent E1)]
\item Das Wegintegral $\int\limits_\gamma  \lambda  $ ist unabhängig von einer orientierungsverhaltenden umparametrisierung. \\
D.h. Sei $\gamma :\lbrack a,b\rbrack\rightarrow\Omega$, $C'$ und $\varphi :\lbrack a',b'\rbrack\rightarrow\lbrack a,b\rbrack$, $C'$ mit $\varphi\left( a'\right)=a$, $\varphi\left( b'\right)=b$, $\varphi'\left( t\right) >0$ $\forall t\in\lbrack a', b'\rbrack$. Dann ist 
\begin{align*}
\int\limits_{\gamma  \circ \varphi } \lambda   = &\int\limits_{a'}^{b'} {\lambda \left( {\gamma \left( {\varphi (t)} \right)} \right)\left( {\gamma  \circ \varphi } \right)'(t)dt} \\
 = &\int\limits_{a'}^{b'} {\lambda \left( {\gamma \left( {\varphi (t)} \right)} \right)\gamma '\left( {\varphi (t)} \right)\varphi '(t)dt} \\
 = &\int\limits_a^b {\lambda \left( {\gamma \left( s \right)} \right)\gamma '\left( s \right)ds = \int\limits_\gamma  \lambda  } 
\end{align*}
Geometrisch heisst dies, dass $\int\limits_\gamma  \lambda$ nur vom Bild $\gamma\left( \lbrack a,b\rbrack\right)$ mit vorgegebenen Durchlaufsinn abhängt
\item Seien $\gamma_1:\lbrack a_1,b_1\rbrack\rightarrow\Omega$ und $\gamma_2:\lbrack a_2,b_2\rbrack\rightarrow\Omega$ zwei Wege mit $\gamma_1\left( b_1\right)=\gamma_2\left(a_2\right)$ 
\missingfigure{Page 173, middle to top}
Wir definieren $\gamma_1+\gamma_2$ der Weg der durch aneinanderhängen von $\gamma_1$ mit $\gamma_2$ entsteht, d.h.
\[{\gamma _1} + {\gamma _2} = \left\{ {\begin{array}{*{20}{c}}
{{\gamma _1}(t)}&{t \in \left[ {{a_1},{b_1}} \right]}\\
{{\gamma _2}\left( {t - {b_1} + {a_2}} \right)}&{t \in \left[ {{b_1},{b_1} + {b_2} - {a_2}} \right]}
\end{array}} \right.\]
Dann gilt \[\int\limits_{{\gamma _1} + {\gamma _2}} \lambda   = \int\limits_{{\gamma _1}} \lambda   + \int\limits_{{\gamma _2}} \lambda  \]
\item Sei $\gamma:\lbrack a,b\rbrack\rightarrow\Omega$ ein Weg. Dann sei $-\gamma:\lbrack a,b\rbrack\rightarrow\Omega$ der Gleiche Weg aber im entgegengesetzen Durchlaufsinn, d.h. $\left( -\gamma\right) (t)=\gamma\left( -t+a+b\right)$ 
\missingfigure{page 174, middle to top}
Dann gilt \[\int\limits_{ - \gamma } \lambda   =  - \int\limits_\gamma  \lambda  \]
\item Sei $f:\Omega\rightarrow\mathbb{R}$ eine $C'-$Funktion, sowie $\gamma:\lbrack a,b\rbrack\rightarrow\Omega$ Stückweise $C'$. Dann gilt \[\int\limits_\gamma  {df}  = f\left( {\gamma \left( b \right)} \right) - f\left( {\gamma \left( a \right)} \right)\]
$\gamma$ ist $C'$, dann ist 
\begin{align*}
\int\limits_\gamma  {df}  = & \int\limits_a^b {df\left( {\gamma \left( t \right)} \right)\gamma '\left( t \right)dt} \\
\mathop  = & \int\limits_a^b {\frac{d}{{dt}}\left( {f \circ \gamma } \right)\left( t \right)dt}\\ 
 = &\left( {f \circ \gamma } \right)\left( b \right) - \left( {f \circ \gamma } \right)\left( a \right)\\
 = & f\left( {\gamma \left( b \right)} \right) - f\left( {\gamma \left( a \right)} \right)
\end{align*}
Mittels den Wegintegrals können wir die $C'-$Funktionen charakterisieren deren Differentialverschwinden.
\end{enumerate}

\subsubsection*{Satz 8.39}
Sei $\Omega$ ``Offen'' und \todo{Can't read, page 175 middle} $\left( C'-\right)$Wegzusammenhängend. Sei $f\in C'\left(\Omega\right)$ falls $df(x)=0$, $\forall x\in\Omega$ so ist $f$ konstant.
\subsubsection*{Beweis}
$\Omega$ wegzusammenhängend heisst dass zu je zwei Punkten $x,y\in\Omega$ gibt es in $C'-$Weg $\gamma :\lbrack 0,1\rbrack\rightarrow\gamma$ mit $\gamma(0)=x$ und $\gamma(1)=y$, $\gamma\left(\lbrack 0,1\rbrack\right)\subset\Omega$. Dann folgt
\begin{align*}
f(y)-f(x)=& f\left(\gamma(1)\right)-f\left(\gamma(0)\right)\\
= &\int\limits_{\gamma} df = 0
\end{align*}
$\Rightarrow f(y)=f(x)$, $\forall x,y\in\Omega\Rightarrow f$ ist konstant.\\

\noindent\underline{Frage:} Wann ist eine $1-$Form $\lambda$, von der form $\lambda = df$, d.h. differential einer Funktion? d.h. gegeben eine $1-$Form $\lambda$, gibt es eine Funktion $f:\Omega\rightarrow\mathbb{R}$ s.d. $df=\lambda$\\

Wenn ein $f:\Omega\rightarrow\mathbb{R}$ gibt so dass $df=\lambda$, heisst $f$ ein Potential. (Potential ist wie ein Stammfunktion für ein $1-$Form). Mittels Wegintegral,stellen wir jetzt ein Kriterium

\subsubsection*{Satz 8.30}
Sei $\lambda\in\Omega\rightarrow L\left( \mathbb{R}^n,\mathbb{R}\right)$ eine Stetige $1-$Form. Folgende Aussage sind äquivalent
\begin{enumerate}
\item Es gibt $f\in C'\left( \Omega\right)$ mit $df=\Omega$
\item Für je zwei Stückweise $C'-$Wege $\gamma_i=\lbrack a_i,b_i\rbrack\rightarrow\Omega$ mit selben Anfangs und Endpunkten (d.h. $\gamma_1\left( a_1\right)=\gamma_2\left(a_2\right)$, $\gamma_1\left( b_1\right)=\gamma_2\left(b_2\right)$) gilt \[\int\limits_{{\gamma _1}} \lambda   = \int\limits_{{\gamma _2}} \lambda  \]
\item Für jede geschlossene $C'$Weg $\gamma$ gilt \[\int\limits_\gamma  \lambda   = 0\]
\end{enumerate} 

\subsubsection*{Beweis}
\begin{enumerate}[align=left]
\item[$(1)\Rightarrow (2)$:] Folgt aus E4) 
\item[$(2)\Leftrightarrow (3)$:] Klar
\item[$(2)\Rightarrow (1)$:] Sei $p_0\in\Omega$; für jedes $x\in\Omega$. Sei $\gamma:\lbrack 0,1\rbrack\rightarrow\Omega$ Stückweise $C'$ mit $\gamma(0)=p_0$, $\gamma(1)=x$. Definiere $f(x): = \int\limits_\gamma  \lambda $.\\

Dann ist $f$ nach Annahme (2) Wohldefiniert (d.h. unabhängig von dem Weg von $p_0$ nach $x$) (Wir können $f$ auch mit $\int\limits_{{p_0}}^x \lambda  $ bezeichnen)
\end{enumerate}

\subsubsection*{Behauptung}
$f\in C'\left( \Omega\right)$ und $df=\lambda$. Um zu zeigen dass $df=\lambda$ müssen wir zeigen dass für $x,x_0\in\Omega$
$$f\left(x\right)-f\left(x_0\right)=\lambda\left(x_0\right)\left( x-x_0\right)+R\left( x,x_0\right)$$ mit $\frac{R\left( x,x_0\right)}{\left| x-x_0\right|}\rightarrow 0$.\\

\noindent Sei $x_0\in\mathbb{R}$. Sei $\gamma_1:\lbrack -1,0\rbrack\rightarrow\Omega$ ein Weg von $p_0$ nach $x_0$. Dann gilt \[\int\limits_{{\gamma _1}} \lambda   = f\left( {{x_0}} \right)\]
Sei
\begin{align*}
\gamma_x:\lbrack 0,1\rbrack &\rightarrow\Omega\\
t &\rightarrow (1-t)x_0+tx
\end{align*}
\missingfigure{Page 178 bottom}
Um $\gamma^x\left( \lbrack 0,1\rbrack\right)\subset\Omega$ zu garantieren, nehmen wir $r>0$ so dass $B_r\left( x_0\right)\subset\Omega$ und nehmen an, dass $x\in B_r\left( x_0\right)$. Dann ist
\[f(x) = \int\limits_{{\gamma _1} + {\gamma _x}} \lambda   = \int\limits_{{\gamma _1}} \lambda   + \int\limits_{{\gamma _x}} \lambda   = f\left( {{x_0}} \right) + \int\limits_{{\gamma ^x}} \lambda  \]
Nun ist
\begin{align*}
\int\limits_{{\gamma ^x}} \lambda   = &\int\limits_0^1 {\lambda \left( {{\gamma _x}\left( t \right)} \right){\gamma _x}'\left( t \right)} dt\\
 = &\int\limits_0^1 {\lambda \left( {{\gamma _x}\left( t \right)} \right)\left( {x - {x_0}} \right)} dt\\
 = &\lambda \left( {{x_0}} \right)\left( {x - {x_0}} \right) + \int\limits_0^1 {\left( {\lambda \left( {{\gamma ^x}\left( t \right)} \right) - \lambda \left( {{x_0}} \right)} \right)\left( {x - {x_0}} \right)dt} 
\end{align*}

Sei $\lambda  = \sum {{\lambda ^i}d{x^i}} $ dann ist obigen Integral gleich 
\begin{align*}
&\sum {\int\limits_0^1 {\left[ {{\lambda _i}\left( {{\gamma ^x}\left( t \right)} \right) - {\lambda _i}\left( {{x_0}} \right)} \right]\left( {{x^i} - x_0^i} \right)dt} }\\
 \le &\sum {{{\left( {\int\limits_0^1 {{{\left[ {{\lambda _i}\left( {{\gamma ^x}\left( t \right)} \right) - {\lambda _i}\left( {{x_0}} \right)} \right]}^2}} } \right)}^{\frac{1}{2}}}} \left| {x - {x_0}} \right| 
\end{align*}
Also $f\left( x \right) - f\left( {{x_0}} \right) = \lambda \left( {{x_0}} \right)\left( {x - {x_0}} \right) + R\left( {x,{x_0}} \right)$, wobei 
\[\frac{{R\left( {x - {x_0}} \right)}}{{\left| {x - {x_0}} \right|}} \le {\left( {\sum\limits_{i = 1}^n {{{\left( {\int\limits_0^1 {\left( {{\lambda _i}\left( {{\gamma ^x}\left( t \right)} \right) - {\lambda _i}\left( {{x_0}} \right)} \right)dt} } \right)}^2}} } \right)^{\frac{1}{2}}}\]
Aus stetigkeit der \todo{Can't read, page 179 bottom} folgt das
\[\mathop {\lim }\limits_{x \to {x_0}} \frac{{R\left( {x,{x_0}} \right)}}{{\left| {x - {x_0}} \right|}} \to 0\]

\subsubsection*{Beispiel 8.31}
\begin{enumerate}
\item Sei $\lambda = 2xy^2dx+2x^2ydy$. \\

\noindent\underline{Ansatz:} $$f\left(x,y\right)=\int\limits_{\gamma_{\left( x,y\right)}}\lambda$$
wobei $\gamma_{\left( x,y\right)}\left(t\right)=\left(t_x,t_y\right)$, $t\in\left( 0,1\right)$. Dann ist
 
\begin{align*}
\int\limits_\gamma  \lambda   = &\int\limits_0^1 {\lambda \left( {tx,ty} \right)\left( {x,y} \right)dt} \\
= &\int\limits_0^1 {\left[ {2\left( {tx} \right){{\left( {ty} \right)}^2} \cdot x + 2{{\left( {tx} \right)}^2}\left( {ty} \right) \cdot y} \right]dt} \\
= & 4{x^2}{y^2}\int\limits_0^1 {{t^3}} dt = {x^2}{y^2}
\end{align*}
und $df\left( x,y\right)=2xy^2dx+2x^2ydy$.\\

\noindent\underline{Oder:} Ansatz: 
$$df:\lambda \Rightarrow\frac{\partial f}{\partial x}=  2xy^2, \frac{\partial f}{\partial y}=  2x^2y$$
$$\Rightarrow \frac{\partial f}{\partial x}=2xy^2 \Rightarrow f\left( x,y \right) = \int 2xy^2 dx =  x^2y^2+C\left( y\right)$$
$$\Rightarrow\frac{\partial f}{\partial y}=2x^2y+\frac{d}{dy}C\left(y\right)=2x^2y \Rightarrow \frac{d}{dy}C\left( y\right) =  0 \Rightarrow C\left( y\right) = \text{ Konstant}$$
$$\Rightarrow f\left( x,y\right)=  x^2y^2+C$$
\end{enumerate}\todo{Where is number 2?? page 180}
Analog wie für $1-$Formen kann man Satz 8.30 für Vektorfelder Formulieren

\begin{definition}{8.32}
Ein Vektorfeld $v:\Omega\rightarrow\mathbb{R}^n$ heisst konservative falls $\forall\gamma:\lbrack 0,1\rbrack\rightarrow\Omega$ geschlossen $$\int\limits_\gamma v ds=0$$
\end{definition}
Aus Satz 8.30 Folgt
\subsubsection*{Satz 8.33}
Für eine Stetige Vektorfeld $v:\Omega\rightarrow\mathbb{R}^n$ sind folgende Aussagen equivalent
\begin{enumerate}
\item $v$ ist Konservative
\item Es gibt $f\in C'\left( \Omega\right)$ mit $v=\nabla f$. In diesem Fall heisst $v$ Potentialfeld mit dem Potential $f$.
\end{enumerate}

Im Nächsten Kapitel, mittels höhere Partielle Ableitungen, erhalten wir eine einfach zu \todo{can't understand, page 182 top} notwendige Bedingung für ein Konservatives Vektorfeld. Wir werden sehen dass
\[v = {\left( {{v^i}} \right)_{1 \le i \le n}} \in C'\left( {\Omega ,\mathbb{R}^n} \right)\text{ konservative }\]
$$\Rightarrow \frac{\partial v^i}{\partial x^j}=\frac{\partial v^j}{\partial x^i}\hspace{5mm}1\leq i,j\leq n$$

\section{Höhere Ableitungen}
\begin{definition}{8.44}
$f:\Omega\rightarrow\mathbb{R}$, $\Omega\subset\mathbb{R}^n$ $f\in C'\left( \Omega\right)$ heisst von Klasse $C^2$ falls $\frac{\partial f}{\partial x^i}\in C'{\left( \Omega\right)}_{1\leq i\leq n}$
\end{definition}\todo{Where does the definition end? page 183 top}
Für beliebiges $m$, die Funktion $f\in C'\left( \omega\right)$ heisst von der Klasse $C^m$, $f\in C^m\left( \omega\right)$ falls $\frac{\partial f}{\partial x^i}\in C^{m-1}\left( \Omega\right)$, $1\leq i\leq n$\\

Für eine $f\in C^2\left( \Omega\right)$, die Funktionen \[\frac{{{\partial ^2}f}}{{\partial {x^i}\partial {x^j}}}: = \frac{\partial }{{\partial {x^i}}}\left( {\frac{{\partial f}}{{\partial {x^j}}}} \right)\] heissen die zweiten partiellen Ableitungen von $f$.\\

Analog definiert man die $m-$ten partielle Ableitungen von $f$ oder partielle Ableitungen vom Grad $m$ für jedes $m>0$ (Für $f\in C^m \left(\Omega\right)$)

\subsubsection*{Satz 8.45}
Sei $f\in C^2\left(\Omega\right)$. Dann gilt
\begin{align*}
\frac{{{\partial ^2}f}}{{\partial {x^i}\partial {x^j}}} = &\frac{\partial }{{\partial {x^i}}}\left( {\frac{{\partial f}}{{\partial {x^j}}}} \right)\\
 = &\frac{\partial }{{\partial {x^j}}}\left( {\frac{\partial }{{\partial {x^i}}}f} \right) = \frac{{{\partial ^2}f}}{{\partial {x^j}{x^i}}}
\end{align*}
Im Allgemein
\subsubsection*{Satz 8.46}
Für jede $C^k-$Funktion sind alle Partielle Ableitungen vom Grad $\leq k$ von der Reihenfolge der Ableitungen unabhängig. Von Satz 8.35 erhalten wir folgende notwendige Bedingung für konservativität

\subsubsection*{Korollar 8.47}
Sei $v:\Omega\rightarrow\mathbb{R}^n$, $v={\left( v^i\right)}_{1\leq i \leq n}$ ein $C'-$Vektorfeld. Falls $v$ konservativ ist, folgt 
\[\frac{{\partial {v^i}}}{{\partial {x^j}}} = \frac{{\partial {v^j}}}{{\partial {x^i}}}\hspace{5mm}1 \le i,j \le n\]

\subsubsection*{Beweis}
Nach voraussetzung gibt es $f\in C'\left( \Omega\right)$ mit $v^i(x)=\frac{\partial f}{\partial x^i}$. Da nun $v^i \in C'$, $1\leq i\leq n$ folgt $f\in C^2\left( \Omega\right)$. Woraus 
\[\frac{{\partial {v^i}}}{{\partial {x^j}}} = \frac{\partial }{{\partial {x^j}}}\left( {\frac{{\partial f}}{{\partial {x^i}}}} \right) = \frac{\partial }{{\partial {x^i}}}\left( {\frac{{\partial f}}{{\partial {x^j}}}} \right) = \frac{\partial }{{\partial {x^i}}}{v^j}\]
folgt. 

\subsubsection*{Beispiel 8.48}
\begin{enumerate}
\item \[v\left( {x,y} \right) = \left( {\begin{array}{*{20}{c}}
{4x{y^2}}\\
{2y}
\end{array}} \right)\]
Es gilt $\frac{\partial v'}{\partial y}=8xy$, $\frac{\partial v^2}{dx}=2$. Also ist $v$ nicht konservativ
\item Sei $\Omega = \left\{ \left( x,y\right) \in\mathbb{R}^2:\left( x,y\right) \not= (0,0) \right\} = \mathbb{R}^2\backslash\left\{ 0,0\right\}$ und \[v(x,y) = \left( {\frac{{ - y}}{{{x^2} + {y^2}}},\frac{x}{{{x^2} + {y^2}}}} \right)\] Dann $v:\Omega \rightarrow\mathbb{R}^2$ mindestens $C'$. Ausserdem \[\frac{{\partial v'}}{{\partial y}} = \frac{{{y^2} - {x^2}}}{{{{\left( {{x^2} + {y^2}} \right)}^2}}} = \frac{{\partial {v^2}}}{{\partial x}}\] Jetzt berechnen wir $\int\limits_C v ds$, wobei C: 
\missingfigure{page 186, middle. Also add the formula describing $C(t)$ using a minipage}
\begin{align*}
\int\limits_C vds  = & \int\limits_0^{2\pi } { < v\left( {C\left( t \right)} \right),C'\left( t \right) > dt} \\
= & \int\limits_0^{2\pi } \left( -\sin t,\cos t\right)\cdot  \left( {\begin{array}{*{20}{c}}
{ - \sin t}\\
{\cos t}
\end{array}} \right) dt\\
= & \int\limits_0^{2\pi } \left( \sin^2 t+\cos^2 t\right) dt=2\pi\not= 0\\
\Rightarrow &\text{ $v$ auf $\Omega$ ist nicht konservativ!}
\end{align*}
Jetzt betrachten wir $\Omega'=\left\{ \left( x,y\right) \mid x>0 \right\}$ \todo{can't understand image on page 187, top} und führen Polarkoordinaten ein $$x=r\cos\theta, y=r\sin\theta\hspace{10mm} -\frac{\pi}{2}<\theta <\frac{\pi}{2}$$
Dann ist $\tan\theta = \frac{y}{x}$ und $$\theta=\arctan\frac{y}{x}$$
Wir betrachten $\theta:\Omega'\rightarrow\mathbb{R}$ als ein Funktion der Variabeln $x,y$ und berechnen 
\begin{align*}
\frac{{\partial \theta }}{{\partial x}} = & \frac{1}{{1 + {{\left( {\frac{y}{x}} \right)}^2}}}\left( { - \frac{y}{{{x^2}}}} \right) =  - \frac{y}{{{x^2} + {y^2}}}\\
\frac{{\partial \theta }}{{\partial x}} = & \frac{x}{{{x^2} + {y^2}}}
\end{align*}
Also gilt 
$$\nabla \theta\left( x,y\right) = v\left( x,y\right)$$
$$v\left( x,y\right) \in\Omega'$$
$$\Rightarrow\text{ $v$ ist konservative auf $\Omega$}$$
Das heisst konservativität ist eine Eigenschaft zugleich des Vektorfeldes $v$ \underline{v} der Region $\Omega$
\end{enumerate}

\begin{definition}{8.49}
Eine offene Menge $\Omega\subset\mathbb{R}^n$ heisst einfach zusammenhängend falls 
\begin{enumerate}
\item $\Omega$ ist stückweise $C'-$Wegzusammenhängend
\item Jeder Stückweise $C'-$Weg in $\Omega$ kann stetig innerhalb $\Omega$ auf einen Punkt zusammengezogen werden 
\end{enumerate}
Die Region $\Omega=\mathbb{R}\backslash\{ 0\}$ ist nicht einfach zu $\Omega' = \left\{ \left( x,y\right) \mid x>0\right\}$ ist es aber.
\end{definition}

\subsubsection*{Satz 8.50}
Sei $\Omega\in\mathbb{R}^2$ beschränkt zusammenhängend sowie einfachzusammenhängend, sei $v\in C'\left( \Omega:\mathbb{R}^2\right)$ Vektorfeld. Dann sind äquivalent 
\begin{enumerate}
\item $v$ ist konservativ
\item $\frac{\partial v_1}{\partial y}=\frac{\partial v_2}{\partial x}$
\end{enumerate}

\subsection*{Taylorentwicklung und der lokale Verhalten von $C^m-$Funktionen}\todo[inline]{Not sure how big of a title...}
Wir werden jetzt ein Verallgemeinerung der $1.$ Variablen Taylorentwicklung herleiten. \\

Sei $f:\mathbb{R}^n\rightarrow\mathbb{R}$ eine $C^m-$Funktion sowie $x_0,x_1\in\mathbb{R}^n$. (Allgemein könnte man $\mathbb{R}^n$ durch eine offene konvexe Menge ersetzen)\\

\noindent Sei
\begin{align*} 
\varphi:\mathbb{R} &\rightarrow\mathbb{R}^n\\
t & \rightarrow \left( 1-t\right) x_0+x_1
\end{align*}
Dann ist $g:=f\circ\varphi:\mathbb{R}\rightarrow\mathbb{R}$ eine \todo[inline]{Dont know where this actually fits: $\left( g(0)=f=\left( x_0\right), g(1)=f\left( x_1\right)\right)$}$C^m-$Funktion und (nach Taylor \todo{can't read, page 189 bottom} von Funktionen $1-$variable) es gibt $\xi\in (0,1)$ so dass 

\[
g\left( {} \right) = g\left( 0 \right) + g'\left( 0 \right) +  \ldots  + \frac{{{g^{\left( {m - 1} \right)}}\left( 0 \right)}}{{\left( {m - 1} \right)!}} + \frac{{{g^{\left( m \right)}}\left( \xi  \right)}}{{m!}}\tag{\textasteriskcentered\label{Chap8Equation,Taylor}}
\] 
\todo[inline]{can't read between brackets before equal sign, page 190 very top}
Jetzt berechnen wir $g^{(i)}(t)$ im Funktion von $f$ und seinem Ableitungen. Für $g'(t)$ benutzen wir die Kettenregel:
$$g'(t)=df\left( \varphi (t)\right) \cdot\varphi' (t)$$ mit \[\varphi '(t) = {x_1} - {x_0} = \left( {{x_1}' - {x_0}',{x_1}^2 - {x_0}^2, \ldots ,{x_1}^n - {x_0}^n} \right)\] Erhalten wir:

\begin{align*}
g'(t) = & \sum\limits_{i = 1}^n {\frac{{\partial f}}{{\partial {x^i}}}\left( {\varphi (t)} \right)\left( {x_1^i - x_0^i} \right)}  = \nabla f\left( {\varphi (t)} \right) \cdot \left( {{x_1} - {x_0}} \right)\\
g'(0) = & \sum\limits_{i = 1}^n {\frac{{\partial f}}{{\partial {x^i}}}\left( {x_0} \right)\left( {x_1^i - x_0^i} \right)}  = \nabla f\left( {x_0} \right) \cdot \left( {{x_1} - {x_0}} \right)
\end{align*}
Jetzt berechnen wir $g^{(2)}(t)$:\[{g^{(2)}}(t) = \frac{d}{{dt}}\left( {g'\left( t \right)} \right) = \sum\limits_{i = 1}^n {\frac{d}{{dt}}\left( {\frac{{\partial f}}{{\partial {x^i}}}\left( {\varphi \left( t \right)} \right)} \right)} \left( {x_1^i - x_0^i} \right)\]
Analog gilt:
\[\frac{d}{{dt}}\left( {\frac{{\partial f}}{{\partial {x^i}}}\left( {\varphi \left( t \right)} \right)} \right) = \sum\limits_{j = 1}^n {\frac{{{\partial ^2}f}}{{\partial {x^j}\partial {{\text{x}}^i}}}\left( {\varphi \left( t \right)} \right)\left( {x_1^j - x_0^j} \right)} \]
Eingesetzt gilt: 
\begin{align*}
{g^{(2)}}(t) = & \sum\limits_{i = 1}^n {\sum\limits_{j = 1}^n {\left( {\frac{{{\partial ^2}f}}{{\partial {x^j}\partial {x^i}}}\left( {\varphi (t)} \right)} \right)} } \left( {x_1^i - x_0^i} \right)\left( {x_1^j - x_0^j} \right)\\
{g^{(2)}}(0) = & \sum\limits_{i,j = 1}^n {\left( {\frac{{{\partial ^2}f}}{{\partial {x^j}\partial {x^i}}}\left( {{x_0}} \right)} \right)} \left( {x_1^i - x_0^i} \right)\left( {x_1^j - x_0^j} \right)
\end{align*}
Daraus schliesst man induktive das
\[{g^{(k)}}(t) = \sum\limits_{{i_1},{i_2}, \ldots ,{i_k} = 1}^n {\left( {\frac{{{\partial ^k}f}}{{\partial {x^{{i_1}}} \ldots \partial {x^{{i_k}}}}}\left( {\varphi (t)} \right)} \right)} \prod\limits_{l = 1}^k {\left( {x_1^{{i_l}} - x_0^{{i_l}}} \right)} \]
Eingesetzt in (\textasteriskcentered) (s.\pageref{Chap8Equation,Taylor}) ergibt \todo[inline]{MISSING CONTENT?? page 191 bottom}

\subsubsection*{Satz 8.51(Taylor entwicklung)}
\begin{align*}
f\left( {{x_1}} \right) = & f\left( {{x_0}} \right) + \sum\limits_{i = 1}^n {\frac{{\partial f}}{{\partial x'}}\left( {{x_0}} \right)\left( {x_1^i - x_0^i} \right) +  \ldots } \\
 + & \frac{1}{{\left( {m - 1} \right)!}}\sum\limits_{{i_1}, \ldots ,{i_{m - 1}} = 1}^n {\frac{{{\partial ^{m - 1}}f}}{{\partial {x^{{i_1}}} \ldots \partial {x^{{i_{m - 1}}}}}}\left( {{x_0}} \right)\prod\limits_{l = 1}^{\left( {m - 1} \right)} {\left( {x_1^{{i_l}} - x_0^{{i_l}}} \right)} } \\
 + & \frac{1}{{m!}}\sum\limits_{{i_1}, \ldots ,{i_m} = 1}^n {\frac{{{\partial ^m}f}}{{\partial {x^{{i_1}}} \ldots \partial {x^{{i_m}}}}}\left( {{x_\xi }} \right)\prod\limits_{l = 1}^m {\left( {x_1^{{i_l}} - x_0^{{i_l}}} \right)} } 
\end{align*}
mit eine Zahl $\xi\in\left( 0,1\right)$, $x_\xi = \left( 1-\xi\right)x_0+\xi x_1$.

\subsubsection*{Bemerkung 8.52}
Insbesondere für $m=2$ erhalten wir für $f$ die quadratische Näherung
\begin{align*}
f\left( {{x_1}} \right) = & f\left( {{x_0}} \right) + \nabla f\left( {{x_0}} \right)\left( {{x_1} - {x_0}} \right)\\
 + &\frac{1}{2}\sum\limits_{i,j = 1}^2 {\frac{{{\partial ^2}f}}{{\partial {x^i}\partial {x^j}}}\left( {{x_0}} \right)\left( {x_1^i - x_0^i} \right)} \left( {x_1^j - x_0^j} \right) + {r_2}\left( {f,{x_1},{x_0}} \right)
\end{align*}
mit Fehler 
$$\frac{r_2\left( f,y_1,x_0\right)}{\left| x_1-x_0\right|}\rightarrow 0, \left( x_1\rightarrow x_0\right)$$

\begin{definition}{8.53}
Die Matrix der Zweiten partiellen Ableitungen heisst die Hesse - Matrix von $f$, und mit $\Hess(f)$ oder $\nabla^2 f$ bezeichnet
\begin{align*}
\Hess(f)= & \nabla^2 f:=\left( \frac{\partial^2 f}{\partial x^i\cdot\partial x^j}\right)_{i,j=1\dots n}\\
 = & \left( {\begin{array}{*{20}{c}}
{\frac{{{\partial ^2}f}}{{\partial x'\partial x'}}}&{\frac{{{\partial ^2}f}}{{\partial x'\partial {x^2}}}}& \ldots &{\frac{{{\partial ^2}f}}{{\partial x'\partial {x^n}}}}\\
{\frac{{{\partial ^2}f}}{{\partial {x^2}\partial x'}}}&{\frac{{{\partial ^2}f}}{{\partial {x^2}\partial {x^2}}}}& \ldots &{\frac{{{\partial ^2}f}}{{\partial {x^2}\partial {x^n}}}}\\
 \vdots & \vdots & \ddots & \vdots \\
{\frac{{{\partial ^2}f}}{{\partial {x^n}\partial x'}}}& \ldots & \ldots &{\frac{{{\partial ^2}f}}{{\partial {x^n}\partial {x^n}}}}
\end{array}} \right)
\end{align*}
\end{definition}
Seien $\nabla f,x_1-x_0$ Zeilenvektoren und sei $\left( x-x_0\right)^t$ der zu $x_1-x_0$ transponierte Spaltenvektor . Dann wird die Taylorentwicklung von Grad 2 äquivalent zu 

\begin{align*}
f(x) = & f\left( x_0\right) + \nabla f\left( x_0\right) \left( x-x_0\right)^t\\
+ & \frac{1}{2}\left( x-x_0\right)\cdot\nabla^2 f\left( x_0\right)\left( x-x_0\right)^t\\
+ & r_3\left( f,x,x_0\right)
\end{align*}

\subsubsection*{Bemerkung}
Die Hesse - Matrix von $f$, nach Satz von Schwarz ist eine Symmetrische Matrix.

\subsubsection*{Beispiel}
$f\left( x,y\right)=e^{x+y}\cos x$ im Punkt $(0,0)$. Die Taylorentwicklung vom Grad 2:
\begin{align*}
\frac{{\partial f}}{{\partial x}} = & {e^{x + y}}\cos x - {e^{x + y}}\sin x,\text{ }\frac{{\partial f}}{{\partial x}}(0,0) = 1\\
\frac{{\partial f}}{{\partial x}} =  & {e^{x + y}}\cos x ,\text{ }\frac{{\partial f}}{{\partial x}}(0,0) = 1\\
\end{align*}
\todo[inline]{shouln't it be $\frac{{\partial f}}{{\partial y}}(0,0) = 1$ for second one??}
$$\left( \nabla f\right)(0,0)=(1,1)\hspace{10mm}f(0,0)=1$$
\begin{align*}
\frac{\partial }{{\partial x}}\left( {\frac{{\partial f}}{{\partial y}}} \right) = & {e^{x + y}}\cos x - {e^{x + y}}\sin x,\text{ }\frac{{{\partial ^2}f}}{{\partial x\partial y}}(0,0) = 1\\
\frac{{\partial^2 f}}{{\partial x^2}} = \frac{\partial }{{\partial x}}\left( {\frac{{\partial f}}{{\partial x}}} \right) = & {e^{x + y}}\cos x - {e^{x + y}}\sin x - {e^{x + y}}\sin x - {e^{x + y}}\cos x\\
= & -2e^{x+y}\sin x\\
\frac{{{\partial ^2}f}}{{\partial {x^2}}}(0,0) = & 0\\
\frac{{{\partial ^2}f}}{{\partial {y^2}}} = & {e^{x + y}}\cos x\hspace{10mm}\frac{{{\partial ^2}f}}{{\partial {y^2}}}(0,0) = 1
\end{align*}
\[{\nabla ^2}f(0,0) = \left( {\begin{array}{*{20}{c}}
0&1\\
1&1
\end{array}} \right)\]

\begin{align*}
\left( {\left( {x,y} \right) - \left( {0,0} \right)} \right){\nabla ^2}f\left( {0,0} \right){\left( {\left( {x,y} \right) - \left( {0,0} \right)} \right)^T} = & \left( {x,y} \right)\left( {\begin{array}{*{20}{c}}
0&1\\
1&1
\end{array}} \right)\left( {\begin{array}{*{20}{c}}
x\\
y
\end{array}} \right)\\
 = & \left( {x,y} \right)\left( {\begin{array}{*{20}{c}}
y\\
{x + y}
\end{array}} \right)\\
 = &\text{ }2xy+y^2
\end{align*}
\begin{align*}
f\left( {x,y} \right) = & \text{ }{e^{x + y}}\cos x = 1 + \left( {\begin{array}{*{20}{c}}
1\\
1
\end{array}} \right)\left( {x,y} \right) + \frac{1}{2}\left( {x,y} \right)\left( {\begin{array}{*{20}{c}}
0&1\\
1&1
\end{array}} \right)\left( {\begin{array}{*{20}{c}}
x\\
y
\end{array}} \right)\\
 = & \text{ } 1 + \left( {x,y} \right) + \frac{1}{2}\left( {2xy + {y^2}} \right) + {r_3}\left( {f,\left( {x,y} \right)} \right)
\end{align*}
Taylorpolynom von Grad 2: $1+\left( x+y\right)+\frac{1}{2}\left( 2xy+y^2\right)$\\

Die Hesse - Matrix bestimmt ob die Funktion $f$ in der Nähe von $x$ konvex oder konkav ist (oder nicht). Sie ``spielt'' die gleiche Rolle, wie die zweite Ableitung von Funktionen in einer Variable. \\

Als nächstes benötigen wir eine mehrdimensionale Entsprechung zu den positivität \todo{Can't understand word between brackets, page 197 middle} in den eindimensionalen Beziehungen $f''(z)>0$ bzw. $f''(z)<0$.

\begin{definition}{8.54}
Eine symmetrische Matrix $A=\left( a_{ij}\right)\in\mathbb{R}^{n\times n}$ heisst
\begin{enumerate}
\item \textbf{Positiv definit} wenn $${}^txAx = \sum\limits_{i,j = 1}^n {{a_{ij}}{x^i}{x^j} > 0}\hspace{10mm}\forall x\in\mathbb{R}^n$$ (oder wenn ihre Eigenwerte sämtlich positive sind)
\item \textbf{Negativ definit} wenn \[{}^txAx < 0\hspace{10mm}\forall x\in\mathbb{R}^n\] (wenn ihre Eigenwerte sämtlich negativ sind)
\item Sonst \textbf{indefinit} (wenn sie sowohl positive als auch negative Eigenwerte besitzt)
\end{enumerate}
Im symmetrischen $2\times 2$ Fall ist die Gleichung auf Definitheit besonders leicht
\end{definition}

\subsubsection*{Satz 8.55}
Eine Symmetrische Matrix 
\[A = \left( {\begin{array}{*{20}{c}}
{{a_{11}}}&{{a_{12}}}\\
{{a_{12}}}&{{a_{11}}}
\end{array}} \right)\]
ist genau dann
\begin{enumerate}
\item Positive definit, wenn $\det A>0$ und $a_{11}>0$
\item Negativ definit, wenn $\det A>0$ und $a_{11}<0$
\item Indefinit, wenn $\det A<0$
\end{enumerate}

\subsection*{Extrema von Funktionen mehrere Variablen}
Jetzt werden wir nach Punkten $x\in\mathbb{R}^n$ sehen, in denen eine funktion $F:\mathbb{R}^n\rightarrow\mathbb{R}$ ein lokales Extremum annimmt. Wir erinnern uns an das Vorgehen im $f:\mathbb{R}\rightarrow\mathbb{R}$:
\begin{enumerate}
\item Finde alle Punkte $x\in\mathbb{R}$, für die $f'(x)=0$ gilt (Notwendige Bedingung)
\item Falls in einem solchen Punkt zusätzlich $f''(x)>0$ (bzw. $f''(z)<0$) gilt so handelt es sich um ein lokales Minimum (bzw. Maximum) (hinreichende Bedingung)
\end{enumerate}
Jetzt verallgemeinern wir diese Strategie Zunächst

\begin{definition}{8.55}
Ein Punkt $x_0\in\mathbb{R}^n$ mit $df\left( x_0\right)=0$ heisst \underline{kritischer Punkt} von $f$ (oder \underline{stationärer} Punkt von $f$)
\end{definition}

\subsubsection*{Satz 8.56}
Sei 
\begin{align*}
f: &\text{ } \Omega \subset\mathbb{R}^n\rightarrow\mathbb{R}\\
f \in & \text{ }C^2\left(\Omega\right); x_0\in\Omega
\end{align*}
\begin{enumerate}
\item Falls $x_0\in\Omega$ lokale Extremum (min oder max) von $f$ ist, so gilt $df\left(x_0\right)=0$
\item Falls $df\left( x_0\right)=0$, und falls $\Hess \left( f\right)\left( x_0\right)$ positive definiert ist, so ist $x_0$ eine lokale Minimalstelle
\item Falls $df\left( x_0\right)$, und falls $\Hess_f\left( x_0\right)<0$ negativ definiert ist, so ist $x_0$ eine lokale Maximalstelle
\item Falls $df\left( x_0\right)=0$, und $\Hess_f\left( x_0\right)$ indefinite ist, so ist $x_0$ ein Sattelpunkt (d.h. jede Umgebung $U$ von $x_0$ enthält Punkte $p,q\in U$ mit $f\left( P\right) > f\left( x_0\right)>f\left( q\right)$)
\end{enumerate}

\subsubsection*{Beispiel}
\begin{enumerate}
\item \begin{align*}
f\left( x,y,z\right)= & \left(x-1\right)^2+\left(y+2\right)^2+\left(z+1\right)^2\\
\nabla f= & \left( 2\left(x-1\right),2\left(y+2\right),2\left(z+1\right)\right)\\
\nabla f\left( x_0\right)=&\left( 0,0,0\right)\Rightarrow x_0=\left( 1,-2,-1\right)
\end{align*}
\[{H_f}\left( {{x_0}} \right) = \left( {\begin{array}{*{20}{c}}
2&0&0\\
0&2&0\\
0&0&2
\end{array}} \right)\]
$H_f\left( x_0\right)$ ist positiv definiert $\Rightarrow x_0\left( 1,-2,-1\right)$ ist ein lokales Minimum. 
\item \begin{align*}
f\left( x,y\right)= & \cos\left(x+2y\right)+\cos\left(2x+3y\right)\\
\nabla f= & \left( -\sin\left(x+2y\right)-2\sin\left( 2x +3y\right)\right.,\\
&\left.-2\sin\left(x+2y\right)-3\sin\left(2x+3y\right)\right)=\left( 0,0\right)\\
\Rightarrow &-\sin\left( x+2y\right)-2\sin\left( 2x+3y\right)=0\\
&-2\sin\left( x+2y\right)-3\sin\left( 2x+3y\right)=0
\end{align*}
$$\Rightarrow \sin\left( 2x+3y\right) = 0,\text{ }\sin\left( x+2y\right)=0$$
\[ \Rightarrow \left. {\begin{array}{*{20}{r}}
{2x + 3y = k\pi }\\
{x + 2y = l\pi }
\end{array}} \right\} \Rightarrow y = k\pi {\text{ und }}x = l\pi \]
Kritische punkte: $\left( \pi l,\pi k\right)\hspace{5mm}k,l\in\mathbb{Z}$
\begin{align*}
\frac{{\partial f}}{{\partial y\partial x}} =&\frac{{\partial f}}{{\partial x\partial y}} =  - 2\cos \left( {x + 2y} \right) - 6\cos \left( {2x + 3y} \right)\\
\frac{{{\partial ^2}f}}{{\partial {x^2}}} =&- \cos \left( {x + 2y} \right) - 4\cos \left( {2x + 3y} \right)\\
\frac{{{\partial ^2}f}}{{\partial {y^2}}} =&- 4\cos \left( {x + 2y} \right) - 9\cos \left( {2x + 3y} \right)
\end{align*}
$$\left( 0,0\right):\frac{{{\partial ^2}f}}{{\partial {x^2}}}=-5<0$$
\[\left| {{\nabla ^2}f\left( {0,0} \right)} \right| = \left| {\begin{array}{*{20}{c}}
{ - 5}&{ - 8}\\
{ - 8}&{ - 13}
\end{array}} \right| = 13 \cdot 5 - 64 = 1 < 0\]
$\Rightarrow\nabla^2 f\left( 0,0\right)$ ist negative definiert und $\left( 0,0\right)$ ist eine lokale maximalestelle.\\

Auch alle punkte $\left( -2\pi k,2\pi l\right)$ sind lokale maxima. Analog, bis auf Addition von Vielfachen von $2\pi$ $f$ hat lokale minimalestelle in $\left( \pi,\pi\right)$ und Sattelpunkte in $\left( 0,\pi\right)$ und $\left( \pi,0\right)$
\end{enumerate}

\section{Vektorwertige Funktionen}
Sei $\Omega\in\mathbb{R}^n$, $f=\left( f^i\right)_{1 < i < l}\Omega\rightarrow\mathbb{R}^l$
\begin{definition}{8.57}
\begin{enumerate}
\item Die Funktion \todo[inline]{can,t understand the function, page 203 top} heisst an der stelle $x_0\in\Omega$ differenzierbar, falls jede komponente $f^i$, $1 < i < l$ an der stelle $x_0$ differenzierbar ist.\\

Das Differential $df\left( x_0\right)$ hat die Gestalt \[df\left( {{x_0}} \right) = \left( {\begin{array}{*{20}{c}}
{df'\left( {{x_0}} \right)}\\
{d{f^n}\left( {{x_0}} \right)}
\end{array}} \right)\]
\item $f$ heisst auf $\Omega$ differenzierbar (bzw. von der Klasse $C^m$, $m\geq 1$) falls jedes $f^i$ differenzierbar ist (bzw. $f^i\in C^m\left( \Omega\right)$) $1\leq i \leq l$
\end{enumerate}
\end{definition}
\subsubsection*{Bemerkung 8.58}
\begin{enumerate}
\item Bezüglich der Standardbasis $dx^j$, $1\leq j\leq n$ erhalten wir 
\[d{f^i}\left( {{x_0}} \right) = \sum\limits_{j = 1}^n {\frac{{\partial {f^i}}}{{\partial {x^j}}}\left( {{x_0}} \right)d{x^j} = \left( {\frac{{\partial {f^i}}}{{\partial x'}}\left( {{x_0}} \right), \ldots ,\frac{{\partial {f^i}}}{{\partial {x^n}}}\left( {{x_0}} \right)} \right)} \]
die Darstellung
\[df\left( {{x_0}} \right) = \left( {\begin{array}{*{20}{c}}
{\frac{{\partial f'}}{{\partial x'}}\left( {{x_0}} \right)}&{\frac{{\partial f'}}{{\partial {x^2}}}\left( {{x_0}} \right)}& \ldots &{\frac{{\partial f'}}{{\partial {x^n}}}\left( {{x_0}} \right)}\\
{\frac{{\partial {f^2}}}{{\partial x'}}\left( {{x_0}} \right)}&{\frac{{\partial {f^2}}}{{\partial {x^2}}}\left( {{x_0}} \right)}& \ldots &{\frac{{\partial {f^2}}}{{\partial {x^n}}}\left( {{x_0}} \right)}\\
 \vdots & \vdots & \ddots & \vdots \\
{\frac{{\partial {f^l}}}{{\partial x'}}\left( {{x_0}} \right)}&{\frac{{\partial {f^l}}}{{\partial x'}}\left( {{x_0}} \right)}& \ldots &{\frac{{\partial {f^l}}}{{\partial {x^n}}}\left( {{x_0}} \right)}
\end{array}} \right)\]
Die $l\times n$ matrix $df\left( x_0\right) = {\left( \frac{\partial f^i}{\partial x^j}\left( x_0\right)\right)}_{1\leq i\leq l, 1\leq j\leq n}$ heisst Jacobi oder Funktionalmatrix von $f$ an der Stelle $x_0$.
\item Auch im Vektorwertigen Fall ist die Funktion $f$ genau dann differenzierbar in $x_0$, wenn eine lineare Abbildung $A:\mathbb{R}^n\rightarrow\mathbb{R}^l$ existiert mit 
\[\mathop {\lim }\limits_{x \to {x_0}} \frac{{f\left( x \right) - f\left( {{x_0}} \right) - A\left( {x - {x_0}} \right)}}{{\left| {x - {x_0}} \right|}} = 0\]
\end{enumerate}

\subsubsection*{Beispiel 8.59}
$$f:\mathbb{R}^2\rightarrow\mathbb{R}$$
\[f\left( {x,y} \right) = \left( {\begin{array}{*{20}{c}}
{{x^2} - {y^2}}\\
{2xy}
\end{array}} \right)\]
$$f\in C^\infty\left( \mathbb{R}^2\mathbb{R}^2\right) \text{ mit } df\left( {x,y} \right) = \left( {\begin{array}{*{20}{c}}
{2x}&{ - 2y}\\
{2y}&{2x}
\end{array}} \right) $$\todo[inline]{Can't understant the symbol between the $\mathbb{R}^2$, page 205 middle to top}
Es gelten die üblichen Differentiationsregeln

\subsubsection*{Satz 8.60}
Sei $f,g:\Omega\subset\mathbb{R}^n\rightarrow\mathbb{R}^l$ an der Stelle $x_0\in\Omega$ differenzierbar und $\alpha\in\mathbb{R}$. dann sin die Funktionen $\alpha f$ und $f+g$ sowie das Skalarprodukt von $f$ und $g$ an der Stelle $x_0$ differenzierbar und 
\begin{enumerate}
\item $d\left( \alpha f\right) \left( x_0\right)=\alpha df\left( x_0\right)$ 
\item $d\left( f+g\right) \left( x_0\right)=df\left( x_0\right) + dg\left( x_0\right)$
\item $d\left( f\cdot g\right)\left( x_0\right) = f\left( x_0\right)\cdot dg\left( x_0\right)+g\left( x_0\right)\cdot df\left( x_0\right)$\\
wobei $f\left( x_0\right)\cdot dg\left( x_0\right)=\sum\limits_{i = 1}^l {{f^i}\left( {{x_0}} \right)d{g^i}\left( {{x_0}} \right)}$
\end{enumerate}
\subsubsection*{Satz 8.61}
Seien $g:\Omega\rightarrow\mathbb{R}^l$ an der Stelle $x_0\in\Omega$ und $f:\mathbb{R}^l\rightarrow\mathbb{R}^m$ an der Stelle $g\left(x_0\right)$ differenzierbar.\\

Dann ist die Funktion $f\circ g:\Omega\rightarrow\mathbb{R}^m$ an der Stelle $x_0$ differenzierbar, und 
$$d\left( f\circ g\right)\left( x_0\right)=df\left( g\left( x_0\right)\right)\cdot dg\left( x_0\right)$$

\subsubsection*{Beispiel}
Sei
\begin{align*}
f:\mathbb{R}^2 & \rightarrow\mathbb{R}^2\\
\left( x,y\right) & \rightarrow\left( {\begin{array}{*{20}{c}}
{{x^2} - {y^2}}\\
{2xy}
\end{array}} \right) \hspace{20mm}df = \left( {\begin{array}{*{20}{c}}
{2x}&{ - 2y}\\
{2y}&{2x}
\end{array}} \right)
\end{align*} 
\begin{align*}
g:\mathbb{R}^3 & \rightarrow\mathbb{R}^3\\
\left( x,y,z\right) & \rightarrow\left( {\begin{array}{*{20}{c}}
{{x^2} +{y^2}+{z^2}}\\
{xyz}
\end{array}} \right) \hspace{15mm}dg = \left( {\begin{array}{*{20}{c}}
{2x}&{2y}&{2z}\\
{yz}&{xz}&{xy}
\end{array}} \right)
\end{align*} 
\begin{align*}
\left( f\circ g\right):\mathbb{R}^3 & \rightarrow\mathbb{R}^2\\
\left(x,y,z\right)&\rightarrow \left( {\begin{array}{*{20}{c}}
{{{\left( {{x^2} + {y^2} + {z^2}} \right)}^2} - {{\left( {xyz} \right)}^2}}\\
{2\left( {{x^2} + {y^2} + {z^2}} \right)\left( {xyz} \right)}
\end{array}} \right)
\end{align*}
\begin{align*}
d\left( {f \circ g} \right)\left( {x,y,z} \right)&= df\left( {g\left( {x,y,z} \right)} \right) \cdot dg\left( {x,y,z} \right)\\
&= \left( {\begin{array}{*{20}{c}}
{2\left( {{x^2} + {y^2} + {z^2}} \right)}&{ - 2xyz}\\
{2xyz}&{2\left( {{x^2} + {y^2} + {z^2}} \right)}
\end{array}} \right)\left( {\begin{array}{*{20}{c}}
{2x}&{2y}&{2z}\\
{yz}&{xz}&{xy}
\end{array}} \right)\\
&=\left( {\begin{array}{*{20}{c}}
{4x\left( {{x^2} + {y^2} + {z^2}} \right) - 2x{y^2}{z^2}}& * & * \\
 * & * & * \\
 * & * & * 
\end{array}} \right)
\end{align*}