\chapter{Folgen und Reihen (Der Limes Begriff)}
\section{Folgen, allgemeines}
\begin{framed}
\centerline{\textbf{Definition 3.1}}
\noindent Eine Folge reeler zahlen ist eine Abbildung $a:\mathbb{N}\backslash\{0\}\rightarrow\mathbb{R}$ wobei wir das Bild con $n\geq 1$ mit $a_n$ (statt $a(n)$) bezeichen.\\

Eine Folge wird dann meistens mit $(a_n)_{n\geq 1}$, daher mit der geordneten Bildmenge bezeichnet.
\end{framed}

\noindent  Folgen können auf verschiedene Arten gegeben sein.
\subsubsection*{Beispiel 3.2}
\begin{enumerate}
\item $a_n=\frac{1}{n}$, $n\geq 1$
\item $a_1=0.9$, $a_2=0.99$, \dots, ${a_n} = 0.\underbrace {99 \ldots 9}_{n-\text{mal}}$
\item $a_n=\left( 1+\frac{1}{n}\right)^n$, $n\geq 1$
\item (Rekursiv) Sei $d>0$ eine reelle Zahl
\end{enumerate}