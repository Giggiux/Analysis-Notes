\documentclass[a4paper]{book}

\usepackage{enumerate}%used for custom enumerates
\usepackage{amsmath}%Math package
\usepackage{amssymb}%Math Package
\usepackage[utf8]{inputenc}%for the umlaut
\usepackage[german]{babel}%localized string of Chapter and other keywords
\usepackage{framed}%Framed used for the definitions
\usepackage{tabularx}%needed for the tebles environment
\usepackage{tikz}%used to draw all of the graphs
\usetikzlibrary{arrows}%custom arrows for chapter 1
\usepackage{pgfplots}%used for drawn graphs, for the figures
\usepackage{xcolor}%Color package, used for some diagrams
\usepackage{romanbar}%Used for chapter one for the X with over and underline
\newcolumntype{Y}{>{\centering\arraybackslash}X}%Used to center the text in the columns
\usepackage{fancyhdr}%Styling package used to dispaly custom headers, as well as custom chapter numbering in page numbering
\usepackage{float}%Used to position images in specified places
\pgfplotsset{compat=newest}%used for the pgfplots
\usetikzlibrary{decorations.pathmorphing,patterns}
\usepackage[bookmarks,bookmarksdepth=2]{hyperref}%Depth of index in pdf program
\usetikzlibrary{calc} %required for cube drawing in chapter 2
\usetikzlibrary{patterns,decorations.pathreplacing}%required for underbrace in tikz

% THESE PACKAGES ARE EDITING HELPERS, PROVIDING TOOLS FOR WHEN EDITING. THEY CAN BE DISABLED WHEN DONE
%\usepackage{showframe}%used to show margins in the preview, deactivate at the end
\usepackage{verbatim}%used for multiline comments, can be removed at the end
\usepackage{todonotes}%Notes to add throughout the document
\newenvironment{definition}[1]{\begin{framed}\centerline{\textbf{Definition #1}}\noindent\hspace{-1.1mm}}{\end{framed}}
\newenvironment{beweis}[1]{\subsubsection*{Beweis #1}}{\begin{flushright}$\blacksquare$\end{flushright}}
\DeclareMathOperator{\arccosh}{arccosh}%Defined here since it doesn't exist
\DeclareMathOperator{\arcsinh}{arcsinh}%Defined here since it doesn't exist
\DeclareMathOperator{\arctanh}{arctanh}%Defined here since it doesn't exist
\newcommand{\verteq}[0]{\begin{turn}{90} $=$\end{turn}}%added vertical equal sign
\DeclareMathOperator{\Hess}{Hess}%Defined here since it doesn't exist

\begin{document}
%The following generates a new chapter
\chapter{Title of a chapter}

%Subchapter
\section{Title of the section}
%Definition
\begin{definition}{1.1}
This is the environment for the definition. It can also contain equations, like so:
\[\frac{{ - b \pm \sqrt {{b^2} - 4ac} }}{{2a}}\]
\end{definition}

%Beweis
\subsubsection*{Beweis 1.2}
This is the text for a beweis. This will be substituted with a custom environment, but for now just use this method. DO NOT FORGET THE ``*''

%Bemerkung
\subsubsection*{Bemerkung 1.3}
Just like for the Beweis.\\

\noindent When typing up math formulas, there are 2 ways:
\begin{enumerate}
\item Inline Mode: Use a single $\$$ symbol to display an equation like this: $3x^2+6$
\item Math mode: Use $\$\$$ symbols as delimiters, to display an equation like this: $$3x^2+6$$
\end{enumerate}


%Lining up multiple lines:
In order to line up multiple lines, use the ``align*'' environment to have such a result: 
\begin{align*}
	f(x) & =3(x+2)\\	
	& = 3x+6
\end{align*}
Feel free to move the ``\&'' around, as this is the marker used to line up the equations. 


\end{document}

