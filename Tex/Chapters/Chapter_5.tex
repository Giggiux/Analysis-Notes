\chapter{Differentialrechnung auf $\R$}
\section{Differential (Ableitung), Elementare Eigenschaften}
\begin{definition}{5.1}
Sei $f:\Omega\to\R$, $\Omega\subset\R$, und $x_0\in\Omega$
\begin{enumerate}
\item $f$ heisst differenzierbar an der Stelle $x_0$ falls 
\[\mathop {\lim }\limits_{\begin{array}{*{20}{c}}
{x \to {x_0}}\\
{x\not  = {x_0}}
\end{array}} \frac{{f\left( x \right) - f\left( {{x_0}} \right)}}{{x - {x_0}}}\]
existiert. Dieser Grenzwert wird dann mit $f'\left( x_0\right)$ oder $\frac{df}{dx}\left( x_0\right)$ bezeichnet. Die Zahl $f'\left( x_0\right)$ heisst die Ableitung oder das Differential von $f$ an der Stelle $x_0$
\item $f$ heisst in $\Omega$ differenzierbar, falls sie an jeder Stelle $x_0\in\Omega$ differenzierbar. In diesem Fall, nennt sich die Funktion $x\to f'(x)$ Ableitung von $f$
\end{enumerate}
\end{definition}
\subsubsection*{Bemerkung 5.2}
In der Definition 5.1, verlangen wir also, dass für jede in $\Omega \backslash\{ x_0\}$ erhaltene folge $\left( x_n\right)_{n\geq 1}$ mit Grenzwert $x_0$, der Limes 
\subsubsection*{Bemerkung 5.3}
Sei $f$ differenzierbar in $x_0$ 
\missingfigure{Page 184 bottom}
Dann ist \[\frac{{f\left( x \right) - f\left( {{x_0}} \right)}}{{x - {x_0}}}\] die Steigung der Geraden durch die Punkte $\left( x_0,f\left( x_0\right) \right)$ und $\left( x,f\left( x\right) \right)$.\\

Geometrisch ist also $f'\left( x_0\right)$ die Steigung der Tangenten am Graphen von $f$ im Punkt $\left( x_0,f\left( x_0\right) \right)$. Diese Tangente hat die Gleichung 
\[T(x) = f'\left( {{x_0}} \right)\left( {x - {x_0}} \right) + f\left( {{x_0}} \right)\]
Sei 
\[f\left( x \right)= f'\left( {{x_0}} \right)\left( {x - {x_0}} \right) + f\left( {{x_0}} \right) + {R_{{x_0}}}\left( x \right) = T\left( x \right) + {R_{{x_0}}}\left( x \right)\]
\[ \Rightarrow \frac{{f\left( x \right) - f\left( {{x_0}} \right)}}{{x - {x_0}}} = \frac{{f'\left( {{x_0}} \right)\left( {x - {x_0}} \right)}}{{x - {x_0}}} + \frac{{R\left( x \right)}}{{x - {x_0}}}\]
Dann folgt 
\[\mathop {\lim }\limits_{\begin{array}{*{20}{c}}
{x \to {x_0}}\\
{x\not  = {x_0}}
\end{array}} \frac{{R\left( x \right)}}{{x - {x_0}}} = 0\]
Die Lineare Funktion $f\left( {{x_0}} \right) + f'\left( {{x_0}} \right)\left( {x - {x_0}} \right)$ stellt eine gute Approximation der Funktion $f(x)$ dar:\\

\noindent Es gilt
\[f\left( x \right) = f\left( {{x_0}} \right) + f'\left( {{x_0}} \right)\left( {x - {x_0}} \right) + {R_{{x_0}}}\left( x \right)\]
mit 
\[\mathop {\lim }\limits_{x \to {x_0}} \frac{{R\left( x \right)}}{{x - {x_0}}} = 0\]
\missingfigure{page 186, very top}
\subsubsection*{Beispiel 5.4}
\begin{enumerate}
\item \begin{align*}
f:\R&\to\R\\
x&\to mx+b
\end{align*}
ist überall differenzierbar mit $f'\left( x\right)=m$, $\forall x\in\R$ 
\begin{align*}
f\left( x \right) - f\left( {{x_0}} \right)&= m\left( {x - {x_0}} \right)\\
\mathop {\lim }\limits_{x \to {x_0}} \frac{{f\left( x \right) - f\left( {{x_0}} \right)}}{{\left( {x - {x_0}} \right)}}&= m
\end{align*}
\item $f\left( x\right)=\abs{x}$ ist für alle $x_0\not=0$ differenzierbar aber nicht für $x_0=0$
\[f\left( x \right) - f\left( 0 \right) = \left\{ {\begin{array}{*{20}{c}}
x&{{\text{ für }}x \ge 0}\\
{ - x}&{{\text{ für }}x \le 0}
\end{array}} \right.\]
Also ist \[\frac{{f\left( x \right) - f\left( 0 \right)}}{{x - 0}} = \left\{ {\begin{array}{*{20}{c}}
1&{{\text{ fur }}x > 0}\\
{ - 1}&{{\text{ fur }}x < 0}
\end{array}} \right.\]
Besitzt also keinen Grenzwert für $x\to 0$, $x\not=0$
\item $\exp :\R\to\R$ ist überall auf $\R$ differenzierbar und $\exp'(x)=\exp(x)$. Sei $x_0\in\R$, $x_0\not=x=x_0+h\in\R$ 
\begin{align*}
\exp \left( {{x_0} + h} \right) - \exp \left( {{x_0}} \right)&= \exp \left( {{x_0}} \right)\left( {\exp \left( h \right) - 1} \right)\\
\exp \left( h \right) - 1&= h + \frac{{{h^2}}}{{2!}} +  \ldots \\
 \Rightarrow \frac{{\exp \left( h \right) - 1}}{h}&= 1 + \frac{h}{{2!}} + \frac{{{h^2}}}{{3!}}+\dots
\end{align*}
Also 
\begin{align*}
\left| {\frac{{\exp \left( h \right) - 1}}{h} - 1} \right|&\le \left| h \right|\left[ {\frac{1}{{2!}} + \frac{{\left| h \right|}}{{3!}} + \frac{{{{\left| h \right|}^2}}}{{4!}} +  \ldots } \right]\\
&\le \left| h \right|\left[ {1 + \left| h \right| + \frac{{{{\left| h \right|}^2}}}{{2!}} +  \ldots } \right]\\
&\le \left| h \right|\exp \left( h \right)
\end{align*}
Woraus 
\[\mathop {\lim }\limits_{\begin{array}{*{20}{c}}
{h \to 0}\\
{h\not  = 0}
\end{array}} \frac{{\exp \left( h \right) - 1}}{h} = 1\]
und somit 
\begin{align*}
{\mathop{\rm exp'}\nolimits} \left( {{x_0}} \right)&= \mathop {\lim }\limits_{\begin{array}{*{20}{c}}
{h \to 0}\\
{h\not  = 0}
\end{array}} \frac{{\exp \left( {{x_0} + h} \right) - \exp \left( {{x_0}} \right)}}{h}\\
&= \mathop {\lim }\limits_{h \to 0} \exp \left( {{x_0}} \right)\left( {\frac{{\exp \left( h \right) - 1}}{h}} \right)\\
&= \exp \left( {{x_0}} \right)
\end{align*}
\item $\sin(x)$ und $\cos(x)$ sind überall differenzierbar und 
\begin{align*}
\sin'&=\cos\\
\cos'&=-\sin
\end{align*}
Aus der Additionsgesetzen:
\begin{align*}
\sin \left( {x + h} \right) - \sin \left( x \right) &= \sin \left( x \right)\cos \left( h \right) + \cos \left( x \right)\sin \left( h \right) - \sin \left( x \right)\\
& = \sin \left( x \right)\left( {\cos \left( h \right) - 1} \right) + \cos \left( x \right)\sin \left( h \right)
\end{align*}
Nun ist 
\[\mathop {\lim }\limits_{h \to 0} \frac{{\sin \left( h \right)}}{h} = 1\]
und 
\begin{align*}
\frac{{\cos \left( h \right) - 1}}{h} &= \frac{{{{\cos }^2}\left( h \right) - 1}}{{h\left( {\cos \left( h \right) + 1} \right)}} = \frac{{{{\sin }^2}\left( h \right)}}{{h\left( {\cos \left( h \right) + 1} \right)}}\\
& = \frac{1}{{\mathop {\cos \left( h \right) + 1}\limits_{\begin{array}{*{20}{c}}
 \downarrow \\
{\frac{1}{2}}
\end{array}} }} \cdot \frac{{{{\sin }^2}\left( h \right)}}{{\mathop h\limits_{\begin{array}{*{20}{c}}
 \downarrow \\
0
\end{array}} }}
\end{align*}
\todo[inline]{There is a sin h/h which doesn't seem to belong anywhere, page 188 bottom right corner}
\begin{align*}
\frac{{\sin \left( {x + h} \right) - \sin \left( x \right)}}{h} &= \sin \left( x \right)\left( {\frac{{\cos \left( h \right) - 1}}{h}} \right) + \cos \left( x \right)\frac{{\sin \left( h \right)}}{h}\\
 \Rightarrow \mathop {\lim }\limits_{h \to 0} \frac{{\sin \left( {x + h} \right) - \sin \left( x \right)}}{h} &= \lim \left( {\sin \left( x \right)\mathop {\lim }\limits_{h \to 0} \left( {\frac{{\cos \left( h \right) - 1}}{h}} \right)} \right.\\
&\left. { \hspace{3mm}+ \cos \left( x \right)\mathop {\lim }\limits_{h \to 0} \left( {\frac{{\sin \left( h \right)}}{h}} \right)} \right)\\
 &= \sin \left( x \right)\lim \left( {\frac{{\cos \left( h \right) - 1}}{h}} \right)\\
 &\hspace{3mm}+ \cos \left( x \right)\lim \left( {\frac{{\sin \left( h \right)}}{h}} \right)\\
& = \left( {\sin \left( x \right)} \right) \cdot 0 + \left( {\cos \left( x \right)} \right) \cdot 1 = \cos \left( x \right)
\end{align*}
Analog 
\begin{align*}
\cos \left( {x + h} \right) - \cos \left( x \right) &= \cos \left( x \right)\cos \left( h \right) - \sin \left( x \right)\sin \left( h \right) - \cos \left( x \right)\\
& = \cos \left( x \right)\left( {\cos \left( h \right) - 1} \right) + \sin \left( x \right)\sin \left( h \right)
\end{align*}
Da wie oben $\frac{{\cos \left( h \right) - 1}}{h} \to 0$, $\frac{{\sin \left( h \right)}}{{\left( h \right)}} \to 1$ folgt $\cos'=-\sin$
\end{enumerate}
Der Zusammenhang zwischen differenzierbarkeit und stetigkeit ist
\subsubsection*{Satz 5.5}
Sei $\Omega\subseteq\R$, $x_0\in\Omega$ und $f:\Omega\to\R$ in $x_0$ differenzierbar. Dann ist $f$ in $x_0$ stetig. (Also, ``Diff'' ist mehr als ``Stetigkeit'')
\begin{beweis}{}
$f$ differenzierbar in $x_0$. Sei 
\begin{align*}
T:\Omega \backslash \left\{ {{x_0}} \right\}&\to\R\\
 x &\to \frac{{f\left( x \right) - f\left( {{x_0}} \right)}}{{x - {x_0}}}
\end{align*}
Da $f$ differenzierbar in $x_0$ ist, hat $T$ ein Grenzwert in $x_0$, und 
\[\mathop {\lim }\limits_{x \to {x_0}} T\left( x \right) = f'\left( x \right)\]
Für $x\not=x_0$ 
\[f\left( x \right) = T\left( x \right)\left( {x - {x_0}} \right) + f\left( {{x_0}} \right)\]
$f\left( x \right)$ ist die Summe von 2 funktionen $T\left( x\right)\left( x-x_0\right)$ und $f\left( x_0\right)=$ konstant.\\

Da beide funktionen ein Grenzwert an der Stelle $x_0$ besitzen, hat auch $f$ eine Grenzwert in $x_0$ und 
\begin{align*}
\mathop {\lim }\limits_{x \to {x_0}} f\left( x \right)&= \mathop {\lim }\limits_{x \to {x_0}} \left( {T\left( x \right)} \right)\mathop {\lim }\limits_{x \to {x_0}} \left( {x - {x_0}} \right) + \mathop {\lim }\limits_{x \to {x_0}} f\left( {{x_0}} \right)\\
&= f'\left( x \right) \cdot 0 + f\left( {{x_0}} \right)= f\left( {{x_0}} \right)
\end{align*}
$\Rightarrow$ ist stetig in $x_0$.
\end{beweis}
\subsubsection*{Bemerkung}
Die Umkehrung von Satz 5.5\todo{Add page + reference, page 190 middle} gilt nicht, z.B. $f\left( x \right)=\abs{x}$ ist stetig in $x=0$ aber nicht differenzierbar. 

\subsubsection*{Beispiel 5.6}
Das folgende Beispiel zeigt dass, es stetige funktionen $f:\R\to\R$ gibt, die an keiner Stelle $x_0\in\R$ differenzierbar sind. (Von der Waerden (1930))\\

\noindent Sei für $x\in\R$ 
\begin{align*}
<x> =&\text{Distanz von $x$ zur nächsten ganzen Zahl}\\
=&\min\left\{ \left| x-m\right|:m\in\mathbb{Z}\right\}
\end{align*}
Der Graph von $<x>$ sieht so aus 
\missingfigure{page 191 top}
Graph von $\frac{10x}{10}$ 
\missingfigure{Page 191, top to middle}
Sei 
\[f\left( x \right): =  < x >  + \frac{{ < 10x > }}{{10}} + \frac{{ < {{10}^2}x > }}{{100}} +  \ldots \]
Da 
\[0 \le  < {10^n}x >  \le \frac{1}{2}\]
folgt absolut konvergenz. Ausserdem sei 
\[{f_k}\left( x \right) = \sum\limits_{n = 0}^k {\frac{{ < {{10}^n}x > }}{{{{10}^n}}}} \]
Dann ist 
\[\left| {f\left( x \right) - {f_k}\left( x \right)} \right| = \left| {\sum\limits_{n = k + 1}^\infty  {\frac{{ < {{10}^n}x > }}{{{{10}^n}}}} } \right| \le \frac{1}{2}\left| {\sum\limits_{n = k + 1}^\infty  {\frac{1}{{{{10}^n}}}} } \right| = \frac{1}{2} \cdot \frac{{{{10}^{ - k}}}}{9}\]
$\forall k\geq 1$ ist $f_k:\R\to\R$ stetig. \\

Da die Folge $\left( f_k\right)_{k\geq 1}$ gleichmässig gegen $f$ konvergiert ist $f$ stetig. Man kann zeigen, dass $f$ in keinem Punkt von $\R$ differenzierbar ist. \todo{End of beweis is put here, I think it is better if it stays up when the bsp begins. Page 192 middle}
\subsubsection*{Satz 5.7}
Seien $f,g:\Omega\to\R$ Funktionen, $x_0\in\R$. Wir nehmen an dass $f$ und $g$ in $x_0$ differenzierbar sind. Dann sind $f+g$, $f\cdot g$ und falls $g\left( x_0\right)\not=0$ auch $f/g$\todo{Is this supposed to be a fraction?? page 192 bottom} an der Stelle $x_0$ differenzierbar. Es gelten dann folgende Formel:
\begin{enumerate}
\item $\left( {af + bg} \right)'\left( {{x_0}} \right) = af'\left( {{x_0}} \right) + bf'\left( {{x_0}} \right)\hspace{5mm}\forall a,b \in \R$
\item $\left( {f \cdot g} \right)'\left( {{x_0}} \right) = f'\left( {{x_0}} \right) \cdot g\left( {{x_0}} \right) + f\left( {{x_0}} \right) \cdot g'\left( {{x_0}} \right)$
\item $\left( {\frac{f}{g}} \right)'\left( {{x_0}} \right) = \frac{{f'\left( {{x_0}} \right) \cdot g\left( {{x_0}} \right) - f\left( {{x_0}} \right) \cdot g'\left( {{x_0}} \right)}}{{g{{\left( {{x_0}} \right)}^2}}}$
\end{enumerate} 
%Page 193 top
\begin{beweis}{}

\end{beweis}
