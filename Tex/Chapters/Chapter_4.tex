\chapter{Stetigkeit}
% Starts at p127 (week8)
\section{Grenzwerte von Funktionen}
Sei $\Omega \subset \R^{d}$ eine Teilmenge und $f:\Omega \to \R^{n}$ eine Abbildung.

\begin{definition}{4.1}
$f$ hat an der Stelle $x_{0} \in \R^{d}$ den \emph{Grenzwert} $a$, falls für jede Folge $(x_k)_{k\in \N}$ in $\Omega$ mit $x_{k} \to x_{0}\, (k \to \infty)$ gilt $f(x_{k}) \to a$. \\

Wir schreiben: $\lim_{x\to x_{0}}{f(x)} = a$
\end{definition}
\textbf{Bemerkung}: $x_{0}$ muss nicht im Definitionsbereich von $f$ sein.
\begin{definition}{4.2}
$f:\Omega \to \R^{d}$ heisst \emph{stetig} an der Stelle $x_{0} \in \Omega$ falls:
\begin{enumerate}
\item $f$ an der Stelle $x_{0}$ definiert ist,
\item $\lim_{x\to x_{0}}{f(x)}$ existiert, und
\item $\lim_{x\to x_{0}}{f(x)} = f(x_{0})$.
\end{enumerate}
\end{definition}
\begin{definition}{4.2'}
Die Abbildung $f:\Omega \to \R^{n}$ ist im Punkt $x_{0} \in \Omega$ \emph{stetig}, falls für jede gegen $x_{0}$ konvergierende Folge $(x_{n})_{n\geq 1}$ in $\Omega$, die Folge $(f(x_{n}))_{n\geq 1}$ zum Grenzwert $f(x_{0})$ konvergiert, d.h.
\[\lim_{n\to\infty}f(x_{n}) = f(\lim_{n\to\infty}x_{n})\]
Anders gesagt:
\begin{itemize}
\item Grenzwerte von Folgen werden von stetigen Funktionen nicht verändert.
\item Stetige Funktionen erhalten Grenzwerte von Folgen.
\end{itemize}
\end{definition}
\begin{definition}{4.2''}
Die Abbildung $f:\Omega\to\R^{n}$ ist auf $\Omega$ \emph{stetig} (oder \emph{einfach stetig}, wenn der Kontex klar ist), falls $f$ in jedem Punkt $x\in\Omega$ stetig ist.
\end{definition}
\subsection*{Beispiele}
Mittels Resultate aus dem dritten Kapitel haben wir wichtige Beispiele von stetigen Funktionen.
\begin{itemize}
\item Diese Funktion ist auf ganz $\R\times\R$ stetig: \begin{align*}f:\R\times\R&\to\R \\ (a,b) &\mapsto (a+b) \end{align*} (Seien $(a_{n}), (b_{n})$ Folgen mit $a = \lim{a_{n}}, b = \lim{b_{n}}$. Dann ist die Folge $(a_{n} + b_{n})$ konvergent, und $\lim{a_{n}+b_{n}} = a + b$, nach Satz 3.8) 
\item Diese Funktion ist auf ganz $\R\times\R$ stetig:
\begin{align*}f:\R\times\R&\to\R \\ (a,b)&\mapsto ab \end{align*}
\item Diese Funktion is auf $\R\times\R^{x}$ stetig:
\begin{align*}f:\R\times\R^{x}&\to\R \\ (a,b)&\mapsto a/b \end{align*}
\item Aus wiederholter Anwendung von 1. und 2. ergibt sich die \emph{Polynomiale Funktion\todo{heisst die wirklich so?}}:
\[\text{Sei } n \geq 0,\: a_{0}, \dots, a_{n}\in\R: p(x)\defeq a_{0} + a_{1}x + \dots + a_{n}x^{n} \]
Die Polynomiale Funktion ist stetig auf ganz $\R$.
\item Die beiden folgenden Abbildungen sind stetig auf ihrem Definitionsbereich. 
\begin{figure}[htbp]
\begin{minipage}[t][4mm][b]{0.5\textwidth}
\begin{align*} \R^{d}\times\R^{d}&\to\R^{d} \\ (a,b)&\mapsto(a+b)\end{align*} 
\end{minipage}
\begin{minipage}[t][4mm][b]{0.5\textwidth}
\begin{align*}  \R\times\R^{d}&\to\R^{d} \\ (\lambda,a)&\mapsto\lambda a \end{align*}
\end{minipage}
\end{figure}

\item Die folgenden Abbildungen sind stetig.
\begin{figure}[htpb]
\begin{minipage}[t][7mm][b]{0.31\textwidth}
\begin{align*} \mathbb{C} &\to \mathbb{C} \\ z &\mapsto \bar{z}\end{align*}
\end{minipage}
\begin{minipage}[t][7mm][b]{0.31\textwidth}
\begin{align*} \mathbb{C}\times\mathbb{C} &\to \mathbb{C} \\ (z,w) &\mapsto z*w \end{align*}
\end{minipage}
\begin{minipage}[t][7mm][b]{0.31\textwidth}
\begin{align*} \mathbb{C}\times\mathbb{C}^{x} &\to \mathbb{C} \\ (z,w) &\mapsto z/w \end{align*}
\end{minipage}
\end{figure}
\item Die folgenden Funktionen sind auf [...] \todo{what goes there? p130 (week8sem1)} stetig:
\begin{figure}[htpb]
\begin{minipage}[t][1cm]{0.5\textwidth}
\begin{align*} \R^{d}&\to\R \\ x&\mapsto \norm{x} \end{align*}
\end{minipage}
\begin{minipage}[t][1cm]{0.5\textwidth}
\begin{align*} \mathbb{C}&\to\R \\ z&\mapsto \abs{z} \end{align*}
\end{minipage}
\end{figure}
%I hope this doesn't break too badly...
\item Die charakteristische Funktion von $\mathbb{Q}$: \\

Sei $f(x) =\mathcal{X}_{\mathbb{Q}} = \begin{cases} 1 &x\in\mathbb{Q} \\ 0 &x\in\R\backslash\mathbb{Q} \end{cases} $ \\

Sei $x\in\R\backslash\mathbb{Q}$ fest mit $(x_{k})\in\mathbb{Q}, x_{k}\to x$. Dann ist ${f(x_{k})=\mathcal{X}(x_{k})=1 \nrightarrow 0=\mathcal{X}(x)}$. \\
(Zu $x\in\R\backslash\mathbb{Q}$, sei $x_{k}$ die an der k-ten Nachkommastelle abgebrochene Dezimaldarstellung von $x$. Dann gilt $x_{k} \in \mathbb{Q}\; \forall k \in \N$ und $x_{k}\to x_{1}$.)
\item Sei $f: \begin{cases} x &x < 1 \\x &x>1 \end{cases}$ \\

\missingfigure{p131, week8 sem1} $f$ ist in $x=1$ nicht stetig, weil $f$ an der Stelle $x=1$ nicht definiert ist. In diesem Beispiel ist die Funktion $f$ nicht stetig, aber sie ist eigentlich eine ``gute'' \todo{Does that really say gute?}Funktion.

\todo{no ozlem number...}
\end{itemize}
\begin{definition}{(Struwe 4.1.3 (ii))}
$\Omega \subset \R^{d}, f:\Omega\to\R^{n}, x_{0}\in\R^{d}\backslash\mathbb{Q}$ so dass $\exists (x_{k})\in\Omega$ mit $\lim{x_{k}=x_{0}}$. \\

Dann ist $f$ an der Stelle $x_{0}$ \emph{stetig ergänzbar} falls $a=\lim{f(x_{k})}$ existiert. In diesem Fall setzen wir \[ f(x_{0} = a\]
Die durch $f(x_{0})=a$ ergänzte Funktion $f$ ist offenbar stetig an der Stelle $x_{0}$.
\end{definition}

\todo{offenbar $\to$ offensichtlich?}
\begin{itemize}
%Random ozlem enumeration ignored for consistency
\item Diese stückweise konstante Funktion ist stetig an jeder Stelle $x_{0} \neq 0$. Sie ist jedoch für $a\neq b$ an der Stelle $x_{0}=0$ nicht stetig ergänzbar. (Struwe Beispiel 4.1.3 (vii))
\begin{align*}f:\R^{x}&\to\R \\ f(x)&=\begin{cases}a &\text{falls } x<0 \\ b &\text{falls } x>0\end{cases} \end{align*}
\missingfigure{p132, week8 Sem1}
\item Sei $f:(a,b)\to\R$ monoton wachsend, d.h. $\forall x,y\in(a,b)$ mit $x\leq y$ folgt $f(x)\leq f(y)$. Sei ausserdem $x_{0}\in (a,b)$. Dann existieren die \emph{links- und rechtsseitigen Grenzwerte} 
\[ f(x_{0}^{+}) \defeq \lim_{\substack{x\to x_{0} \\ x > x_{0} \\ x \downarrow x_{0}}}f(x), \quad\quad f(x_{0}^{-})\defeq \lim_{\substack{x\to x_{0} \\ x < x_{0} \\ x \uparrow x_{0}}}f(x) \]
und $f$ ist stetig an der Stelle $x_{0}$ genau dann, wenn $f(x_{0}^{-})=f(x_{0}^{+})=f(x_{0})$.
\end{itemize}
\subsubsection*{Beweis}
Wir behaupten, dass für jede Folge $(y_{n})_{n\geq 1}$ mit $\{y_{n}:n\geq 1\} \subset (a,x_{0})$ und $\lim{y_{n}} = x_{0}$ die Folge $(f(y_{n}))_{n\geq 1}$ kovergent und der linksseitige Limes $l_{-}(x_{0})$ unabhängig von der Wahl der Folge ist.
\missingfigure{p133, week8 sem1} \\

\noindent Wir betrachten zuächst die ``spezielle'' Folge $x_{n}=(x_{0}-\frac{1}{n})_{n\geq r}$. Hier ist $r$ so gewählt, dass $x_{0}-\frac{1}{r}\geq a$. \\
Dann ist $(f(x_{0}-\frac{1}{n}))_{n\geq r}$ monoton wachsend ($x_{0}-\frac{1}{n+1} > x_{0}-\frac{1}{n}$ und $f$ monoton wachsend) und $(f(x_{0}-\frac{1}{n}))_{n\geq r}$ beschränkt ($f(a)<[...]<f(b)$\todo{missing in source material p134week8sem1}).
\[\text{Sei }l_{-}\defeq\lim_{n\to\infty}{f(x_{0}-\frac{1}{n})}\]
\noindent Wir möchten zeigen, dass für jede $(y_{n})\subset (a, x_{0})$ mit $\lim{y_{n}}=x_{0}$ $\lim{f(y_{n})}$ existiert und $\lim{f(y_{n})}=l_{-}$. \todo{$=l_{-}$ oder $=l$.?}

Da es für jedes $x<x_{0}$ ein $n$ gibt, mit $x\leq x_{0}-\frac{1}{n}$ folgt \[f(x) \leq f(x_{0}-\frac{1}{n} \leq l_{-}\]
Sei nun $(y_{n})_{n\geq 1}$ beliebig in \todo{unreadable p134 mid}$(?a?, x_{0})$ mit $\lim{y_{n}}=x_{0}$. Sei $\varepsilon > 0$, ($y_{n} < x_{0}$) und $n_{0}(\varepsilon)$ mit \[l_{-}-\varepsilon < f(x_{0}-\frac{1}{n})\leq l_{-} \quad \forall n>n_{0}(\varepsilon) \]
Insbesondere
\[l_{-} - \varepsilon < f(x_{0}-\frac{1}{n_{0}(\varepsilon)}) \leq l_{-} \]
Sei jetzt $n_{1}(\varepsilon)=n_{1}(n_{0}(\varepsilon))>0$ so dass \[ x_{n_{0}(\varepsilon)} = x_{0} - \frac{1}{n_{0}(\varepsilon)} < y_{n} < x_{0} = \lim_{n\to\infty}{x_{n}} \quad \forall n \geq n_{1}(\varepsilon)\]
\[ \left((y_{n})<(a,x_{0}), \lim{y_{n}}=x_{0}\right)\]
Da $f$ monoton ist, folgt \[ l_{-} - \varepsilon < f(x_{0}-\frac{1}{n_{0}(\varepsilon)}) \leq f(y_{n}) \leq l_{-} = \lim{f(x_{n})} \]
Insbesondere $\lim{f(y_{n})} = l_{-}$. \\

\noindent Der Beweis für $L_{+}$ verläuft ganz analog. \\

\noindent Nun zur Stetigkeit: Es gilt immer \[ l_{-}(x_{0})\leq f(x_{0}) \leq l_{+}(x_{0}) \]
Falls $l_{-}(x_{0}) < l_{+}(x_{0})$ sei $(t_{n})_{n \geq 1}$ wie folgt definiert:
\[ t_{n}=\begin{cases}x_{0}-\frac{1}{n} &n\text{ gerade} \\ x_{0}+\frac{1}{n} &n\text{ ungerade} \end{cases} \]
Dann gilt $\lim{t_{n}} = x_{0}$. Aber $f(t_{2n+1})-f(t_{n}) \geq l_{+}(x_{0})-l_{+}(x_{0}) >0$, woraus folgt dest \todo{dest? p 135 bottom} $(f(t_{n}))_{n\geq1}$ nicht konvergent. \\
Falls $l_{-}(x_{0})=l_{+}(x_{0})$ folgt die Stetigkeit sofort. 

\subsection*{Satz 4.3}
Sei $f:(a,b)\to\R$ monoton wachsend. Dann ist die Menge der Unstetigkeitspunkte von $f$ entweder endlich oder abzählbar.
\subsection*{Beweis}
Sei $U(f) = \{x\in(a,b):f \text{ ist nicht stetig an x}\}$. Dann ist $\forall x\in U(f), \quad l_{-}(x) < l_{+}(x)$ und wir wählen ein \todo{unreadable.. p136 mid}$g(x)\in ??n(l_{-}(x),l_{+}(x))$ .
Falls $x_{1}<x_{2}$ in $U(f)$ folgt $l_{+}(x_{1})<l_{-}(x_{2})$ und somit $g(x_{1})<g(x_{2})$. Damit ist $g:U(f)\to ??$ \todo{same unreadable character} injektiv. 

\noindent Stetigkeit verhält \todo{verträgt?} sich gut mit den üblichen Operationen auf Funktionen.

\subsection*{Satz 4.4}
Seien $f,g:\Omega \to \R^{n}$ und $x_{0}\in\Omega$. Falls $f$ und $g$ in $x_{0}$ stetig sind, so sind es auch $f+g$ und $\alpha f, \; \alpha \in \R$.

\subsection*{Korollar 4.5}
Falls $f,g$ auf $\Omega$ stetig sind, so sind es $f+g$ und $\alpha f$.
 
\begin{definition}{4.6}
\[ C\,(\,\Omega\, , \R\,) \] bezeichnet die Menge der stetigen Abbildungen $f: \Omega\to\R$. Nach Korollar 4.5 ist es ein Vektorraum.
\end{definition}

\subsection*{Satz 4.7}
Seien $f:\Omega\to\R^{n}, \Omega \subset \R^{d}$ und $g:\Gamma \to \R^{n}$ mit $f(\Omega)\subset\Gamma$ und ${x_{0}\in\Omega}, {y_{0}=f(x_{0})\subset\Gamma}$. Falls $f$ in $x_{0}$ und $g$ in $y_{0}$ stetig sind, folgt, dass $g\circ f:\Omega\to\R^{n}$ in $x_{0}$ stetig ist.

\subsubsection*{Beweis}
Sei $(t_{n})_{n\geq1}$ in $\Omega$ mit $\lim{t_{n}} = x_{0}$. Da $f$ stetig ist, $\lim{f(t_{n})} = f(x_{0}) = y_{0}$, und aus der Stetigkeit von $g$ folgt, dass \[ \lim_{n\to\infty}{g(f(t_{n}))} = g(y_{0}) = (g \circ f)(x_{0}) \]

\subsection*{Korollar 4.8}
Falls $f:\Omega \to \R^{d}, f(\Omega) \subset \Gamma$ und $g:\Gamma \to \R^{m}$, auf $\Omega$ bzw auf $\Gamma$ stetig sind, so folgt, dass $g\circ f:\Omega\to\R^{m}$ auf $\Omega$ stetig ist.  
\section{Stetige Funktionen}
In diesem Abschnitt behandeln wir die erste der fundamentalen Eigenschaften von stetigen Funktionen, nämlich das eine auf einem endlichen Intervall $[a,b]$ (Endpunkte eingeschlossen) stetige Funktion immer ein Max und Min besitzt. Dies veralgemeinern wir dann auf Abbildungen von $\Omega\subseteq\R^{d}$ nach $\R{n}$ wobei $\Omega$ eine spezielle Eigenschaft haben muss (Kompaktheit).

\subsection*{Satz 4.9}
Seien $-\infty < a \leq b < \infty$ und $f:[a,b] \to \R$ stetig. \\
\noindent Dann ist $f([a,b])$ in $\R$ beschränkt und es gibt $c_{-}, c_{+} \in [a,b]$ mit \begin{align*} f(c_{+}) &= \text{sup } \{ f(x):x\in[a,b]\} \\ f(c_{-}) &= \text{inf }\{ f(x):x\in[a,b]\} \end{align*} d.h. Supremum und Infimum werden angenommen.

\subsubsection*{Beweis}
\begin{enumerate}
\item $f([a,b])$ ist nach oben beschränkt (Indirekter Beweis) \\

Falls nicht, so gibt es $\forall n \in \N$ ein $t_{t}\in [a,b]$ mit $f(t_{n}) \geq n$. \\
$(t_{n})_{n\geq 1}$ ist beschränkt, nach Bolzano-Weierstrass. Sei $(t_{l(n)})$ eine konvergente Teilfolge mit $\lim{t_{l(n)}}=x$. \\
Dann ist $x\in [a,b]$, da $a\leq t_{n}\leq b$ \\
(Satz: $(a_{n}), (b_{n})$ konvergente Folgen mit $\lim{a_{n}} = a, \lim{b_{n}} = b$. Falls $a_{n} \leq b_{n}$, folgt $a \leq b$.) \\
Aus der Stetigkeit von $f$ folgt: $\lim_{n\to\infty}{f(t_{n})} = f(x)$. Insbesondere ist $f(t_{l(n)})$ beschränkt, was im Widerspruch mit $f(t_{l(n)})\geq l(n)$ steht. \\

$\implies$ $f$ ist nach oben beschränkt.

\item $f$ ist nach unten beschränkt (analog) \\

Sei $M \defeq \text{Sup }\{f(x):x\in [a,b]\}$, welches als Folge von 1. existiert. \\
Sei für jedes $n\geq 1\; x_{n}\in [a,b]$ mit \[M-\frac{1}{n} < f(x) \leq M \quad\quad (\ast)\]
($M-\frac{1}{n}$ ist kein Supremum $\implies$ $\exists x_{n}$ mit $M-\frac{1}{n} < f(x_{n})$)\\
\item $(x_{n}) \subset [a,b]$ beschränkt. \\
Sei nach Bolzano-Weierstrass $(x_{l(n)})_{n\geq 1}$ eine konvergente Teilfolge mit Limes $c_{+}$. Aus der Stetigkeit von $f$ folgt: \[ f(c_{+}) = \lim_{n\to\infty}{f(x_{l(n)})}\]
Aus $(\ast)$ folgt
\[ \lim_{n\to\infty}{f(x_{l(n)})} = M\]
d.h. $\exists c_{+} \in [a,b]$ mit \[ f(c_{+}) = \lim{f(x_{l(n)})} = M\]
\item Infimum ist ähnlich.
\end{enumerate}

\subsubsection*{Bemerkung}
Satz 4.9 kann man als eine Eigenschaft des Intervalls $[a,b]$ auffassen. Sie gilt zum Beispiel nicht für $(0,1]$ wie das Beispiel der auf $(0,1]$ stetigen Funktion $f(x)=\frac{1}{x}$ zeigt.
\missingfigure{Page 143, week8 Semester1}
Die grundlegende Eigenschaft ist Kompaktheit.
\begin{definition}{4.10}
Eine Teilmenge $K\subset\R^{d}$ heisst \emph{kompakt}, falls jede Folge $(x_{n})_{n\geq 1}$ von Punkten aus $K$ einen Häufungspunkt \emph{in} $K$ besitzt, d.h. falls jede Folge in $K$ eine \emph{in} $K$ konvergierende Teilfolge hat.
\end{definition}

\subsubsection*{Beispiel}
\begin{enumerate}
\item $(0,1]$ ist nicht kompakt: \\
$(\frac{1}{n})_{n\geq 1} \subset (0,1]$ konvergiert gegen $0 \notin (0,1]$.
\item $[a,b]$ ist kompakt. \\
Sei $(t_{n})_{n\geq 1}$ eine Folge mit $a\leq t_{n} \leq b$. $(t_{n})$ ist beschränkt, nach Bolzano-Weierstrass sei $(t_{l(n)})$ eine konvergente Teilfolge mit Limes $l$. Dann folgt aus $a\leq t_{n} \leq b$. $(t_{l(n)})\quad \forall n\geq 1$, dass \[a\leq \lim{t_{l(n)}} \leq b\]
D.h. $l\in [a,b]$.
\end{enumerate}

\subsection*{Lemma 4.11}
Falls $K\subset\R^{d}$ kompakt ist, ist es beschränkt und besitzt zudem ein Minimum und Maximum.

\subsubsection*{Beweis}
Sonst gibt es zu jedem $n\geq 1, n\in \N$ ein $x_{n} \in K$ mit $\norm{x_{n}} \geq n$. Dann kann aber $(x_{n})_{n\geq 1}$ keine konvergente Teilfolge besitzen: $(\abs{x_{l(n)}} > l(n))$. \\
$\implies$ $K$ ist beschränkt. \\
Sei $s\defeq \text{Sup }K$. Dann gibt es $\forall n \geq 1, k_{n} \in K$ mit \[ s-\frac{1}{n}<k_{n}\leq s\]
Insbesondere gilt $\lim{k_{n}}=s$. Da $K$ kompakt ist, hat $k_{n}$ eine \emph{in $K$} konvergierende Teilfolge. Daraus folgt, dass $s\in K$.

\subsubsection*{Beispiel}
$S^{d} \defeq \{ x\in\R^{d+1}: \norm{x} = 1\}$, die d-dimensionale Sphäre, ist kompakt. \\

\subsubsection*{Beweis}
Sei $(x_{n})_{n\geq 1} \subset S^{d}$, dann ist diese Folge offensichtlich beschränkt, besitzt sie (nach Bolzano-Weierstrass) eine konvergente Teilfolge $(x_{l(n)})_{n\geq 1}$. Sei $p \in \R^{d+1}$ deren Limes. Da die Funktion $f(x)\defeq\norm{x}$ stetig ist, folgt \[\norm{p} = f(p) \overset{\text{defn}}=f(\lim{x_{l(n)}}) \overset{f\text{ stetig}}= \lim{f(x_{l(n)})} = 1 \]
$\implies p \in S^{d}$ \\

\noindent Die Verallgemeinerung von Satz 4.9 ist

\subsection*{Satz 4.12}
\begin{enumerate}
\item Sei $K\subset \R^{d}$ kompakt und $f:K\to\R^{n}$ eine stetige Abbildung. Dann ist $f(K)\subseteq\R^{n}$ eine kompakte Teilmenge.
\item $f$ nimmt ihr Supremum und Infimum an, d.h. es gibt $c_{-}, c_{+}\in K$ mit \[ f(x_{-})\leq f(x) \leq f(x_{+}) \quad \forall x \in K\]
\end{enumerate}

\subsubsection*{Beweis}
\begin{enumerate}
\item Sei $(y_{n})_{n\geq 1}$ eine beliebige Folge in $f(K)$. Wir müssen zeigen, dass es eine konvergente Teilfolge mit Limes in $f(K)$ gibt. Sei $(x_{n}) \in K$ mit \[ f(x_{n}) = y_{n},\, n\geq 1\]
Dann ist $(x_{n})_{n\geq 1}$ eine Folge in $K$. Da $K$ kompakt ist, gibt es $p \in K$ und $(x_{l(n)})$, eine konvergente Teilfolge mit $\lim{x_{l(n)}} =p$. \\
Aus der Stetigkeit von $f$ folgt \[ f(p) = f(\lim{x_{n}}) \overset{f\text{ stetig}}= \lim{f(x_{l(n)})} = \lim{y_{l(n)}}\]
D.h. $y_{l(n)}$ ist eine Teilfolge von $y_{n}$ mit Limes $f(p) \in K$. \\
$\implies f(K)$ ist kompakt.
\item Da $f(K)$ kompakt ist, (nach 1.), ist $f(K)$ beschränkt, und besitzt zudem ein Minimum und Maximum (nach Lemma 4.11). \\
\begin{align*}\implies \quad \exists y_{+}, y_{-}\in f(K)\text{, mit } y_{+} &= \text{Sup } f(K) \\ y_{-} &= \text{Inf } f(K) \\
\exists c_{+}, c_{-} \in K \text{, mit } y_{+} &= f(c_{+}) \\ y_{-} &= f(c_{-}) \end{align*}
\end{enumerate}

\section{Norm auf $\R^{d}$}
\todo{In the source notes, this is 4.4, but there is no 4.3 that I can find...}
Der Distanzbegriff auf $\R^{d}$ kommt von einem Skalarprodukt. Es gibt interessante, andere Arten einen Distanzbegriff einzuführen, nämlich mit dem Begriff der Norm.

\begin{definition}{4.13}
Eine \emph{Norm} auf $\R^{d}$ ist eine Abbildung \[ \norm{.}: \R^{d} \to \R \] mit den folgenden Eigenschaften:
\begin{enumerate}
\item \emph{Definiertheit}: $\norm{x} \geq 0$ mit Gleichheit genau dann wenn $x=0$.
\item \emph{Positive Homogenität}: $\norm{\alpha x} = \abs{\alpha}\norm{x} \quad \forall \alpha \in \R, \forall x \in \R^{d}$
\item \emph{Dreiecks-Ungleichung}: $\norm{x+y} \leq \norm{x} + \norm{y} \quad \forall x,y \in \R^{d}$
\end{enumerate}
\end{definition}

\subsubsection*{Beispiel 4.14}
\begin{enumerate}
\item \[ \norm{x}_{2} = \Big(\sum_{i = 1}^{d} \abs{x_{i}}\Big)^{\frac{1}{2}}\quad\quad x=(x_{1}, \dots, x_{d})  \] kommt vom Skalarprodukt.
\item Für $1\leq p<\infty$ sei \[ \norm{x}_{p} \defeq \Big( \sum_{i=1}^{d}\abs{x_{i}^{p}}\Big)^{\frac{1}{p}}\] und $\norm{x}_{\infty} = \max{\{\abs{x_{i}} : 1\leq i \leq d \}}$, dann sind $\norm{.}_{p}, 1\leq p \leq \infty$ Normen auf $\R^{d}$.
\end{enumerate}
Zwischen diesen verschiedenen Normen haben wir die folgenden Verhältnisse:
 \[ \norm{x}_{\infty} = \max{\abs{x_{i}}} \leq \norm{x}_{p} = \sqrt[d]{\sum_{i=1}^{d}{\abs{x_{i}}^{p}}} \leq d\norm{x}_{\infty} \quad\quad (\ast)\]
Bild von $\norm{x}_{1} = \sum_{i=1}^{d}\abs{x_{i}} \leq 1$
\missingfigure{Page 151 week9 Sem1 top}
$\norm{x}_{2} = \sqrt{\sum{x_{i}^{2}}} \leq 1$
\missingfigure{Page 151 week9 Sem1 middle/bottom}
\begin{definition}{4.15}
Zwei Normen $\norm{.}^{(1)}, \norm{.}^{(2)}$ auf $\R^{d}$ heissen \emph{äquivalent}, falls es $c_{1}, c_{2} > 0$ gibt, mit \[ c_{1}\norm{x}^{(1)}\leq\norm{x}^{(2)} \leq c_{2}\norm{x}^{(1)} \quad \forall x \in \R^{d} \]
\end{definition}
\noindent\textbf{Bemerkung}: Sei $C=\max{\{ C_{2}, \frac{1}{C_{1}}\}}$, dann gilt $(\frac{1}{C})\norm{x}^{(1)}\leq\norm{x}^{(2)} \leq C\norm{x}^{(1)}$
\subsubsection*{Beispiel}
Die Normen $\norm{.}_{p} \quad 1\leq p\leq\infty$ sind wegen $(\ast)$ äquivalent.
\subsubsection*{Bemerkung 4.16}
Äquivalente Normen definieren dieselben ``offenen Mengen'' via Distanzfunktion.
\subsubsection*{Beweis}
\todo{marked as skip? p152 week 9 sem1}Für die Normkugeln \[ B_{r}^{(1)}(x_{0})\defeq \{ x:\norm{x-x_{0}}^{(1)}<r\} \] gilt mit $c_{1}\norm{x}^{1}\leq\norm{x}^{2}\leq c_{2}\norm{x}^{1}$ \[ B_{rc_{1}}^{(1)}(x_{0}) \subset B_{r}^{(2)}(x_{0}) \subset B_{c_{2}r}(x_{0})\]
$\implies x_{0} \in \Omega$ innerer Punkt von $\Omega$ bezüglich $\norm{.}^{2} \iff x_{0}\in\Omega$ innerer Punkt von $\Omega$ bezüglich $\norm{.}^{1}$ \\

\noindent Auf $\R^{d}$ haben wir
\subsection*{Satz 4.17}
Je zwei Normen auf $\R^{d}$ sind äquivalent.
\subsubsection*{Beweis}
Es genügt zu zeigen, dass eine beliebige Norm $\norm{.}$ zu $\norm{.}_{2}$ äquivalent ist. \\
Seien $x=\sum{x_{i}e_{i}}$, $y=\sum{y_{i}e_{i}}$.
Dann ist
\begin{align*} \norm{x-y} = \norm{\sum_{i=1}^{d}{(x_{i}-y_{i})e_{i}}} \leq \sum_{i=1}^{d}{\abs{x_{i}-y_{i}}\norm{e_{i}}} &\leq \norm{x-y}\underbrace{\sum^{d}_{i=1}{\norm{e_{i}}}}_{\defeq C} \\
 &\underset{(\ast)}{\leq} C' \norm{x-y}_{2}\end{align*}
\todo{Layout imperfect, but hard to make better.. p153 week9 sem1}
\begin{align*}\text{Also folgt, dass } \R^{d}&\to \R \\ x &\mapsto \norm{x}\text{ stetig ist.}\end{align*}
Da $S^{d-1} = \{ x\in\R^{d}:\norm{x}_{2}=1\}$ kompakt ist, folgt dass es $c_{+}, c_{-}\in S^{d-1}$ gibt, mit ${k_{-}\defeq\norm{c_{-}}}\leq\norm{x}\leq{\norm{c_{+}}\defeq k_{+}} \: \forall x\in S^{d-1}$. Da $c_{0}\neq 0$ folgt $k_{-}>0$.













