\chapter{Stetigkeit}
% Starts at p127 (week8)
\section{Grenzwerte von Funktionen}
Sei $\Omega \subset \R^{d}$ eine Teilmenge und $f:\Omega \to \R^{n}$ eine Abbildung.

\begin{definition}{4.1}
$f$ hat an der Stelle $x_{0} \in \R^{d}$ den \emph{Grenzwert} $a$, falls für jede Folge $(x_k)_{k\in \N}$ in $\Omega$ mit $x_{k} \to x_{0}\, (k \to \infty)$ gilt $f(x_{k}) \to a$. \\

Wir schreiben: $\lim_{x\to x_{0}}{f(x)} = a$
\end{definition}
\textbf{Bemerkung}: $x_{0}$ muss nicht im Definitionsbereich von $f$ sein.
\begin{definition}{4.2}
$f:\Omega \to \R^{d}$ heisst \emph{stetig} an der Stelle $x_{0} \in \Omega$ falls:
\begin{enumerate}
\item $f$ an der Stelle $x_{0}$ definiert ist,
\item $\lim_{x\to x_{0}}{f(x)}$ existiert, und
\item $\lim_{x\to x_{0}}{f(x)} = f(x_{0})$.
\end{enumerate}
\end{definition}
\begin{definition}{4.2'}
Die Abbildung $f:\Omega \to \R^{n}$ ist im Punkt $x_{0} \in \Omega$ \emph{stetig}, falls für jede gegen $x_{0}$ konvergierende Folge $(x_{n})_{n\geq 1}$ in $\Omega$, die Folge $(f(x_{n}))_{n\geq 1}$ zum Grenzwert $f(x_{0})$ konvergiert, d.h.
\[\lim_{n\to\infty}f(x_{n}) = f(\lim_{n\to\infty}x_{n})\]
Anders gesagt:
\begin{itemize}
\item Grenzwerte von Folgen werden von stetigen Funktionen nicht verändert.
\item Stetige Funktionen erhalten Grenzwerte von Folgen.
\end{itemize}
\end{definition}
\begin{definition}{4.2''}
Die Abbildung $f:\Omega\to\R^{n}$ ist auf $\Omega$ \emph{stetig} (oder \emph{einfach stetig}, wenn der Kontex klar ist), falls $f$ in jedem Punkt $x\in\Omega$ stetig ist.
\end{definition}
\subsection*{Beispiele}
Mittels Resultate aus dem dritten Kapitel haben wir wichtige Beispiele von stetigen Funktionen.
\begin{itemize}
\item Diese Funktion ist auf ganz $\R\times\R$ stetig: \begin{align*}f:\R\times\R&\to\R \\ (a,b) &\mapsto (a+b) \end{align*} (Seien $(a_{n}), (b_{n})$ Folgen mit $a = \lim{a_{n}}, b = \lim{b_{n}}$. Dann ist die Folge $(a_{n} + b_{n})$ konvergent, und $\lim{a_{n}+b_{n}} = a + b$, nach Satz 3.8) 
\item Diese Funktion ist auf ganz $\R\times\R$ stetig:
\begin{align*}f:\R\times\R&\to\R \\ (a,b)&\mapsto ab \end{align*}
\item Diese Funktion is auf $\R\times\R^{x}$ stetig:
\begin{align*}f:\R\times\R^{x}&\to\R \\ (a,b)&\mapsto a/b \end{align*}
\item Aus wiederholter Anwendung von 1. und 2. ergibt sich die \emph{Polynomiale Funktion\todo{heisst die wirklich so?}}:
\[\text{Sei } n \geq 0,\: a_{0}, \dots, a_{n}\in\R: p(x)\defeq a_{0} + a_{1}x + \dots + a_{n}x^{n} \]
Die Polynomiale Funktion ist stetig auf ganz $\R$.
\item Die beiden folgenden Abbildungen sind stetig auf ihrem Definitionsbereich. 
\begin{figure}[htbp]
\begin{minipage}[t][4mm][b]{0.5\textwidth}
\begin{align*} \R^{d}\times\R^{d}&\to\R^{d} \\ (a,b)&\mapsto(a+b)\end{align*} 
\end{minipage}
\begin{minipage}[t][4mm][b]{0.5\textwidth}
\begin{align*}  \R\times\R^{d}&\to\R^{d} \\ (\lambda,a)&\mapsto\lambda a \end{align*}
\end{minipage}
\end{figure}

\item Die folgenden Abbildungen sind stetig.
\begin{figure}[htpb]
\begin{minipage}[t][7mm][b]{0.31\textwidth}
\begin{align*} \mathbb{C} &\to \mathbb{C} \\ z &\mapsto \bar{z}\end{align*}
\end{minipage}
\begin{minipage}[t][7mm][b]{0.31\textwidth}
\begin{align*} \mathbb{C}\times\mathbb{C} &\to \mathbb{C} \\ (z,w) &\mapsto z*w \end{align*}
\end{minipage}
\begin{minipage}[t][7mm][b]{0.31\textwidth}
\begin{align*} \mathbb{C}\times\mathbb{C}^{x} &\to \mathbb{C} \\ (z,w) &\mapsto z/w \end{align*}
\end{minipage}
\end{figure}
\item Die folgenden Funktionen sind auf [...] \todo{what goes there? p130 (week8sem1)} stetig:
\begin{figure}[htpb]
\begin{minipage}[t][1cm]{0.5\textwidth}
\begin{align*} \R^{d}&\to\R \\ x&\mapsto \norm{x} \end{align*}
\end{minipage}
\begin{minipage}[t][1cm]{0.5\textwidth}
\begin{align*} \mathbb{C}&\to\R \\ z&\mapsto \abs{z} \end{align*}
\end{minipage}
\end{figure}
%I hope this doesn't break too badly...
\item Die charakteristische Funktion von $\mathbb{Q}$: \\

Sei $f(x) =\mathcal{X}_{\mathbb{Q}} = \begin{cases} 1 &x\in\mathbb{Q} \\ 0 &x\in\R\backslash\mathbb{Q} \end{cases} $ \\

Sei $x\in\R\backslash\mathbb{Q}$ fest mit $(x_{k})\in\mathbb{Q}, x_{k}\to x$. Dann ist ${f(x_{k})=\mathcal{X}(x_{k})=1 \nrightarrow 0=\mathcal{X}(x)}$. \\
(Zu $x\in\R\backslash\mathbb{Q}$, sei $x_{k}$ die an der k-ten Nachkommastelle abgebrochene Dezimaldarstellung von $x$. Dann gilt $x_{k} \in \mathbb{Q}\; \forall k \in \N$ und $x_{k}\to x_{1}$.)
\item Sei $f: \begin{cases} x &x < 1 \\x &x>1 \end{cases}$ \\

\missingfigure{p131, week8 sem1} $f$ ist in $x=1$ nicht stetig, weil $f$ an der Stelle $x=1$ nicht definiert ist. In diesem Beispiel ist die Funktion $f$ nicht stetig, aber sie ist eigentlich eine ``gute'' \todo{Does that really say gute?}Funktion.

\todo{no ozlem number...}
\end{itemize}
\begin{definition}{(Struwe 4.1.3 (ii))}
$\Omega \subset \R^{d}, f:\Omega\to\R^{n}, x_{0}\in\R^{d}\backslash\mathbb{Q}$ so dass $\exists (x_{k})\in\Omega$ mit $\lim{x_{k}=x_{0}}$. \\

Dann ist $f$ an der Stelle $x_{0}$ \emph{stetig ergänzbar} falls $a=\lim{f(x_{k})}$ existiert. In diesem Fall setzen wir \[ f(x_{0} = a\]
Die durch $f(x_{0})=a$ ergänzte Funktion $f$ ist offenbar stetig an der Stelle $x_{0}$.
\end{definition}

\todo{offenbar $\to$ offensichtlich?}
\begin{itemize}
%Random ozlem enumeration ignored for consistency
\item Diese stückweise konstante Funktion ist stetig an jeder Stelle $x_{0} \neq 0$. Sie ist jedoch für $a\neq b$ an der Stelle $x_{0}=0$ nicht stetig ergänzbar. (Struwe Beispiel 4.1.3 (vii))
\begin{align*}f:\R^{x}&\to\R \\ f(x)&=\begin{cases}a &\text{falls } x<0 \\ b &\text{falls } x>0\end{cases} \end{align*}
\missingfigure{p132, week8 Sem1}
\item Sei $f:(a,b)\to\R$ monoton wachsend, d.h. $\forall x,y\in(a,b)$ mit $x\leq y$ folgt $f(x)\leq f(y)$. Sei ausserdem $x_{0}\in (a,b)$. Dann existieren die \emph{links- und rechtsseitigen Grenzwerte} 
\[ f(x_{0}^{+}) \defeq \lim_{\substack{x\to x_{0} \\ x > x_{0} \\ x \downarrow x_{0}}}f(x), \quad\quad f(x_{0}^{-})\defeq \lim_{\substack{x\to x_{0} \\ x < x_{0} \\ x \uparrow x_{0}}}f(x) \]
und $f$ ist stetig an der Stelle $x_{0}$ genau dann, wenn $f(x_{0}^{-})=f(x_{0}^{+})=f(x_{0})$.
\end{itemize}
\subsubsection*{Beweis}
Wir behaupten, dass für jede Folge $(y_{n})_{n\geq 1}$ mit $\{y_{n}:n\geq 1\} \subset (a,x_{0})$ und $\lim{y_{n}} = x_{0}$ die Folge $(f(y_{n}))_{n\geq 1}$ kovergent und der linksseitige Limes $l_{-}(x_{0})$ unabhängig von der Wahl der Folge ist.
\missingfigure{p133, week8 sem1} \\

\noindent Wir betrachten zuächst die ``spezielle'' Folge $x_{n}=(x_{0}-\frac{1}{n})_{n\geq r}$. Hier ist $r$ so gewählt, dass $x_{0}-\frac{1}{r}\geq a$. \\
Dann ist $(f(x_{0}-\frac{1}{n}))_{n\geq r}$ monoton wachsend ($x_{0}-\frac{1}{n+1} > x_{0}-\frac{1}{n}$ und $f$ monoton wachsend) und $(f(x_{0}-\frac{1}{n}))_{n\geq r}$ beschränkt ($f(a)<[...]<f(b)$\todo{missing in source material p134week8sem1}).
\[\text{Sei }l_{-}\defeq\lim_{n\to\infty}{f(x_{0}-\frac{1}{n})}\]
\noindent Wir möchten zeigen, dass für jede $(y_{n})\subset (a, x_{0})$ mit $\lim{y_{n}}=x_{0}$ $\lim{f(y_{n})}$ existiert und $\lim{f(y_{n})}=l_{-}$. \todo{$=l_{-}$ oder $=l$.?}

Da es für jedes $x<x_{0}$ ein $n$ gibt, mit $x\leq x_{0}-\frac{1}{n}$ folgt \[f(x) \leq f(x_{0}-\frac{1}{n} \leq l_{-}\]
Sei nun $(y_{n})_{n\geq 1}$ beliebig in \todo{unreadable p134 mid}$(?a?, x_{0})$ mit $\lim{y_{n}}=x_{0}$. Sei $\varepsilon > 0$, ($y_{n} < x_{0}$) und $n_{0}(\varepsilon)$ mit \[l_{-}-\varepsilon < f(x_{0}-\frac{1}{n})\leq l_{-} \quad \forall n>n_{0}(\varepsilon) \]
Insbesondere
\[l_{-} - \varepsilon < f(x_{0}-\frac{1}{n_{0}(\varepsilon)}) \leq l_{-} \]
Sei jetzt $n_{1}(\varepsilon)=n_{1}(n_{0}(\varepsilon))>0$ so dass \[ x_{n_{0}(\varepsilon)} = x_{0} - \frac{1}{n_{0}(\varepsilon)} < y_{n} < x_{0} = \lim_{n\to\infty}{x_{n}} \quad \forall n \geq n_{1}(\varepsilon)\]
\[ \left((y_{n})<(a,x_{0}), \lim{y_{n}}=x_{0}\right)\]
Da $f$ monoton ist, folgt \[ l_{-} - \varepsilon < f(x_{0}-\frac{1}{n_{0}(\varepsilon)}) \leq f(y_{n}) \leq l_{-} = \lim{f(x_{n})} \]
Insbesondere $\lim{f(y_{n})} = l_{-}$. \\

\noindent Der Beweis für $L_{+}$ verläuft ganz analog. \\

\noindent Nun zur Stetigkeit: Es gilt immer \[ l_{-}(x_{0})\leq f(x_{0}) \leq l_{+}(x_{0}) \]
Falls $l_{-}(x_{0}) < l_{+}(x_{0})$ sei $(t_{n})_{n \geq 1}$ wie folgt definiert:
\[ t_{n}=\begin{cases}x_{0}-\frac{1}{n} &n\text{ gerade} \\ x_{0}+\frac{1}{n} &n\text{ ungerade} \end{cases} \]
Dann gilt $\lim{t_{n}} = x_{0}$. Aber $f(t_{2n+1})-f(t_{n}) \geq l_{+}(x_{0})-l_{+}(x_{0}) >0$, woraus folgt dest \todo{dest? p 135 bottom} $(f(t_{n}))_{n\geq1}$ nicht konvergent. \\
Falls $l_{-}(x_{0})=l_{+}(x_{0})$ folgt die Stetigkeit sofort. 

\subsection*{Satz 4.3}
Sei $f:(a,b)\to\R$ monoton wachsend. Dann ist die Menge der Unstetigkeitspunkte von $f$ entweder endlich oder abzählbar.
\subsection*{Beweis}
Sei $U(f) = \{x\in(a,b):f \text{ ist nicht stetig an x}\}$. Dann ist $\forall x\in U(f), \quad l_{-}(x) < l_{+}(x)$ und wir wählen ein \todo{unreadable.. p136 mid}$g(x)\in ??n(l_{-}(x),l_{+}(x))$ .
Falls $x_{1}<x_{2}$ in $U(f)$ folgt $l_{+}(x_{1})<l_{-}(x_{2})$ und somit $g(x_{1})<g(x_{2})$. Damit ist $g:U(f)\to ??$ \todo{same unreadable character} injektiv. 

\noindent Stetigkeit verhält \todo{verträgt?} sich gut mit den üblichen Operationen auf Funktionen.

\subsection*{Satz 4.4}
Seien $f,g:\Omega \to \R^{n}$ und $x_{0}\in\Omega$. Falls $f$ und $g$ in $x_{0}$ stetig sind, so sind es auch $f+g$ und $\alpha f, \; \alpha \in \R$.

\subsection*{Korollar 4.5}
Falls $f,g$ auf $\Omega$ stetig sind, so sind es $f+g$ und $\alpha f$.
 
\begin{definition}{4.6}
\[ C\,(\,\Omega\, , \R\,) \] bezeichnet die Menge der stetigen Abbildungen $f: \Omega\to\R$. Nach Korollar 4.5 ist es ein Vektorraum.
\end{definition}

\subsection*{Satz 4.7}
Seien $f:\Omega\to\R^{n}, \Omega \subset \R^{d}$ und $g:\Gamma \to \R^{n}$ mit $f(\Omega)\subset\Gamma$ und ${x_{0}\in\Omega}, {y_{0}=f(x_{0})\subset\Gamma}$. Falls $f$ in $x_{0}$ und $g$ in $y_{0}$ stetig sind, folgt, dass $g\circ f:\Omega\to\R^{n}$ in $x_{0}$ stetig ist.

\subsection*{Beweis}
Sei $(t_{n})_{n\geq1}$ in $\Omega$ mit $\lim{t_{n}} = x_{0}$. Da $f$ stetig ist, $\lim{f(t_{n})} = f(x_{0}) = y_{0}$, und aus der Stetigkeit von $g$ folgt, dass \[ \lim_{n\to\infty}{g(f(t_{n}))} = g(y_{0}) = (g \circ f)(x_{0}) \]

\subsection*{Korollar 4.8}
Falls $f:\Omega \to \R^{d}, f(\Omega) \subset \Gamma$ und $g:\Gamma \to \R^{m}$, auf $\Omega$ bzw auf $\Gamma$ stetig sind, so folgt, dass $g\circ f:\Omega\to\R^{m}$ auf $\Omega$ stetig ist.  
\section{Stetige Funktionen}
In diesem Abschnitt behandeln wir die erste der fundamentalen Eigenschaften von stetigen Funktionen, nämlich das eine auf einem endlichen Intervall $[a,b]$ (Endpunkte eingeschlossen) stetige Funktion immer ein Max und Min besitzt. Dies veralgemeinern wir dann auf Abbildungen von $\Omega\subseteq\R^{d}$ nach $\R{n}$ wobei $\Omega$ eine spezielle Eigenschaft haben muss (Kompaktheit).

\subsection*{Satz 4.9}
Seien $-\infty < a \leq b < \infty$ und $f:[a,b] \to \R$ stetig. \\
\noindent Dann ist $f([a,b])$ in $\R$ beschränkt und es gibt $c_{-}, c_{+} \in [a,b]$ mit \begin{align*} f(c_{+}) &= \text{sup } \{ f(x):x\in[a,b]\} \\ f(c_{-}) &= \text{inf }\{ f(x):x\in[a,b]\} \end{align*} d.h. Supremum und Infimum werden angenommen.

