\chapter{Integration in $\mathbb{R}^{\lowercase{n}}$}
Im Eindimensionalen hatten wir mit dem Integral $$\int\limits_a^b f(x)dx$$ den Flächeninhalt unter dem Graphen von $f$ berechnet. Wir suchen nach eine Verallgemeinerung mit der \todo{can't understand word, page 207 middle} z.B. Volumen unter dem Graphen einer Funktion von zwei Variablen berechnen kann
\missingfigure{page 207, middle}
\underline{Erinnerung:} Bestimmtes Riemann-Integral einen Funktion $f(x)$ über einem Interval $\lbrack a,b\rbrack:$ 
$$I=\int\limits_a^b f(x)dx$$ Das Integral $I$ war als Grenzwert von Riemannscher Ober und Untersumme definiert (falls diese Grenzwert jeweils existieren und übereinstimmten).\\

Konstruktionsprinzip für Bereichsintegrale ist Analog. Aber die Definitionsbereich $D$ ist komplizierter. Wir betrachten zunächst den Fall zweier Variabler, $n=2$, und einen Definitionsbereich $D\subset\mathbb{R}^2$ der Form
$$D=\lbrack a_1,b_1\rbrack\times\lbrack a_2,b_2\rbrack\subset\mathbb{R}^2$$
d.h. $D$ ist ein kompakter Quader (Rechteck). Sei $f:D\rightarrow\mathbb{R}$ eine beschränkte Funktion.

\begin{definition}{9.1}
Mann nennt $Z=\left\{ \left( x_0,x_1,\dots,x_n\right), \left( y_0,y_1,\dots,y_m\right)\right\}$ eine Zerlegung des Quaders $D=\lbrack a_1,b_1\rbrack\times\lbrack a_2,b_2\rbrack$ falls gilt 
\begin{align*}
a_1&=x_0 < x_1\dots <x_n=b_1\\
a_2&=y_0 < y_1\dots <y_m=b_2
\end{align*}
\begin{enumerate}
\item WHERE IS NUMBER 1??
\item Die Feinheit einer Zerlegung $Z\in Z\left( D\right)$ ist \[\left\| Z \right\|: = \mathop {\max }\limits_{i,j} \left\{ {\left| {{x_{i + 1}} - {x_i}} \right|,\left| {{y_{j + 1}} - {y_j}} \right|} \right\}\]
\item Für eine vorgegebene Zerlegung $Z$, nennt man die Mengen \[{Q_{ij}}: = \left[ {{x_i},{x_{i + 1}}} \right] \times \left[ {{y_j},{y_{j + 1}}} \right]\] die Teilquader der Zerlegung $Z$. Das Volumen des Teilquaders $Q_{ij}$ ist \[\text{vol}\left( {{Q_{ij}}} \right): = \left( {{x_{i + 1}} - {x_i}} \right)\left( {{y_{j + 1}} - {y_j}} \right)\]
\item Für beliebige Punkte $\xi_{ij}\in Q_{ij}$ der Jeweiligen Teilquader nennt man \[{R_f}\left( Z \right): = \sum\limits_{i,j} {f\left( {{\xi _{ij}}} \right){\text{vol}}\left( {{Q_i}j} \right)} \] eine Riemannsche Summe zur Zerlegung $Z$ 
\item Analog zum Integral einer Variablen heissen für eine Zerlegung $Z$ 
\begin{align*}
{U_f}\left( Z \right): =&\sum\limits_{i,j} {\mathop {\inf }\limits_{\Romanbar{X}\in Q_{ij}} } f\left(\Romanbar{X}\right){\text{vol}}\left( {{Q_i}j} \right)\\
{O_f}\left( Z \right): =&\sum\limits_{i,j} {\mathop {\sup }\limits_{\Romanbar{X}\in Q_{ij}} } f\left(\Romanbar{X}\right){\text{vol}}\left( {{Q_i}j} \right)
\end{align*}
die Riemannsche Untersumme bzw. Riemannsche Obersumme con $f\left( x\right)$
\end{enumerate}
\end{definition}
\subsubsection*{Bemerkung 9.2}
\begin{enumerate}
\item Es gilt $$U_f\left( Z\right)\leq R_f\left( Z\right)\leq O_f\left( Z\right)$$ d.h. eine Riemannsche Summe zur Zerlegung $Z$ liegt stets zwischen der Unter und Obersumme dieser Zerlegung. 
\item Entsteht eine Zerlegung $Z_2$ aus der Zerlegung $Z_1$ durch Hinzunahme weiterer zwischenpunkte $x_i$ oder/und $y_j$ so gilt 
\begin{align*}
U_f\left( Z_2\right)&\geq U_f\left( Z_1\right)\text{ und }\\
O_f\left( Z_2\right)&\leq O_f\left( Z_1\right)
\end{align*}
Für zwei beliebige Zerlegungen $Z_1,Z_2$ gilt stets $$U_f\left( Z_1\right)\leq O_f\left( Z_2\right)$$
\missingfigure{Page 211 bottom}
\end{enumerate}
\begin{definition}{9.3}
Sei $f:D\rightarrow\mathbb{R}$ beschränkt 
\begin{enumerate}
\item Riemannsche Unterintegral bsz. Riemannsche Oberintegral der Funktion $f\left( x\right)$ über $D$ ist 
\begin{align*}
{U_f}&: = \sup \left\{ {{U_f}\left( z \right):z \in Z\left( D \right)} \right\}: = \int\limits_{\underline{D}} {f(x)d\mu } \\
{O_f}&: = \inf \left\{ {{O_f}\left( z \right):z \in Z\left( D \right)} \right\}: = \int\limits_D^ -  {f(x)d\mu } 
\end{align*}
\item Die Funktion $f(x)$ nennt man Riemann - integrierbar über $D$, falls Unter und Oberintegral übereinstimmen. Das Riemann Integral von $f(x)$ über $D$ ist \[\int\limits_D {f(x)d\mu  = \int\limits_D^ -  {f(x)d\mu  = } \int\limits_{\underline{D}} {f(x)d\mu } } \]
\end{enumerate}
\end{definition}

\subsubsection*{Bemerkung}
In höheren Dimensionen, $n >2$, ist die Vorgehensweise analog. Schreibweise: Für $n=2$, $n=3$
\[\int\limits_D {f\left( {x,y} \right)d\mu {\text{ bzw. }}} \int\limits_D {f\left( {x,y,z} \right)d\mu } \]
oder auch 
\[\int\limits_D {f\left( {x,y} \right)dxdy{\text{ bzw. }}} \int\limits_D {f\left( {x,y,z} \right)dxdydz} \]
oder 
\[\int\limits_D {fdxdy{\text{ bzw. }}} \int\limits_D {fdxdydz} \]

\subsubsection*{Satz 9.4 (Elementare Eigenschaften des Integrals)}
\begin{enumerate}
\item \underline{Linearität:} Seien $f,g: D\rightarrow\mathbb{R}$ beschränkt und R integrabel, $\beta,\alpha\in\mathbb{R}$. Dann sind $\alpha f$, $f+g$ R - Integrabel
$$\int\limits_D {\left( {\alpha f + \beta g} \right)d\mu  = \alpha \int\limits_D {fd\mu }  + \beta \int\limits_D {gd\mu } } $$
\item \underline{Monotonie:} Gilt $f(x)\leq g(x)$, $\forall x\in D$, so folgt \[\int\limits_D {fd\mu }  \le \int\limits_D {gd\mu } \]
\item\underline{Positivität:} Gilt für alle $x\in D$, $f(x)\geq 0$ (d.h. $f(x)$ ist nichtnegativ), so folgt \[\int\limits_D {fd\mu }  \ge 0\]
\item \underline{Abschätzung} \[\left| {\int\limits_D {f(x)d\mu } } \right| \le \mathop {\sup }\limits_{x \in D} \left| {f(x)} \right|{\text{vol}}\left( D \right)\]
\item Sind $D_1,d_2,D$ Quader, $D=D_1\cup D_2$ und $\text{vol}\left( D_1\cap D_2\right)=0$, so ist $f$ genau dann über $D$ integrierbar, falls $f$ über $D_1$ und über $D_2$ integrierbar ist und es gilt \[\int\limits_D {fd\mu }  = \int\limits_{{D_1}} {fd\mu }  + \int\limits_{{D_2}} {fd\mu } \](Gebietsadditivität)
\end{enumerate}

\section{Der Satz von Fubini}
\todo[inline]{According to the notes it should be 9.2, which one is right??}
Wie kann man das R - Integral konkret berechnen? Der Satz von Fubini hilft uns.

\subsubsection*{Satz 9.5 (Satz von Fubini)}
Sei $Q=\lbrack a,b\rbrack\times \lbrack c,d\rbrack\in\mathbb{R}^2$ und sei $f\in C\degree\left( Q\right)$. Dann gilt \[\int\limits_Q {fd\mu }  = \int\limits_a^b {\left( {\int\limits_c^d {f\left( {x,y} \right)dy} } \right)dx = \int\limits_c^d {\left( {\int\limits_a^b {f\left( {x,y} \right)dx} } \right)dy} } \]
d.h. das Integral von $f$ über $Q$ kann iterativ durch $1-$Dimensionale Integration bestimmt werden 

\subsubsection*{Beispiel 9.6}
\begin{enumerate}
\item Sei $f\left( x,y\right)=2x+2yx$, $Q=\lbrack 0,1\rbrack\times\lbrack -2,2\rbrack$
\begin{align*}
\int\limits_Q {fd\mu }  =&\int\limits_{ - 2}^2 {\left( {\int\limits_0^1 {\left( {2x + 2yx} \right)dx} } \right)dy} \\
=&\int\limits_{ - 2}^2 {\left( {\left. {{x^2} + y{x^2}} \right|_0^1} \right)dy} \\
=&\int\limits_{ - 2}^2 {\left( {1 + y} \right)dy}  = \left. {y + \frac{{{y^2}}}{2}} \right|_{ - 2}^2 = 4
\end{align*}
\underline{Oder:}
\begin{align*}
&\int\limits_0^1 {\left( {\int\limits_{ - 2}^2 {\left( {2x + 2yx} \right)dy} } \right)dx} \\
 =&\int\limits_0^1 {\left( {\left. {2xy + {y^2}x} \right|_{ - 2}^2} \right)dx} \\
 =&\int\limits_0^1 {\left[ {\left( {4x + 4x} \right) - \left( { - 4x + 4x} \right)} \right]dx} \\
 =&\int\limits_0^1 {8xdx}  = \left. {4{x^2}} \right|_0^1 = 4
\end{align*}
\item \begin{align*}
&\int\limits_0^1 {\int\limits_0^{2\pi } {\left( {{e^x}\sin y} \right)dy} dx} \\
=&\int\limits_0^1 {\left( {\left. { - {e^x}\cos y} \right|_0^{2\pi }} \right)dx} \\
=&\int\limits_0^1 0 dx = 0
\end{align*}
\underline{Oder:}
\begin{align*}
&\int\limits_0^{2\pi } {\left( {\int\limits_0^1 {{e^x}\sin ydx} } \right)dy} \\
=&\int\limits_0^{2\pi } {\left( {\left. {\sin y{e^x}} \right|_0^1} \right)dy} \\
=&\int\limits_0^{2\pi } {\left( {e - 1} \right)\sin ydy} \\
=&\left. { - \left( {e - 1} \right)\cos y} \right|_0^{2\pi } = 0
\end{align*}
\end{enumerate}

\subsection*{Geometrische Dehnung}\todo{Not sure about the text size...}
\missingfigure{Page 218 top}
In der Skizze ergibt sch als Volumen der markierten Schicht bei festem $x$ und sehr kleinen Dicke $dx$ näherungswege das Volumen \[\left( {\int\limits_c^d {f\left( {x,y} \right)dy} } \right)dx\] Das Aufaddieren sämtlicher Schichtvolumen entspricht gerader der Integration über die Variable $x$, d.h. für das gesuchte Volumen gilt \[V = \int\limits_a^b {\left( {\int\limits_c^d {f\left( {x,y} \right)dy} } \right)dx} \]
Bis jetzt können wir nur Integrale über achsenparallel rechteckige bzw. quaderförmige Bereiche berechnen.\\

Das reicht für viele praktische Aufgaben nicht aus. Meist ist der Integrationsbereich $D$ krummling oder zumindest anders begrenzt
\missingfigure{Page 219 middle}
Die meisten praktischen Aufgaben lassen sich auf die Integration über sogenannte Normalbereiche zurückführen.

\begin{definition}{9.10}
\begin{enumerate}
\item Eine Teilmenge $D\subset\mathbb{R}^2$ heisst ein Normalbereich bezüglich der $x-$achse bzw. bezüglich der $y-$Achse falls es stetige Funktionen $g,h$ bzw. $\overline{g},\overline{h}$ gibt mit
$$D=\left\{ \left( x,y\right)\mid a\leq x\leq b,\text{ und }g(x)\leq y\leq h(x)\right\}$$ 
bzw.
$$D=\left\{ \left( x,y\right)\mid \overline{a}\leq x\leq \overline{b},\text{ und }\overline{g}(x)\leq y\leq \overline{h}(x)\right\}$$
\end{enumerate}
\end{definition}

\subsubsection*{Beispiel}
Kreise und Rechtecke sind Normalbereiche bzg. beider Achsen 
\missingfigure{Page 220, top}
Über Normalbereiche lässt sich sehr bequem integrieren 
\missingfigure{Page 220 bottom}
Die markierte Scheibe bei $y=$const. mit kleiner Dicke $dx$ besitzt näherungsweise das Volumen
\[V(x) = \left( {\int\limits_{g(x)}^{f(x)} {f\left( {x,y} \right)dy} } \right)dx\]
Nun braucht man $V(x)$ nur noch über $\lbrack a,b\rbrack$ zu integrieren\[V = \int\limits_a^b {\left( {\int\limits_{g(x)}^{h(x)} {f\left( {x,y} \right)dy} } \right)dx} \]

\subsubsection*{Satz 9.11}
\begin{enumerate}
\item Ist $f(x)$ stetig auf einem Normalbereich $$D=\left\{ \left( x,y\right)\in\mathbb{R}^2\mid a\leq x\leq b,\text{ und }g(x)\leq y\leq h(x)\right\}$$ 
so gilt 
\[\int\limits_D {f(x)d\mu  = \int\limits_a^b {\left( {\int\limits_{g(x)}^{h(x)} {f\left( {x,y} \right)dy} } \right)dx} } \]
\item bzw. Falls $$D=\left\{ \left( x,y\right)\in\mathbb{R}^2\mid \overline{a}\leq x\leq \overline{b},\text{ und }\overline{g}(x)\leq y\leq \overline{h}(x)\right\}$$
so gilt 
\[\int\limits_D {f(x)d\mu  = \int\limits_{\overline{a}}^{\overline{b}} {\left( {\int\limits_{\overline{g}(x)}^{\overline{h}(x)} {f\left( {x,y} \right)dx} } \right)dy} } \]
\end{enumerate}
\subsubsection*{Beispiel 9.12}
\begin{enumerate}
\item Sei $f\left( x,y\right)=x-y$
\missingfigure{Page 222, top}
\begin{align*}
\int\limits_D fd\mu =&\int\limits_{x=0}^{x=1}\int\limits_{y=0}^{y=\sqrt{1-x^2}}\left( x-y\right) dy dx\\
 =&\int\limits_0^1 {\left( {\left. {xy - \frac{{{y^2}}}{2}} \right|_0^{\sqrt {1 - {x^2}} }} \right)} dx\\
=&\int\limits_0^1 {\left( {x\sqrt {1 - {x^2}}  - \frac{{1 - {x^2}}}{2}} \right)dx}\\
=&\int\limits_0^1 {x\sqrt {1 - {x^2}} } dx - \frac{1}{2}\int\limits_0^1 {1 - {x^2}dx}\\
 =&\frac{1}{2} - \frac{2}{3} = \frac{1}{3}
\end{align*}
\begin{align*}
&\int\limits_0^1 {x\sqrt {1 - {x^2}} dx} \begin{array}{*{20}{c}}
{u = 1 - {x^2}}\\
{du =  - 2xdx}
\end{array}\\
&=  - \frac{1}{2}\int\limits_0^1 {\sqrt u du}  = \left. { - \frac{1}{2} \cdot \frac{2}{3}{u^{\frac{3}{2}}}} \right|_0^1 = \frac{1}{3}\\
&\int\limits_D {fd\mu  = \frac{1}{3} - \frac{1}{3} = 0} 
\end{align*}
\item Sei $D$ \todo{missing content?? page 223 top} die durch die Gerade $g(x)=x+2$ und die Parabel $b(x)=4-x^2$ begrenzte Gebiet
\missingfigure{Page 223, top}
Schnittpunkte:
\begin{align*}
&x+2=4-x^2\\
&x^2+x-2=0\\
&\left( x-y\right)\left( x+2\right)
\end{align*}
Zu Berechnung des Doppelintegrals zerlegen wir das Gebiet in Streifen parallel zur $y-$Achse. Für festes $x$ variert $y$ von $g(x)=x+2$ bis $h(x)=4-x^2$
\begin{align*}
\int\limits_D xd\mu  =&\int\limits_{ - 2}^1 {\left( {\int\limits_{x + 2}^{4 - {x^2}} {xdy} } \right)dx} \\
 =&\int\limits_{ - 1}^2 {x\left( {4 - {x^2} - x + 2} \right)dx}\\
 =&\int\limits_{ - 1}^2 {\left( {2x - {x^3} - {x^2}} \right)dx} \\
 =&\left. {{x^2} - \frac{{{x^4}}}{4} - \frac{{{x^3}}}{3}} \right|_{ - 1}^2\\
 =&\left( {4 - 4 - \frac{8}{3}} \right) - \left( {1 - \frac{1}{4} + \frac{1}{9}} \right) =  - \frac{{127}}{{36}}
\end{align*}
\item Sei $D:$
\missingfigure{Page 224 top}
\[\int\limits_D {fd\mu } \mathop  = \limits_{\begin{array}{*{20}{c}}
 \downarrow \\
 * 
\end{array}} \int\limits_{ - 1}^1 {\left( {\int\limits_{x = {y^2}}^1 {fdx} } \right)} dy\]
$$\text{(\textasteriskcentered $=$ Zerlegung des Gebietes in Streifen parallel zur $x-$achse)}$$
oder mit Zerlegung in streifen parallel zur $y-$Achse
\[\int\limits_D {fd\mu }  = \int\limits_{x = 0}^1 {\left( {\int\limits_{y =  - \sqrt x }^{y = \sqrt x } {f\left( {x,y} \right)dy} } \right)} dx\]
Manchmal muss man $D$ zerlegen.
\item Bestimme $\int\limits_D {xdxdy} $ wobei $D$ von $y^2=4x$ und $y=2x-4$ begrenzt wird.
\missingfigure{Page 225, middle}
Schnittpunkte $P_1,P_2$:
\begin{align*}
&4x=y^2=\left( 2x-4\right)^2\\
&\Rightarrow\left( 2x-4\right)^2=4x\dots\\
&\Rightarrow x=1 \text{ und } x=4\\
P_1&=\left(1,-2\right)\hspace{5mm}P_2=\left(4,4\right)
\end{align*}
Zerlegung des Gebiets in Streifen parallel zur $y-$Achse
\[\int\limits_D {xd\mu  = \int\limits_0^1 {\left( {\int\limits_{ - 2\sqrt x }^{2\sqrt x } {xdy} } \right)dx} }  + \int\limits_1^4 {\left( {\int\limits_{y = 2x - 4}^{2\sqrt x } {xdy} } \right)} dx =  \ldots  = 14.4\]
Wenn wir Aussen nach $y$ integrieren, brauchen wir keine Unterteilung
\begin{align*}
\int\limits_D xd\mu  =&\int\limits_{y =  - 2}^{y = 4} {\left( {\int\limits_{y = \frac{{{y^2}}}{4}}^{\frac{y}{2} + 2} {xdx} }\right)dy} \\
 =&\int\limits_{ - 2}^4 {\left( {\left. {\frac{{{x^2}}}{2}} \right|_{\frac{{{y^2}}}{4}}^{\frac{y}{2} + 2}} \right)dy}\\
=&\frac{1}{2}\int\limits_{ - 2}^4 {\left( {{{\left( {\frac{y}{2} + 2} \right)}^2} - \frac{{{y^4}}}{{16}}} \right)dy}
\end{align*}
\end{enumerate}

\subsubsection*{Bemerkung 9.13}
\begin{enumerate}
\item Das Integral \[A = \int\limits_D {1d\mu } \] ergibt die Fläche von $D$. Für einen Normal bereich bzg. der $x-$Achse erhalten wir daraus die bekannte Formel
\[A = \int\limits_a^b {\int\limits_{g(x)}^{h(x)} {1dydx} }  = \int\limits_a^b {\left( {h(x) - g(x)} \right)dx} \]
\missingfigure{Page 226, bottom}
\item Interpretiert man $\rho(x,y)$ als ortabhängige Flächendichte, so erhält man mit \[m = \int\limits_D {\rho \left( {x,y} \right)d\mu } \] die Masse von $D$
\end{enumerate}

\begin{definition}{9.14}
Eine Teilmenge $D\subset\mathbb{R}^3$ heisst Normalbereich, falls es eine Darstellung 
\[D = \left\{ {\left. {\left( {x,y,z} \right) \in {R^3}} \right|a \le x \le b;g(x) < y < h(x);\varphi \left( {x,y} \right) \le z \le \psi \left( {x,y} \right)} \right\}\] gibt.\\

(Vertauscht man die Rollen von $x,y$ und $z$ so entstehen weitere Mengen, die auch Normalbereiche genannt werden.)
\end{definition}

\subsubsection*{Satz 9.15}
Sei $D\subset\mathbb{R}^3$ ein Normalbereich mit Darstellung wie oben, und $f:D\rightarrow\mathbb{R}$ stetig. Dann gilt \[\int\limits_D {fd\mu }  = \int\limits_a^b {\int\limits_{g(x)}^{h(x)} {\int\limits_{\varphi \left( {x,y} \right)}^{\psi \left( {x,y} \right)} {f\left( {x,y,z} \right)dzdydx} } } \]
\missingfigure{Page 228, middle}
$z=\varphi \left( {x,y} \right)$ und $z=\psi \left( {x,y} \right)$ stellen die ``Grund'' und Deckelfläche von $D$ dar. \\

Der Normalbereich $A$ ist die Senkrechte Projektion von $D$ in die $x-y$ Ebene. Dessen ``Grund'' und ``Deckelkurve'' sind durch $y=g(x)$ und $y=h(x)$ gegeben.

