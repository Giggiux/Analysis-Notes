\chapter{Folgen und Reihen (Der Limes Begriff)}
\section{Folgen, allgemeines}
\begin{definition}{3.1}
Eine Folge reeler zahlen ist eine Abbildung $a:\mathbb{N}\backslash\{0\}\to\mathbb{R}$ wobei wir das Bild con $n\geq 1$ mit $a_n$ (statt $a(n)$) bezeichen.\\

Eine Folge wird dann meistens mit $(a_n)_{n\geq 1}$, daher mit der geordneten Bildmenge bezeichnet.
\end{definition}

\noindent  Folgen können auf verschiedene Arten gegeben sein.
\subsubsection*{Beispiel 3.2}
\begin{enumerate}
\item $a_n=\frac{1}{n}$, $n\geq 1$
\item $a_1=0.9$, $a_2=0.99$, \dots, ${a_n} = 0.\underbrace {99 \ldots 9}_{n-\text{mal}}$
\item $a_n=\left( 1+\frac{1}{n}\right)^n$, $n\geq 1$
\item (Rekursiv) Sei $d>0$ eine reelle Zahl $a_1,\dots, a_{n+1}:=\frac{1}{2}\left( a_n + \frac{d}{a_n}\right), n\geq 1$\\
z.B. $d=2,a_1=1,a_2=\frac{3}{2},a_3=\frac{17}{12},a_4=\dots$
\item Fibonacci Zahlen. $a_1=1,a_2=2, a_{n+1}=a_n+a_{n-1}\hspace{5mm}\forall n\geq 2$
\end{enumerate}

\begin{definition}{3.3}
Eine Folge $(a_n)_{n\geq 1}$ heisst beschränkt falls die Teilmenge $\{a_n:n\geq 1\}\subseteq\mathbb{R}$ beschränkt ist. d.h. Es gibt $c\in\mathbb{R} (c\geq 0)$ so dass $\left| a_n\right| \leq c, \forall n\geq 1$ 
\end{definition}

\section{Grenzwert oder Limes eine Folge. Ein zentraler Begriff}
\begin{definition}{3.4}
Eine Folge $(a_n)\geq 1$ konvergiert gegen $a$ wann für jedes $\varepsilon>0$ ein Index $N(\varepsilon)\geq 1$ gilt so dass \[  \left| a_n-a\right| <\varepsilon, \forall n>N(\varepsilon)\] 
\end{definition}

\begin{definition}{3.4 (Version 2)}Eine Folge $(a_n)_{n\geq 1}$ konvergiert gegen $a\in\mathbb{R}$ falls für jedes $\varepsilon>0$ die Menge der Indizen $n\geq 1$ für welcher $a_n\not\in (a-\varepsilon,a+\varepsilon)$ endlich ist.\\

\centerline{$\left( \forall\varepsilon>0, \#\{ n\in\mathbb{N}\mid a_n\not\in (a-\varepsilon,a+\varepsilon)\} <\infty\right)$}
\end{definition}

\subsection*{Equivalenz beider Definitionen}\todo{Is this supposed to be a title?}
\begin{enumerate}

\item[(2)] $\Rightarrow(1)$ \\Sei für  $\varepsilon>0$ \[M(\varepsilon):=\{ n\in\mathbb{N}\mid a_n\not\in (a-\varepsilon,a+\varepsilon)\}=\{n\in\mathbb{N}\mid \left| a_n-a \right|\geq \varepsilon\}\]
Da $M(\varepsilon)$ endlich ist, ist es nach oben beschränkt. Es  gibt also $N(\varepsilon)\in\mathbb{N}$ so dass $\forall n\in M(\varepsilon)$, $n\leq N(\varepsilon)-1$. Insbesondere gilt $\forall n\geq N(\varepsilon)$, $n\not\in M(\varepsilon)$ und daher $\left| a_n-a\right|<\varepsilon$.
\item[(1)] $\Rightarrow(2)$ \[M(\varepsilon)=\{ n:\left| a_n-a\right| \geq \varepsilon\} \subset \left[ 0,N(\varepsilon)-1 \right]\] Also endlich.
\end{enumerate}

\noindent Falls die Eigenschaften in Definition 3.4 zutrifft, dann schreibt man\[a = \mathop {\lim }\limits_{n \to \infty } {a_n}{\text{ oder }}{a_n}\mathop  \to \limits_{n \to \infty } a\] Die Zahl $a$ nennt sich Grenzwert oder Limes der Folge $(a_n)_{n\geq 1}$. Eine Folge heisst konvergent falls sie einen Limes besitzt, andernfalls heisst sie divergent.

\subsubsection*{Bemerkung 3.5}
\begin{enumerate}
\item Falls $(a_n)_{n\geq 1}$ konvergent ist der Limes eindeutig bestimmt
\subsubsection*{Beweis}
Seien $a$ und $b$ Grenzwerte von $(a_n)_{n\geq 1}$. Sei $\varepsilon = \left| \frac{b-a}{3}\right|>0$, dann gibt es $N_1,N_2$ so dass \[\left| a_n-a\right| <\varepsilon\hspace{10mm}\forall n>N_1\]\[\left| a_n-b\right| <\varepsilon\hspace{10mm}\forall n>N_2\]
Also$ \forall n\geq \max\{ N_1,N_2\}$ \[(a-b)\cong\left| (a-a_n)+(a_n-b)\right| < 2\varepsilon=\frac{2}{3}\left|b-a\right|\]
\subsection*{Binomischen Lehrsatz}
Für beliebige Zahlen $a,b$ und $n\in\mathbb{N}$ ist \[{\left( {a + b} \right)^n} = \sum\limits_{k = 0}^n {\left( {\begin{array}{*{20}{c}}
n\\
k
\end{array}} \right){a^{n - k}}{b^k}} \]
\item Falls $(a_n)_{n\geq 1}$ konvergent ist, $\{a_n:n\geq 1\}$ beschränkt: Sei $\varepsilon=1$, $\lim a_n=a$ und $N_0$ mit \[\left| a_n-a\right| \leq 1\hspace{10mm} \forall n>N_0\] Dann ist $\forall n$ $\left| a_n\right| \geq \max\{\left| a\right| +1,\left| a_j\right|, 1\leq j\leq N_0  \}$
\end{enumerate}
\subsubsection*{Beispiel 3.6}
\begin{enumerate}
\item Sei $a_n=\frac{1}{n}, n\geq 1$. Dann gilt $\lim a_n=0$ 
\begin{itemize}
\item Sei $\varepsilon>0$. Dann $\frac{1}{\varepsilon}>0$. Sei $n_0\in\mathbb{N}$, $n_0\geq 1$ mit $n_0>\frac{1}{\varepsilon}$ (Archimedische Eigenschaft, Satz 2.13)\\ 
\end{itemize}
Dann gilt für alle $n\geq n_0$, $\frac{1}{\varepsilon}<n_0\leq n \Rightarrow \left| \frac{1}{n}-0\right| = \frac{1}{n}<\varepsilon, \forall n\geq n_0$
\item Sei $0<q<1$ und $a_n:=q^n$, $n 1$\todo{Cannot read, page 54 top}. Dann gilt $\lim a_n=0$ ($a_n$ konvergiert gegen 0)
\subsubsection*{Beweis}
Zu beweisen \[\forall \varepsilon > 0, \exists N_0=N_0(\varepsilon)\in N\]\todo{Should it be $\in\mathbb{R}$??}\[\forall n\geq N_0:q^n <\varepsilon\]
Die Idee ist zu zeigen dass $\frac{1}{q^n}$ sehr Gross wird und deswegen $q^n$ sehr klein wird. Setzen wir $\frac{1}{q}=1+\delta$ mit $\delta>0$ $\left( 1<1\Rightarrow \frac{1}{q}>1\right)$%CHECK FROM HERE
 $$\frac{1}{q^n}=\left( \frac{1}{q}\right)^n=\left( 1+\delta\right)^n=1+n\delta +\left( {\begin{array}{*{20}{c}}
n\\
2
\end{array}} \right) \delta^2+\dots+\delta^n\geq 1+n\delta>n\delta, \forall n\in\mathbb{N}$$
also \[0<q^n<\frac{1}{n\delta}, \forall n\in\mathbb{N}\]
Sei jetzt $\varepsilon >0$, wähle $N_0=N_0(\varepsilon)$ mit $\frac{1}{\varepsilon\delta}<N_0\Rightarrow \frac{1}{N_0\delta}<\varepsilon$
\[\forall n>N_0\hspace{5mm}0<q^n\leq\frac{1}{n\delta}<\frac{1}{N_0\delta}<\varepsilon\]
\item $\sqrt[n]{n}$, $\lim a_n=1$. Klar: $n\geq 1$ also $\sqrt[n]{n}\geq 1$\\
Gegeben ein $\varepsilon>0$, wollen wir $n$ so gross wählen, dass \[\sqrt[n]{n}-1 <\varepsilon\] das heisst, $n<\left( 1+\varepsilon \right)^n$. Wir entwickeln $$\left( 1+\varepsilon\right)^n=1+n\varepsilon+\left( {\begin{array}{*{20}{c}}
n\\
2
\end{array}} \right) \varepsilon^2 + \dots +\varepsilon^n$$\todo{can't read last element of the expansion}
$\varepsilon$ ist klein aber fixiert.\\

Für $n$ sehr gross wird $1+n\varepsilon$ nie grösser als $n$ sein. Wir versuchen unsere Glück mit \[\left( {\begin{array}{*{20}{c}}
n\\
2
\end{array}} \right){\varepsilon ^2}{\text{ term}}\] 
\[\left( {\begin{array}{*{20}{c}}
n\\
2
\end{array}} \right){\varepsilon ^2} = \frac{{n(n - 1)}}{2}\varepsilon \] Wir benutzen also $( 1+\varepsilon )^n\geq \frac{n(n-1)}{2}\varepsilon^2$. Wir wollen $n$ so wählen dass \[\frac{{n(n - 1)}}{2}{\varepsilon ^2} > n\] d.h. $n-1>\frac{2}{\varepsilon^2}$ oder $n>1+\frac{2}{\varepsilon^2}$\\

Setzen wir $N_0:=\left( 1+\frac{2}{\varepsilon^2}\right)+1$. Dann gilt für $\forall n>N_0$ \[(1+\varepsilon)^n > n \geq 1\]
\[\Rightarrow 1\leq \sqrt[n]{n}\leq 1+\varepsilon\]
\[\Rightarrow -\varepsilon<0\leq\sqrt[n]{n}-1\leq\varepsilon\Rightarrow \left| \sqrt[n]{n}-1\right| <\varepsilon, \forall n>N_0\]
\item Nicht alle Folgen sind konvergent. Es gibt zwei verschiedene Verhältnissen einer divergenten Folge \[a_n=(-1)^n \Rightarrow \{1,-1,\dots \}\text{ beschränkt aber nicht konvergent}\]
\item $a_n=n$ unbeschränkt, divergent.
\end{enumerate}

\subsubsection*{Beispiel 3.7}
Seien $p\in\mathbb{N}$, $0<q<1$. Dann gilt $\lim\limits_{n\to\infty}n^p q^n=0$. Dass heisst Exponentialfunktionen wächst schneller als jede Potenz (Wann $x$ genügend Gross ist, $a^x>x^b$)

\subsubsection*{Beweis}
Der Trick ist folgender \[{n^p}{q^n} = {\left( {{n^{\frac{p}{n}}} \cdot q} \right)^n} = {\left( {{{\left( {\sqrt[n]{n}} \right)}^p}{{\left( {{q^{\frac{1}{p}}}} \right)}^p}} \right)^n}\] Wir werden Beispiel 3.6 (2.),(3.) benutzen. \\(d.h. $\lim\sqrt[n]{n}=1 \hspace{10mm}\lim r^n=0, 0<r<1$) \\

Da $\lim\sqrt[n]{n}=1$, $\forall\eta >0$, $\exists N_0(\eta)$ \[\sqrt[n]{n}<1+\eta, n>N_0(\eta)\]
Wir wählen $\eta >0$ so dass ${q^{\frac{1}{p}}} = \frac{1}{{{{\left( {1 + \eta } \right)}^2}}}$. Dann
\[\sqrt[n]{n}\cdot q^{\frac{1}{p}}\leq\frac{\left( 1+\eta\right)}{\left( 1+\eta\right)^2}=\frac{1}{1+\eta}\hspace{10mm} \forall n>N_0\left(\eta\right)\]
Wobei \[\forall n>N_0\left(\eta\right)\]\[a_n=\left( \sqrt[n]{n}q^{\frac{1}{p}}\right)^{pn}<r^n\] mit \[r:=\left( \frac{1}{1+\eta}\right)^p, r<1\]
Sei jetzt $\varepsilon>0$. Da $\lim r^n=0$, $\exists N_1=N_1(\varepsilon)$, $\forall n>N_1(\varepsilon)$, $r^n<\varepsilon$\\

\noindent Für $n>\max\{N_0\left(\eta\right), N_1(\varepsilon)\}$, $a_n<r^n<\varepsilon \Rightarrow \lim a_n=\lim n^pq^n=0$ 

\section{Konvergenzkriterien}
Mit konvergenten Folgen kann man \todo{Can't read, page 59 top} wie folgender Satz zeigt.
\subsubsection*{Satz 3.8}
Seien
