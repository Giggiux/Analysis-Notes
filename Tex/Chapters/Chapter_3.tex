\chapter{Folgen und Reihen (Der Limes Begriff)}
\section{Folgen, allgemeines}
\begin{definition}{3.1}
Eine Folge reeler zahlen ist eine Abbildung $a:\N\backslash\{0\}\to\R$ wobei wir das Bild con $n\geq 1$ mit $a_n$ (statt $a(n)$) bezeichen.\\

Eine Folge wird dann meistens mit $(a_n)_{n\geq 1}$, daher mit der geordneten Bildmenge bezeichnet.
\end{definition}

\noindent  Folgen können auf verschiedene Arten gegeben sein.
\subsubsection*{Beispiel 3.2}
\begin{enumerate}
\item $a_n=\frac{1}{n}$, $n\geq 1$
\item $a_1=0.9$, $a_2=0.99$, \dots, ${a_n} = 0.\underbrace {99 \ldots 9}_{n-\text{mal}}$
\item $a_n=\left( 1+\frac{1}{n}\right)^n$, $n\geq 1$
\item (Rekursiv) Sei $d>0$ eine reelle Zahl $a_1,\dots, a_{n+1}:=\frac{1}{2}\left( a_n + \frac{d}{a_n}\right), n\geq 1$\\
z.B. $d=2,a_1=1,a_2=\frac{3}{2},a_3=\frac{17}{12},a_4=\dots$
\item Fibonacci Zahlen. $a_1=1,a_2=2, a_{n+1}=a_n+a_{n-1}\hspace{5mm}\forall n\geq 2$
\end{enumerate}

\begin{definition}{3.3}
Eine Folge $(a_n)_{n\geq 1}$ heisst beschränkt falls die Teilmenge $\{a_n:n\geq 1\}\subseteq\R$ beschränkt ist. d.h. Es gibt $c\in\R (c\geq 0)$ so dass $\left| a_n\right| \leq c, \forall n\geq 1$ 
\end{definition}

\section{Grenzwert oder Limes eine Folge. Ein zentraler Begriff}
\begin{definition}{3.4}
Eine Folge $(a_n)\geq 1$ konvergiert gegen $a$ wann für jedes $\varepsilon>0$ ein Index $N(\varepsilon)\geq 1$ gilt so dass \[  \left| a_n-a\right| <\varepsilon, \forall n>N(\varepsilon)\] 
\end{definition}

\begin{definition}{3.4 (Version 2)}Eine Folge $(a_n)_{n\geq 1}$ konvergiert gegen $a\in\R$ falls für jedes $\varepsilon>0$ die Menge der Indizen $n\geq 1$ für welcher $a_n\not\in (a-\varepsilon,a+\varepsilon)$ endlich ist.\\

\centerline{$\left( \forall\varepsilon>0, \#\{ n\in\N\mid a_n\not\in (a-\varepsilon,a+\varepsilon)\} <\infty\right)$}
\end{definition}

\subsection*{Equivalenz beider Definitionen}\todo{Is this supposed to be a title?}
\begin{enumerate}

\item[(2)] $\Rightarrow(1)$ \\Sei für  $\varepsilon>0$ \[M(\varepsilon):=\{ n\in\N\mid a_n\not\in (a-\varepsilon,a+\varepsilon)\}=\{n\in\N\mid \left| a_n-a \right|\geq \varepsilon\}\]
Da $M(\varepsilon)$ endlich ist, ist es nach oben beschränkt. Es  gibt also $N(\varepsilon)\in\N$ so dass $\forall n\in M(\varepsilon)$, $n\leq N(\varepsilon)-1$. Insbesondere gilt $\forall n\geq N(\varepsilon)$, $n\not\in M(\varepsilon)$ und daher $\left| a_n-a\right|<\varepsilon$.
\item[(1)] $\Rightarrow(2)$ \[M(\varepsilon)=\{ n:\left| a_n-a\right| \geq \varepsilon\} \subset \left[ 0,N(\varepsilon)-1 \right]\] Also endlich.
\end{enumerate}

\noindent Falls die Eigenschaften in Definition 3.4 zutrifft, dann schreibt man\[a = \mathop {\lim }\limits_{n \to \infty } {a_n}{\text{ oder }}{a_n}\mathop  \to \limits_{n \to \infty } a\] Die Zahl $a$ nennt sich Grenzwert oder Limes der Folge $(a_n)_{n\geq 1}$. Eine Folge heisst konvergent falls sie einen Limes besitzt, andernfalls heisst sie divergent.

\subsubsection*{Bemerkung 3.5}
\begin{enumerate}
\item Falls $(a_n)_{n\geq 1}$ konvergent ist der Limes eindeutig bestimmt
\subsubsection*{Beweis}
Seien $a$ und $b$ Grenzwerte von $(a_n)_{n\geq 1}$. Sei $\varepsilon = \left| \frac{b-a}{3}\right|>0$, dann gibt es $N_1,N_2$ so dass \[\left| a_n-a\right| <\varepsilon\hspace{10mm}\forall n>N_1\]\[\left| a_n-b\right| <\varepsilon\hspace{10mm}\forall n>N_2\]
Also$ \forall n\geq \max\{ N_1,N_2\}$ \[(a-b)\cong\left| (a-a_n)+(a_n-b)\right| < 2\varepsilon=\frac{2}{3}\left|b-a\right|\]
\subsection*{Binomischen Lehrsatz}
Für beliebige Zahlen $a,b$ und $n\in\N$ ist \[{\left( {a + b} \right)^n} = \sum\limits_{k = 0}^n {\left( {\begin{array}{*{20}{c}}
n\\
k
\end{array}} \right){a^{n - k}}{b^k}} \]
\item Falls $(a_n)_{n\geq 1}$ konvergent ist, $\{a_n:n\geq 1\}$ beschränkt: Sei $\varepsilon=1$, $\lim a_n=a$ und $N_0$ mit \[\left| a_n-a\right| \leq 1\hspace{10mm} \forall n>N_0\] Dann ist $\forall n$ $\left| a_n\right| \geq \max\{\left| a\right| +1,\left| a_j\right|, 1\leq j\leq N_0  \}$
\end{enumerate}
\subsubsection*{Beispiel 3.6}
\begin{enumerate}
\item Sei $a_n=\frac{1}{n}, n\geq 1$. Dann gilt $\lim a_n=0$ 
\begin{itemize}
\item Sei $\varepsilon>0$. Dann $\frac{1}{\varepsilon}>0$. Sei $n_0\in\N$, $n_0\geq 1$ mit $n_0>\frac{1}{\varepsilon}$ (Archimedische Eigenschaft, Satz 2.13)\\ 
\end{itemize}
Dann gilt für alle $n\geq n_0$, $\frac{1}{\varepsilon}<n_0\leq n \Rightarrow \left| \frac{1}{n}-0\right| = \frac{1}{n}<\varepsilon, \forall n\geq n_0$
\item Sei $0<q<1$ und $a_n:=q^n$, $n 1$\todo{Cannot read, page 54 top}. Dann gilt $\lim a_n=0$ ($a_n$ konvergiert gegen 0)
\subsubsection*{Beweis}
Zu beweisen \[\forall \varepsilon > 0, \exists N_0=N_0(\varepsilon)\in N\]\todo{Should it be $\in\R$??}\[\forall n\geq N_0:q^n <\varepsilon\]
Die Idee ist zu zeigen dass $\frac{1}{q^n}$ sehr Gross wird und deswegen $q^n$ sehr klein wird. Setzen wir $\frac{1}{q}=1+\delta$ mit $\delta>0$ $\left( 1<1\Rightarrow \frac{1}{q}>1\right)$%CHECK FROM HERE
 $$\frac{1}{q^n}=\left( \frac{1}{q}\right)^n=\left( 1+\delta\right)^n=1+n\delta +\left( {\begin{array}{*{20}{c}}
n\\
2
\end{array}} \right) \delta^2+\dots+\delta^n\geq 1+n\delta>n\delta, \forall n\in\N$$
also \[0<q^n<\frac{1}{n\delta}, \forall n\in\N\]
Sei jetzt $\varepsilon >0$, wähle $N_0=N_0(\varepsilon)$ mit $\frac{1}{\varepsilon\delta}<N_0\Rightarrow \frac{1}{N_0\delta}<\varepsilon$
\[\forall n>N_0\hspace{5mm}0<q^n\leq\frac{1}{n\delta}<\frac{1}{N_0\delta}<\varepsilon\]
\item $\sqrt[n]{n}$, $\lim a_n=1$. Klar: $n\geq 1$ also $\sqrt[n]{n}\geq 1$\\
Gegeben ein $\varepsilon>0$, wollen wir $n$ so gross wählen, dass \[\sqrt[n]{n}-1 <\varepsilon\] das heisst, $n<\left( 1+\varepsilon \right)^n$. Wir entwickeln $$\left( 1+\varepsilon\right)^n=1+n\varepsilon+\left( {\begin{array}{*{20}{c}}
n\\
2
\end{array}} \right) \varepsilon^2 + \dots +\varepsilon^n$$\todo{can't read last element of the expansion}
$\varepsilon$ ist klein aber fixiert.\\

Für $n$ sehr gross wird $1+n\varepsilon$ nie grösser als $n$ sein. Wir versuchen unsere Glück mit \[\left( {\begin{array}{*{20}{c}}
n\\
2
\end{array}} \right){\varepsilon ^2}{\text{ term}}\] 
\[\left( {\begin{array}{*{20}{c}}
n\\
2
\end{array}} \right){\varepsilon ^2} = \frac{{n(n - 1)}}{2}\varepsilon \] Wir benutzen also $( 1+\varepsilon )^n\geq \frac{n(n-1)}{2}\varepsilon^2$. Wir wollen $n$ so wählen dass \[\frac{{n(n - 1)}}{2}{\varepsilon ^2} > n\] d.h. $n-1>\frac{2}{\varepsilon^2}$ oder $n>1+\frac{2}{\varepsilon^2}$\\

Setzen wir $N_0:=\left( 1+\frac{2}{\varepsilon^2}\right)+1$. Dann gilt für $\forall n>N_0$ \[(1+\varepsilon)^n > n \geq 1\]
\[\Rightarrow 1\leq \sqrt[n]{n}\leq 1+\varepsilon\]
\[\Rightarrow -\varepsilon<0\leq\sqrt[n]{n}-1\leq\varepsilon\Rightarrow \left| \sqrt[n]{n}-1\right| <\varepsilon, \forall n>N_0\]
\item Nicht alle Folgen sind konvergent. Es gibt zwei verschiedene Verhältnissen einer divergenten Folge \[a_n=(-1)^n \Rightarrow \{1,-1,\dots \}\text{ beschränkt aber nicht konvergent}\]
\item $a_n=n$ unbeschränkt, divergent.
\end{enumerate}

\subsubsection*{Beispiel 3.7}
Seien $p\in\N$, $0<q<1$. Dann gilt $\lim\limits_{n\to\infty}n^p q^n=0$. Dass heisst Exponentialfunktionen wächst schneller als jede Potenz (Wann $x$ genügend Gross ist, $a^x>x^b$)

\subsubsection*{Beweis}
Der Trick ist folgender \[{n^p}{q^n} = {\left( {{n^{\frac{p}{n}}} \cdot q} \right)^n} = {\left( {{{\left( {\sqrt[n]{n}} \right)}^p}{{\left( {{q^{\frac{1}{p}}}} \right)}^p}} \right)^n}\] Wir werden Beispiel 3.6 (2.),(3.) benutzen. \\(d.h. $\lim\sqrt[n]{n}=1 \hspace{10mm}\lim r^n=0, 0<r<1$) \\

Da $\lim\sqrt[n]{n}=1$, $\forall\eta >0$, $\exists N_0(\eta)$ \[\sqrt[n]{n}<1+\eta, n>N_0(\eta)\]
Wir wählen $\eta >0$ so dass ${q^{\frac{1}{p}}} = \frac{1}{{{{\left( {1 + \eta } \right)}^2}}}$. Dann
\[\sqrt[n]{n}\cdot q^{\frac{1}{p}}\leq\frac{\left( 1+\eta\right)}{\left( 1+\eta\right)^2}=\frac{1}{1+\eta}\hspace{10mm} \forall n>N_0\left(\eta\right)\]
Wobei \[\forall n>N_0\left(\eta\right)\]\[a_n=\left( \sqrt[n]{n}q^{\frac{1}{p}}\right)^{pn}<r^n\] mit \[r:=\left( \frac{1}{1+\eta}\right)^p, r<1\]
Sei jetzt $\varepsilon>0$. Da $\lim r^n=0$, $\exists N_1=N_1(\varepsilon)$, $\forall n>N_1(\varepsilon)$, $r^n<\varepsilon$\\

\noindent Für $n>\max\{N_0\left(\eta\right), N_1(\varepsilon)\}$, $a_n<r^n<\varepsilon \Rightarrow \lim a_n=\lim n^pq^n=0$ 

\section{Konvergenzkriterien}
Mit konvergenten Folgen kann man \todo{Can't read, page 59 top} wie folgender Satz zeigt.
\subsubsection*{Satz 3.8}
Seien $\left( a_n\right)_{n\geq 1}$ und $\left( b_n\right)_{n\geq 1}$ konvergente Folgen mit 
\[\lim a_n=a\text{, }\lim b_n=b\]
\begin{enumerate}[\hspace{2mm}i)]
\item Die folge $\left( a_n+b_n\right)_{n\geq 1}$ konvergiert und $\lim\left( a_n+b_n\right)=a+b$
\item Die folge $\left( a_n\cdot b_n\right)_{n\geq 1}$ konvergiert und $\lim\left( a_n\cdot b_n\right)=a\cdot b$
\item Falls $b\not=0$ und $b_n\not=0$ $\forall n\geq 1$ gilt $\lim\frac{a_n}{b_n}=\frac{a}{b}$ 
\item Falls $a_n\leq b_n$ folgt $a\leq b$
\end{enumerate}

\begin{beweis}{}
Sei $\varepsilon >0$, es gibt  $N_1(\varepsilon)$, $N_2(\varepsilon)$ so dass
\begin{align*}
\left| a_n-a\right|&<\varepsilon, \forall n>N_1(\varepsilon)\\
\left| b_n-b\right|&<\varepsilon, \forall n>N_2(\varepsilon)
\end{align*}
\begin{enumerate}[\hspace{2mm}i)]
\item Für $n\geq\max\left\{ N_1(\varepsilon),N_2(\varepsilon)\right\}$ gilt \[\left| {\left( {{a_n} + {b_n}} \right) - \left( {a + b} \right)} \right| \le \left| {{a_n} - a} \right| + \left| {{b_n} - b} \right|\]
Da dies für alle $\varepsilon>0<2\varepsilon$ gilt, folgt auch 
\[\forall n > \max \left\{ {{N_1}\left( {\frac{\varepsilon }{2}} \right),{N_2}\left( {\frac{\varepsilon }{2}} \right)} \right\}: = N(\varepsilon )\]
gilt \[\left| a_n+b_n-(a+b)\right| <\varepsilon\]
\item Sei $C$ eine Schränke für $\left\{ \left|b_n\right| : n\geq 1\right\}$ (Bemerkung 3.5: Falls $\left( a_n\right)_{n\geq 1}$ konvergent ist, $\left\{b_n : n\geq 1\right\}$ beschränkt). Für $N_1(\varepsilon)$, $N_2(\varepsilon)$ wie oben folgt $\forall n\geq\max\left\{ N_1(\varepsilon), N_2(\varepsilon)\right\}$

\begin{align*}
\left| {{a_n}{b_n} - ab} \right| =&\left| {{a_n}{b_n} - a{b_n} + a{b_n} - ab} \right|\\
 =&\left| {{b_n}\left( {{a_n} - a} \right) + a\left( {{b_n} - b} \right)} \right|\\
 \le&  \varepsilon \left| {{b_n}} \right| + \left| a \right|\varepsilon  \le \varepsilon \left( {C + \left| a \right|} \right)
\end{align*}

Also folgt \[\forall n \ge N(\varepsilon ): = \max \left( {{N_1}\left( {\frac{\varepsilon }{{C + \left| a \right|}}} \right),{N_2}\left( {\frac{\varepsilon }{{\left| C \right| + a}}} \right)} \right)\]dass $\left| a_nb_n-ab\right| < \varepsilon$
\item Wegen (ii) genügt es, dem Fall $a=a_n=1$, $\forall n\in\N$ zu betrachten 
\[\left| {{b_n}} \right| = \left| {{b_n} - b + b} \right| \ge \left| b \right| - \left| {{b_n} - b} \right| \ge \left| b \right| - \varepsilon \]
Sei $0 < \varepsilon < \frac{\left| b\right|}{2}$, dann gilt $\left| b_n\right| > \frac{\left| b\right|}{2}$. Es folgt 

\[\left| {\frac{1}{{{b_n}}} - \frac{1}{b}} \right| = \left| {\frac{{{b_n} - b}}{{{b_n}b}}} \right| < \frac{2}{{{{\left| b \right|}^2}}}\left| {{b_n} - b} \right| \le \frac{2}{{{{\left| b \right|}^2}}}\varepsilon \hspace{10mm}\forall n > {n_0}(\varepsilon )\]
Also folgt $\forall n > N(\varepsilon):=n_0\left( \frac{\varepsilon\left| b\right|^2}{2}\right)$ dass $\left| \frac{1}{b_n}-\frac{1}{b}\right| < \varepsilon$ 
\item (Indirekter Beweis) Falls $a>b$, $a-b>0$. Sei 
\begin{align*}
\varepsilon&:= \frac{a-b}{2}>0\\
2\varepsilon&= b-a\\
\Rightarrow b-\varepsilon&= a+\varepsilon\\
b_n\to b\Rightarrow&b_n < b+\varepsilon\hspace{5mm}\forall n>n_0(\varepsilon)\\
a_n\to a\Rightarrow&-\varepsilon < a_n -a<\varepsilon\Rightarrow a-\varepsilon < a_n\hspace{5mm}\forall n>TODO
\end{align*}
Aber denn die Ungleichung 
\[b_n < b+\varepsilon = a-\varepsilon < a_n\hspace{5mm}\forall n\geq n_0\]
im Widerspruch zur Annahme $a_n\leq b_n$, $\forall n\in\N$
\end{enumerate}
\end{beweis}
Es ist nicht unbedingt nötig, den ganzen Beweis zu führen um zu wissen dass eine Folge konvergent ist. Es gibt Folgen deren Konvergenz, durch eine Strukturelle Eigenschaft gesichert ist ohne dass man den Limes apriori kennen muss.\\

Folgender Satz illustriert dieses, es benützt die Vollständigkeitsaxiom
\subsubsection*{Satz 3.9 (Monotone Konvergenz)}
\begin{enumerate}
\item Sei $\left( a_n\right)_{n > 1}$ eine monoton Wachsende beschränkte Folge. Dann ist sie konvergent und es gilt zudem \[\mathop {\lim }\limits_{n \to \infty } {a_n} = \sup \left\{ {{a_n}:n \ge 1} \right\}\]
\item Sei $\left( b_n\right)_{n \geq 1}$ eine Monotone fallende beschränkte Folge. Dann ist es konvergent und es gilt zudem \[\mathop {\lim }\limits_{n \to \infty } {b_n} = \inf \left\{ {{b_n}\mid n \ge 1} \right\}\]
\end{enumerate}

\begin{definition}{3.10}
\begin{itemize}
\item \textbf{Monotone wachsend}: \[a_n\leq a_{n+1}\hspace{5mm}\forall n\geq 1\]
\item \textbf{Monotone fallend}: \[b_{n+1}\leq b_{n}\hspace{5mm}\forall n\geq 1\]
\end{itemize}
\end{definition}
\todo[inline]{Number might be wrong, page 63 middle}

\begin{beweis}{3.9}
\begin{enumerate}[\hspace{2mm}i)]
\item $a_1\leq a_2\leq \dots$ und $\left\{ a_n:n\geq 1\right\}$ nach oben beschränkt $\Rightarrow\exists C$ mit $a_n\leq C$ $\forall n\geq 1$. Sei nach Satz 2.9 (Jede nach oben beschränkte Teilmenge $A\subset R$ besitzt ein kleinste obere Schränke)   $a:=\sup \left\{ a_n:n\geq 1\right\}$ die Kleinste Obere Schranke. \\

Wir behaupten dass: $a = \mathop {\lim }\limits_{n \to \infty } {a_n}$. Sei $\varepsilon>0$, dann ist $a-\varepsilon$ keine Obere Schranke und deswegen gibt es $n(\varepsilon)\geq 1$ mit $a_{n(\varepsilon)}>a-c$. Aus monotonität folgt 
\[a_n>a_{n(\varepsilon)} > a-c\hspace{5mm}\forall n>n(\varepsilon)\]
und folgt somit 
\[ \left| a_n-a\right| < \varepsilon\hspace{5mm}\forall n> n(\varepsilon)\]
\item Ähnlich.
\end{enumerate}
\end{beweis}
Sätze 3.8, 3.9 haben vielfähige Anwendungen die wir durch einige Beispiele illustrieren.
\subsubsection*{Beispiel 3.10}
\begin{enumerate}
\item Sei \[{a_n} = \frac{{3{n^6} + 11{n^4} - 1}}{{2{n^6} - 7{n^3} + n}} = \frac{{3 + \frac{{11}}{{{n^2}}} - \frac{1}{{{n^6}}}}}{{2 - \frac{7}{{{n^3}}} + \frac{1}{{{n^5}}}}} \to \frac{3}{2}\]
\item $\mathop {\lim }\limits_{n \to \infty } {\left( {1 + \frac{1}{n}} \right)^n}$ existiert. \\

Wir werden beweisen dass $a_n$ monotone wachsend und beschränkt ist. Der limes wird mit ``$e$'' beteichnet, wobei $e=2.71828\dots$ (Eulerische Konstant)
\begin{beweis}{}
\[a_n={\left( {1 + \frac{1}{n}} \right)^n}\]
Wir möchten den Binomischen Lehrsatz anwenden 
\begin{align*}
{a_n} =&{\left( {1 + \frac{1}{n}} \right)^n}\\ 
=&1 + n\left( {\frac{1}{n}} \right) + \frac{{n(n - 1)}}{{2!}}{\left( {\frac{1}{n}} \right)^2} + \frac{{n(n - 1)(n - 2)}}{{3!}}{\left( {\frac{1}{n}} \right)^3} +  \ldots  + {\left( {\frac{1}{n}} \right)^n}\\
=&1 + 1 + \frac{1}{{2!}}\left( {1 - \frac{1}{n}} \right) + \frac{1}{{3!}}\left( {1 - \frac{1}{n}} \right)\left( {1 - \frac{2}{n}} \right)\\
&+\ldots  + \frac{1}{{k!}}\left( {1 - \frac{1}{n}} \right)\left( {1 - \frac{2}{n}} \right) \ldots \left( {1 - \frac{{k - 1}}{n}} \right) +  \ldots 
\end{align*}
Nun ist aber 
\begin{align*}
\frac{1}{{2!}}\left( {1 - \frac{1}{n}} \right)&< \frac{1}{{2!}}\left( {1 - \frac{1}{{n + 1}}} \right)\\
\frac{1}{{3!}}\left( {1 - \frac{1}{n}} \right)\left( {1 - \frac{2}{n}} \right)&< \frac{1}{{3!}}\left( {1 - \frac{1}{{n + 1}}} \right)\left( {1 - \frac{2}{{n + 1}}} \right)\\
&\text{usw}
\end{align*}
deswegen folgt \[2 < a_n < a_{n+1}\hspace{5mm}\forall n\geq 1\]
d.h. $a_n$ ist monoton wachsend.\\

\noindent Die Produkte der Form
\begin{align*}
\left( {1 - \frac{1}{n}} \right) &\ldots \left( {1 - \frac{{k - 1}}{n}} \right) < 1\\
 \Rightarrow {a_n} &< 1 + 1 + \frac{1}{{2!}} + \frac{1}{{3!}} +  \ldots \\
&< 1 + 1 + \frac{1}{2} + \frac{1}{{{2^2}}} + \frac{1}{{{2^3}}} \ldots  = 3
\end{align*}
d.h. $a_n$ ist beschränkt. Monotone Konvergenz $\Rightarrow \left( a_n\right)_{n\geq 1}$ konvergiert
\end{beweis}
\item Rekursive Definitionen\\
Sei $\mathop {c > 1}\limits_{c \in\R }$, $a_1=c$, ${a_{n + 1}} = \frac{1}{2}\left( {{a_n} + \frac{c}{{{a_n}}}} \right)$, $n\geq 1$. Dann ist $\lim a_n=\sqrt{c}$
\begin{beweis}{}
Dies ist ein wichtiges Beispiel. Hier wird vorgeführt wie man aus der apriori Existenz des Limes dessen Wert schliessen kann.\\

\noindent\underline{1. Schnitt:}
\[a_{n + 1}^2 \ge c\hspace{5mm}\forall n\geq 1\]
$a_n$ ist (nach unten) beschränkt. 
\begin{align*}
{a_{n + 1}} =&\frac{{c + a_n^2}}{{2an}} = {a_n} + \frac{{c - a_n^2}}{{2{a_n}}}\\
 \Rightarrow a_{n + 1}^2 =&a_n^2 + \left( {c - a_n^2} \right) + {\left( {\frac{{c - a_n^2}}{{2{a_n}}}} \right)^2}\\
 =&c + {\left( {\frac{{c - a_n^2}}{{2{a_n}}}} \right)^2} \ge c\hspace{2mm}\text{(\textasteriskcentered)}
\end{align*}
\noindent\underline{2. Schnitt:}
\[a_{n+1}\leq a_n\]
d.h. $a_n$ ist monoton fallend. 
\begin{align*}
\text{(\textasteriskcentered)}: a_{n+1}^2&\geq c\\
\Rightarrow\frac{c}{a_{n+1}}&\leq a_{n+1}\hspace{5mm} \forall n\in\N\text{, insbesondere}\\
\frac{c}{a_n}&\leq a_n \\
\Rightarrow a_{n+1}&=\frac{1}{2}\left( {{a_n} + \frac{c}{{{a_n}}}} \right) \le \frac{1}{2}\left( {{a_n} + {a_n}} \right) = {a_n}
\end{align*}
Monotone Konvergenz Satz $\Rightarrow\left( a_n\right)$ konvergiert.\\

\noindent Sei $a=\lim a_n$. Da $a_n^2\geq c$, $\forall n\geq 2$ folgt $a^2\geq c$. Aus $a_{n+1}=\frac{1}{2}\left( a_n+\frac{c}{a_n}\right)$ und Satz 3.8 folgt $a=\frac{1}{2}\left( a+\frac{c}{a}\right)$ $\Rightarrow\frac{c}{a}=a\Rightarrow c=a^2\Rightarrow a=\sqrt{c}$. Schliesslich $\lim a_n=\sqrt{c}$
\end{beweis}
\end{enumerate}

\section{Teilfolgen, Häufungspunkte}
\begin{definition}{3.11}
Sei $\left( a_n\right)_{n\in\N}\subset\R$ eine folge und sei $l(n)_{n\in\N}$ eine strict monotone wachsend Folge von positiven Natürliche Zahlen. Die Verkettung von $l(n)$ und $\left( a_n\right)$ heisst eine Teilfolge $\left( a_{l(n)}\right)_{n\in\N}$ von $\left( a_{n}\right)_{n\in\N}$ 
\[n\to l(n)\to a_{l(n)}\]
\end{definition}
Die Idee ist sehr einfach: Wir haben die Folge \[a_1,a_2,a_3,a_4,a_5,a_6,\dots,a_j,\dots,a_{j+1},\dots \]
und wir definieren eine neue Folge mit einigen Elementen von $\left( a_n\right)$ 
\[a_1,a_3,a_6,a_{j+1},\dots \]

\subsubsection*{Beispiel}
\begin{enumerate}
\item \begin{align*}
{a_n} =&\left\{ {1,0, - 1,1,0, - 1,1,0, - 1, \ldots } \right\}\\
 =&\left\{ {\begin{array}{*{20}{c}}
0&{{\text{falls}}}&{n = 3k + 2}\\
1&{{\text{falls}}}&{n = 3k + 1}\\
{ - 1}&{{\text{falls}}}&{n = 3k + 3}
\end{array}} \right.
\end{align*}
\begin{align*}
n&\to 3n + 2 \to {a_{3n + 2}} = \left( {0,0, \ldots } \right)\\
n&\to 3n + 1 \to {a_{3n + 1}} = \left( {1,1, \ldots } \right)\\
n&\to 3n \to {a_{3n}} = \left( { - 1, - 1, \ldots } \right)
\end{align*}
\item $\left( a_n\right)_{n\in\N}$, $a_n=n\Rightarrow \left( 2^n\right)_{n\in\N}$ ist eine Teilfolge $n\to 2^n\to a_{2^n}$
\item $a_n=\left( -1\right)^n$, $\left( a_{2n}\right)_{n\geq 1}$ $\left( a_{2n+1}\right)$ sind Teilfolgen
\end{enumerate}

\subsubsection*{Bemerkung 3.12}
Im Definition 3.11 ist $l\left( \N\backslash\{ 0\}\right)$ eine unendliche Teilmenge von $\N\backslash\{ 0\}$. Umgekehrt, falls $\wedge\subset\N\backslash\{ 0\}$ eine unendliche Teilmenge ist dann enthält man eine Teilfolge von $\left( a_n\right)_{n\geq 1}$ mittels einer Monoton Abzählung $l:\N\backslash\{ 0\}\to\wedge$ von $\wedge$ ($l(n):=\min\left( \wedge \backslash\{ l(1),l(2),\dots ,l(n-1)\}\right)$)
\begin{definition}{3.13}
$a\in\R$ ist Häufungspunkt von $\left( a_n\right)_{n\geq 1}$ falls es eine gegen $a$ konvergierende Teilfolge $\left( a_{l(n)}\right)_{n\geq 1}$ gibt. 
\end{definition}
\subsubsection*{Beispiel 3.14}
\todo[inline]{Looks like there is no number 2, removed list in its entirety, page 71 middle to top}
$a_n=(-1)^n$ hat +1 und -1 als Häufungspunkte. Wir werden jetzt die Menge der Häufungspunkte einer Folge $\left( a_n\right)_{n\geq 1}$ näher studieren und Insbesondere zeigen dass sie für beschränkte Folgen nicht leer ist. \\

Sei $\left( a_n\right)_{n\geq 1}$ beschränkt und $C$ eine Obere Schranke für $\{ \left| a_n\right| : n\geq 1\}$. Für jedes $k\geq 1$ ist die Menge 
\[A_k:=\left\{ a_n:n\geq k\right\} = \left\{ a_k,a_{k+1},\dots\right\} \] beschränkt und zudem gilt \[A_{k+1}\subset a_k\text{, }\forall k\geq 1\]
Sei also 
\begin{itemize}
\item $m_k:=\inf A_k \nearrow\left( \inf A_k < \inf A_{k+1}\right)$
\item $M_k:=\sup A_k\searrow\left( \sup A_{k+1} < \sup A_k\right)$
\end{itemize} 
Dann folgt aus Korollar 2.11
\begin{enumerate}[\hspace{2mm}i)]
\item $\left( m_k\right)_{k\geq 1}$ monotone wachsend
\item $\left( M_k\right)_{k\geq 1}$ monotone fallend
\end{enumerate}
Nach Monotone Konvergenz Satz (Satz 3.9, s.\todo{add reference!!}) konvergieren beide Folgen

\begin{definition}{3.15}
Wir definieren
\begin{itemize}
\item $\mathop {\lim }\limits_{n \to \infty } \inf {a_n}: = \mathop {\lim }\limits_{k \to \infty } {m_k}$ limes inferior
\item $\mathop {\lim }\limits_{n \to \infty } \sup {a_n}: = \mathop {\lim }\limits_{k \to \infty } {M_k}$ limes superior
\end{itemize}
Offensichtlich gilt $\lim\inf a_n\geq \lim\sup a_n$
\end{definition}
\noindent Interessant ist nun:
\subsubsection*{Lemma 3.16}
Sei $\left( a_n\right)_{n\geq 1}$ beschränkt. Dann sind $\lim\sup a_n$ und $\lim\inf a_n$ Häufungspunkte von $\left( a_n\right)_{n\geq 1}$

\begin{beweis}{}
Sei $\mathop {\lim }\limits_{n \to \infty } \sup {a_n} = a$. Wir möchten zeigen dass, eine Teilfolge $a_{l(n)}$ gibt mit $\lim a_{l(n)}=a$. Wir definieren $l:\N\backslash\{ 0\}\to\N\backslash\{ 0\}$ Induktive wie folgt:\\

\noindent $l(1)\geq 1$ sei so gewählt, dass $M_1-1\leq a_{l(1)}\leq M_1=\sup A_1=\{ a_1,a_2,\dots\}$
\begin{framed}
\noindent\textbf{Korollar 2.11}\\
Sei $h\in\R$, $h>0$
\begin{enumerate}
\item[4.] Falls $E$ ein $\sup$ besitzt $\Rightarrow\exists x\in E$ mit $x>\sup E-h$ 
\end{enumerate}
Sei $E=\{ a_1,\dots\}=A_1$, $h=1$
\end{framed}
Sei $l(2)\in\{ k\in\N\mid k>l(1)+1\}$ so dass \[M_{l(1)+1}\leq a_{l(2)}\leq M_{l(1)+1}\]
(Sei $E=\{ a_{l(1)+1},a_{l(1)+2},\dots\}$, $h=\frac{1}{2}$). Falls $l(1),l(2),\dots,l(n-1)$ definiert ist, wählen wir $l(n)$ so dass:
\[ l(n)\in\{ k\in\N : k>l(n-1)+1\}\]
und 
\[
M_{l(n-1)+1}-\frac{1}{n}\leq a_{l(n)}\leq M_{l(n-1)+1}\tag{\text{\textasteriskcentered}}
\]
\[ \left| M_{l(n-1)+1}-a_{l(n)}\right| < \frac{1}{n}\]
Dann ist $l(n)$ strikt monotone steigend und 
\[ \left| a_{l(n)} -M_{l(n-1)+1}\right|\leq\frac{1}{n}\]
Sei nun $\varepsilon > 0$ und $n(\varepsilon)$ so gewählt dass $\frac{1}{n(\varepsilon)}<\frac{\varepsilon}{2}$ und 
\[a-\frac{\varepsilon}{2}\leq M_{l\left( n(\varepsilon)-1\right) +1}\leq a+\frac{\varepsilon}{2} \]
($a=\lim M_k$: d.h. $\forall\varepsilon>0$, $\exists N(\varepsilon)$ so dass $\left| M_n -a\right| < \frac{\varepsilon}{2}$ $\forall n > N(\varepsilon)$). Dann gilt $\forall n > n(\varepsilon):\frac{1}{n}<\frac{\varepsilon}{2}$
\[\left| M_{l(n-1)+1} - a\right| < \frac{\varepsilon}{2}\tag{\text{\textasteriskcentered\textasteriskcentered}}\]
und
\[\left| M_{l(n-1)+1} - a_{l(n)}\right| < \frac{1}{n}< \frac{\varepsilon}{2}\tag{\text{\textasteriskcentered}}\]
(\textasteriskcentered) und (\textasteriskcentered\textasteriskcentered) $\Rightarrow\left| a_{l(n)}-a\right| < \varepsilon$ d.h. $\lim a_{l(n)}=a.$
\end{beweis}
Wir schliessen aus Lemma 3.16 den folgenden wichtiger Satz
\subsubsection*{Satz 3.18 (Bolzano - Weierstrass)}
Jede Beschränkte Folge $\left( a_n\right)_{n\geq 1}$ besitzt eine konvergente Teilfolge.
\todo[inline]{MISSING CONTENT: Can't really understand the layout of this part, page 76 middle}
Folgende Aussagen sind direkte Konsequenz 
\subsubsection*{Satz 3.19}
Sei $\left( a_n\right)_{n\geq 1}\subset\R$ beschränkt. $a_{-}:=\lim\inf a_n$, $a_+:=\lim\sup a_n$
\begin{enumerate}
\item $\forall\varepsilon > 0$ gibt es nur endlich viele $n\in\N$ mit $a_n\not\in\left( a_- -\varepsilon,a_+ +\varepsilon\right)$
\item $a_+$ ist der grösste, $a_-$ der kleinste Häufungspunkt
\item Folgende Aussagen sind äquivalent
\begin{enumerate}[(i)]
\item $\left( a_n\right)_{n\geq 1}$ konvergiert
\item Jede Teilfolge von $\left( a_n\right)_{n\geq 1}$ konvergiert
\item $a_-=a_+$ 
\end{enumerate}
\end{enumerate}
\subsubsection*{Bemerkung}
$\left( a_n\right)$ konvergiert gegen $a\Rightarrow$ jede Teilfolge konvergiert gegen $a$. Das ist ein sehr nutzliches Kriterion für Konvergenz

\subsubsection*{Beispiel 3.20}
Wir definieren rekursiv 
\begin{align*}
g_1&=1, g_{n+1}=1+\frac{1}{g_n}, n\geq 1\\
g_1&=1, g_2=2, g_3=\frac{3}{2}, g_4=\frac{5}{3}
\end{align*}
\missingfigure{Page 77 bottom}
So die Folge ist nicht monoton. Offensichtlich gilt $g_n\geq 1$ und damit auch $g_n\leq 2$ d.h. $g_n$ ist beschränkt.\\

Aber Wir werden werden zwei Monoton Teilfolgen finden
\begin{align*}
{g_{n + 2}}&= 1 + \frac{1}{{{g_{n + 1}}}} = 1 + \frac{1}{{1 + \frac{1}{{{g_{n + 1}}}}}}\\
&= \frac{{2 + \frac{1}{{{g_{n + 1}}}}}}{{1 + \frac{1}{{{g_{n + 1}}}}}} = \frac{{2{g_{n + 1}}}}{{{g_{n + 1}}}} = 2 - \frac{1}{{{g_n} + 1}}
\end{align*}
Daraus folgt 
\[{g_{n + 2}} - {g_n} = \frac{1}{{{g_{n - 2}} + 1}} - \frac{1}{{{g_n} + 1}} = \frac{{{g_n} - {g_n} - 2}}{{\left( {{g_{n - 2}} + 1} \right)\left( {{g_n} + 1} \right)}}\]
Nun ist: $g_3-g_1 = \frac{3}{2}-1>0$ und somit ist $g_{2k+3}-g_{2k+1}>0$, $\forall k$ d.h. die Teilfolge $\left( g_{2k+1}\right)_{k\geq 0}$ ist monotone Wachsend.\\

\noindent$g_4-g_2=\frac{5}{3}-2<0$ woraus folgt $\left( g_{2k}\right)_{k\geq 1}$ monotone fallend ist. Seien also 
\begin{align*}
a&:=\lim_k g_{2k+1}\\
b&:=\lim_k g_{2k}
\end{align*}
Dann
\begin{align*}
a&:=\lim_k g_{2k+1}=\lim\left( 1+\frac{1}{g_{2k}}\right)\\
&=1+\frac{1}{b}\Rightarrow ab=b+1
\end{align*}
und Analog $b=1+\frac{1}{a}\Rightarrow ab=1+a$ woraus $ab=1+a=b+1$ und somit $a=b$. Folgt mit $g:=a=b$ $g=+\frac{1}{g}$ $\Rightarrow g=\frac{1+\sqrt{5}}{2}$\\

$g_+:=\lim\sup g_n$ und $g_-:=\lim\inf g_n$ sind Häufungspunkte, d.h. es gibt Teilfolgen $a_n$ und $b_n$ mit $\lim a_n=g_+$, $\lim b_n=g_-$. Da jede Teilfolge von $\left( g_n\right)$ entweder unendlich viele gerade oder ungerade Indizen enthält folgt 
\[g=g_+=g_-\]

Jede Teilfolge hat (ent.) unendliche viele Elemente von $\left( g_{2n}\right)$ (oder $\left( g_{2n+1}\right)$)\todo{Put in big brackets (including math)}
\[\left. {\begin{array}{*{20}{c}}
{{g_ + } = \lim {a_n} = \lim {g_{2n}} = g}\\
{{g_ + } = \lim {b_n} = \lim {g_{2n}} = g}
\end{array}} \right\}{g_ + } = {g_ - } \Rightarrow \lim {g_n} = g = {g_ - } = {g_ + } = \frac{{1 + \sqrt 5 }}{2}\]

\section{Cauchy Kriterium}
\todo[inline]{Chapter numbering is, according to the handwritten notes, wrong. WHich one is the correct one??}
\subsubsection*{Frage}
Wie sieht man allgemein ob eine Folge konvergiert? Sei $\left( a_n\right)_{n\geq 1}$ konvergiert mit $\lim a_n=a$. Also gibt es zu jedem $\varepsilon>0$, $n(\varepsilon)$ mit $\left| a_n-a\right|<\varepsilon$ $\forall n\geq n(\varepsilon)$ .\\
Daraus folgt, dass $\forall n,m\geq n(\varepsilon)$ 
\begin{align*}
\left|a_n-a_m\right|&= \left| a_n-a+a-a_m\right|\\
&\leq \left| a_n-a \right|+\left| a_m-a\right| < 2\varepsilon
\end{align*}
\begin{definition}{3.21}
$\left( a_n\right)_{n\geq 1}$ ist eine Cauchy Folge falls für $\varepsilon>0$ ein $n(\varepsilon)\geq 1$ gibt so dass $\left| a_n-a_m\right|<\varepsilon$, $\forall n,m\geq n(\varepsilon)$
\end{definition}
Wir haben gesehen dass 
\[ \left( a_n\right) \text{ konvergiert }\Rightarrow\left( a_n\right) \text{ Cauchy}\]
Wir haben auch
\subsubsection*{Satz 3.22 (Cauchy Kriterium)}
Sei $\left( a_n\right)_{n\geq 1}\subset\R$ eine folge. Die folgende Aussagen sind äquivalent 
\begin{enumerate}
\item $\left( a_n\right)$ ist eine Cauchy Folge
\item $\left( a_n\right)$ ist konvergent
\end{enumerate}

\begin{beweis}{}
\begin{enumerate}[align=left]
\item[$(2)\Rightarrow(1)$]\checkmark
\item[$(1)\Rightarrow(2)$] Wegen das Satzes von Bolzano - Weierstrass besitzt jede beschränkte Folge eine konvergente Teilfolge  \\

\underline{Strategie:} 
\begin{enumerate}[I)]
\item $\left( a_n\right)$ beschränkt
\item $\exists\left( a_{l(n)}\right)\subset \left( a_n\right)$ Konvergente Teilfolge
\end{enumerate}
Sei $\lim a_{l(n)}=a$, $\left( a_n\right)$ Cauchy.
\begin{align*}
\left|a_n-a\right|&= \left| a_n-a_{l(n)}+a_{l(n)}-a\right|\\
&\leq \left| a_n-a_{l(n)} \right|+\left| a_{l(n)}-a\right| < 2\varepsilon
\end{align*}

\begin{enumerate}[I)]
\item $\left(a_n\right)$ ist beschränkt: Sei $\varepsilon=1$. Sei $n(1)$ so dass $\left| a_n-a_m\right|<1$, $\forall n,m\geq n(1)$, insbesondere $\left| a_n-a_m\right|<1$. Woraus $\left| a_n\right| < a_{n(1)}+1$, $\forall n\geq n(1)$ folgt und somit $\forall n\geq 1$ 
\[\left| {{a_n}} \right| \le \max \left\{ {\left| {{a_1}} \right|, \ldots ,\left| {{a_{n(1)}}} \right|,\left| {{a_{n(1)}}} \right| + 1} \right\}\]
d.h. $\left( a_n\right)$ ist beschränkt
\item Sei $a$ ein Häufungspunkt von $\left( a_n\right)$ (Bolzano - Weierstrass) und $l:\N\to\N$ strikt monotone mit 
\[\mathop {\lim }\limits_{n \to \infty } {a_{l(n)}} = a\hspace{10mm}\left( {{\text{Bem.: }}l(n) \ge n} \right)\]
\end{enumerate}
Sei $\varepsilon>0$ und $n_0(\varepsilon)$ so dass 
\[\left| {{a_{l(n)}} - a} \right| < \varepsilon \hspace{10mm}\forall n > {n_0}(\varepsilon )\]
Sowie $n_1(\varepsilon)$ mit \\\todo{Can't understand, page 83 bottom}

$\forall n \ge \max \left\{ {{n_0}(\varepsilon ),{n_1}(\varepsilon )} \right\}$ gilt:
\begin{align*}
\left| {{a_n} - a} \right| &\le \left| {{a_n} - {a_{l(n)}}} \right| + \left| {{a_{l(n)}} - a} \right|\\
&\le \varepsilon  + \varepsilon  = 2\varepsilon 
\end{align*}
\[ \Rightarrow \lim a_n=a\Rightarrow \left( a_n\right) \text{ Konvergiert}\]
\end{enumerate}
\end{beweis}

\subsubsection*{Beispiel 3.23}
\begin{enumerate}
\item Sei \[{a_n}: = 1 + \frac{1}{2} +  \ldots  + \frac{1}{n} = \sum\limits_{k = 1}^n {\frac{1}{k}} \] Dann ist also $1\leq a_n\leq a_{n+1}\dots$ monotone wachsend, aber divergent. Dann:
\[{a_{2n}} - {a_n} = \frac{1}{{n + 1}} +  \ldots  + \frac{1}{{2n}} \ge \frac{n}{{2n}} = \frac{1}{2}\hspace{5mm}\forall n \ge 1\] 
Es erfüllt also nicht das Cauchy - Kriterium
\item Sei ${b_n}: = 1 - \frac{1}{2} + \frac{1}{3} +  \ldots  + {\left( { - 1} \right)^{n + 1}}\frac{1}{n}$ die alternierende harmonische Reihe. Insbesondere:
\begin{align*}
{b_{2k - 2}}&= \left( {1 - \frac{1}{2}} \right) + \left( {\frac{1}{3} - \frac{1}{4}} \right) +  \ldots  + \left( {\frac{1}{{2k - 3}} - \frac{1}{{2k - 2}}} \right)\\
{b_{2k}}&= \left( {1 - \frac{1}{2}} \right) +  \ldots  + \left( {\frac{1}{{2k - 1}} - \frac{1}{{2k}}} \right)
\end{align*}
also folgt \[ 0 < b_{2k-2} < b_{2k}\hspace{10mm}\forall k\geq 1\]
und 
\begin{align*}
{b_{2k + 1}}&= {b_{2k - 1}} - \frac{1}{{2k}} + \frac{1}{{2k + 1}}\\
&= {b_{2k - 1}} + \underbrace {\left( {\frac{1}{{2k + 1}} - \frac{1}{{2k}}} \right)}_{ < 0}
\end{align*}
Woraus: $b_{2k+1}<b_{2k-1}$. Zudem 
\begin{align*}
b_{2k}&=b_{2k-1}-\frac{1}{2k}
b_{2k}&<b_{2k-1}
\end{align*}
$\forall k\in\N$
\[\frac{1}{2} = {b_2} < {b_4} \ldots  < {b_{2k - 2}} < {b_{2k}} < {b_{2k - 1}} < ?? < {b_1} = 1\]
\todo[inline]{Check question marks right above, page 85 bottom}
Die Teilfolgen $\left( b_{2k}\right)_{k\in\N}$, $\left( b_{2k+1}\right)_{k\in\N}$ konvergiert nach Satz 3.9 (Monotone konvergenz Satz). Da 
\[\forall n \in\N:\left| {{b_n} - {b_{n + 1}}} \right| = \frac{1}{{n + 1}}\]
haben sie zudem denselben Limes und $\left( b_n\right)_{n\in\N}$ konvergiert nach Satz 3.19 (Analog wie in Beispiel 3.20)
\end{enumerate}

\section{Folgen in $\R^d$ oder $\C$}
Die Theorie der Folgen in $\R$, der Konvergenzbegriff usw. lassen sich leicht auf Folgen in $\R^d$ oder $\C$ übertragen $\norm{\cdot}$ bezeichnet die Euklidische Norm auf $\R^d$ und $\C$
\[\norm{x}=\sqrt{x_1^2+x_2^2+\dots+x_d^2}\]
Von diesem Standpunkt identifiziert sich $\C$ mit $\R^2$ so dass wir von Jetzt an Folgen in $\R^d$ betrachten. Die in 3.1 eingeführte Begriffe lassen sich leicht auf $\R^d$ übertragen
\begin{definition}{}
Eine Folge in $\R^d$ ist eine Abbildung 
\begin{align*}
a:\N\backslash\{ 0\}&\to\R^d\\
n&\to a_n
\end{align*}
\end{definition}
\begin{definition}{3.24}
Eine Folge $\left( a_n\right)_{n\geq 1}$ in $\R^d$ heisst beschränkt falls es $c>0$ gibt mit $\norm{a_n}\leq c$, $\forall n\geq 1$
\end{definition}
\subsubsection*{Bemerkung}
Für $d\geq 2$ haben wir keine Vollständige Ordnung, deswegen lassen sich Begriffe wie ``nach oben beschränkt'' nicht übertragen

\begin{definition}{3.25}
Eine Folge $\left( a_n\right)_{n\geq 1}$ in $\R^d$ konvergiert gegen $a\in\R^d$ falls für jedes $\varepsilon>0$ einen Index $N(\varepsilon)\geq 1$ gibt so dass 
\[\norm{a_n-a}<\varepsilon\hspace{10mm}\forall n\geq N(\varepsilon)\]
\end{definition}
\todo[inline]{Not sure where the definition ends}
Die andere Version lässt sich auch übertragen. Wir definieren dafür der (offene) \todo{??r.Ball?? page 88 bottom} mit Zentrum $a\in\R^d$ 
\[B_{<r}(a):=\left\{ x\in\R^d : \norm{x-a}<r\right\}\]
\missingfigure{page 88 bottom}
$B_{<r}(a)$ ist die Verallgemeinerung von $\left( a-\varepsilon,a+\varepsilon\right)$. Nützlich ist auch der (geschlossene) \todo{??r.Ball?? page 89 top}
\[\overline{B}_r(a):B_{\leq r}(a):=\left\{ x\in\R^d : \norm{x-a}\leq r\right\}\] 
der das Interval $\left[ a-r,a+r\right]$ verallgemeine.

\begin{definition}{3.25}
Eine Folge $\left( a_n\right)_{n\geq 1}\subset\R^d$ konvergiert gegen $a\in\R^d$ falls für jedes $\varepsilon>0$, die Menge der Indizen $n\geq 1$ für welche $a_n\not\in B_{<\varepsilon}(a)$ endlich ist. Falls dieses Zutrifft, schreibt man 
\[\mathop {\lim }\limits_{n \to \infty } {a_n} = a{\text{ oder }}{a_n}\mathop  \to \limits^{n \to \infty } a\] 
\end{definition}

\subsubsection*{Bemerkung}
Die Konvergenz von $\left( a_n\right)_{n\geq 1}\subset\R^d$ ist gleichbedeutend mit der Existenz von einem Vektor $a\in\R^d$ so dass die Folge in $\R$, $\left( \norm{a_n-a}\right)_{n\geq 1}$ gegen 0 konvergiert \\

\noindent Es gilt dann wieder
\subsubsection*{Lemma 3.26}
Sei $\left( a_n\right)_{n\geq 1}\subseteq\R^d$ konvergent
\begin{enumerate}
\item Der Limes ist eindeutig bestimmt
\item Die Folge $\left( a_n\right)_{n\geq 1}$ ist beschränkt
\end{enumerate}
Der Konvergenzbegriff verträgt auch sehr gut mit Vektorraum Struktur wie das folgende Analog von Satz 3.8 zeigt.

\subsubsection*{Satz 3.27}
Seien $\left( a_n\right)_{n\geq 1}$, $\left( b_n\right)_{n\geq 1}$ konvergente Folgen in $\R^d$, sowie $\lambda\in\R$. Sei $a=\lim a_n$, $b=\lim b_n$. Dann sind $\left( a_n\pm b_n\right)_{n\geq 1}$ und $\left( \lambda a_n\right)_{n\geq 1}$ Konvergiert und es gilt 
\[\lim\left( a_n\pm b_n\right) = a\pm b\text{, }\lim\lambda a_n=\lambda a\]
Folgender Satz ist dann grundlegend um Bolzano - Weierstrass sowie der Cauchy Kriterium auf $\R^d$ zu verallgemeinen. \\

Für eine Folge $\left( a_n\right)$ von Vektoren in $\R^d$ ist es Zweckmässig folgende Notation für die Koordinaten von $a_n$ zu benutzen \[a_n=\left( a_n^{(1)},a_n^{(2)},a_n^{(3)}\right)\]
Dann gilt
\subsubsection*{Satz 3.28}
Folgende Aussagen sind äquivalent
\begin{enumerate}[\hspace{2mm}(i)]
\item $\left( a_n\right)_{n\geq 1}$ konvergiert in $\R^d$ 
\item Jede der Folgen $\left( a_n^i\right)$ konvergiert in $\R$
\end{enumerate}
Falls diese zutrifft seien $a = \mathop {\lim }\limits_{n \to \infty } {a_n}$ sowie ${a^i} = \mathop {\lim }\limits_{n \to \infty } a_n^{(i)}$ dann gilt $a=\left( a^1,a^2,\dots,a^d\right)$ 
\begin{beweis}{}
Dazu brauchen wir folgendes geometrisches Lemma:
\end{beweis}
\subsubsection*{Lemma 3.29}
$\forall x=\left( x^1,x^2,\dots,x^d\right) \in\R^d$ gilt 
\[\mathop {\max }\limits_{1 \le i \le d} \left| {{x^i}} \right| \le \mathop {\left\| x \right\|}\limits_{\begin{array}{*{20}{c}}
 \Downarrow \\
{\sqrt {\sum {{{\left| {{x^i}} \right|}^2}} } }
\end{array}}  \le \sqrt d \mathop {\max }\limits_{1 \le i \le d} \left| {{x^i}} \right|\]
\missingfigure{page 92, top}
\[{\left( { - \frac{r}{{\sqrt 2 }},\frac{r}{{\sqrt 2 }}} \right)^2} \subset {B_{ \le r}}\left( 0 \right) \subset {\left( { - r,r} \right)^2}\]

\begin{beweis}{von Satz 3.28}
\begin{enumerate}[align=left]
\item[(i)$\Rightarrow$(ii)]Folgt aus der Ungleichung $\left| {a_n^{(i)} - {a^{(i)}}} \right| \le \left\| {{a_n} - a} \right\|$
\item[(ii)$\Rightarrow$(i)]Sei $a^i=\lim a_n^i$ und $a=\left( a^i\right)_i$. Aus \[\left\| {{a_n} - a} \right\| \le \sqrt d \mathop {\max }\limits_{1 \le i \le d} \left| {a_n^{(i)} - {a^{(i)}}} \right|\] folgt (i)
\end{enumerate}
\end{beweis}

\subsubsection*{Satz 3.29 (Bolzano - Weierstrasse)}
Jede beschränkte Folge $\left( a_n\right)_{n\geq 1}$ in $\R^d$ besitzt eine konvergente Teilfolge
\subsubsection*{Beispiel}
\begin{enumerate}
\item\begin{align*} \left( {{a_n}} \right)&= \left( {\begin{array}{*{20}{c}}
{{\raise0.7ex\hbox{$1$} \!\mathord{\left/
 {\vphantom {1 n}}\right.\kern-\nulldelimiterspace}
\!\lower0.7ex\hbox{$n$}}}\\
{{\raise0.7ex\hbox{$2$} \!\mathord{\left/
 {\vphantom {2 n}}\right.\kern-\nulldelimiterspace}
\!\lower0.7ex\hbox{$n$}}}
\end{array}} \right) \subset\R^2\\
\lim a_n&=\left( {\begin{array}{*{20}{c}}
0\\
0
\end{array}} \right) \end{align*}
\item \[\left( {{a_n}} \right) = \left( {\begin{array}{*{20}{c}}
{\frac{{{n^2} + n + 1}}{{2{n^2} + n + 1}}}\\
n
\end{array}} \right)\]
$\left( a_n\right)$ ist divergent
\[\lim \frac{{{n^2} + n + 1}}{{2{n^2} + n + 1}} \to \frac{1}{2}\] aber $\lim n=\infty$
\end{enumerate}

\begin{definition}{3.30}
$\left( a_n\right)_{n\geq 1}$ ist eine Cauchy Folge falls es $\forall\varepsilon >0$ ein $N(\varepsilon)\geq 1$ gibt so dass 
\[\left\| {{a_n} - {a_m}} \right\| < \varepsilon \hspace{10mm}\forall n,m \ge N(\varepsilon )\]
\end{definition}
\noindent Aus Sätze 3.28 und 3.22 (Cauchy Kriterium) folgt 
\subsubsection*{Satz 3.31}
Es sind äquivalent 
\begin{enumerate}
\item $\left( a_n\right)_{n\geq 1}$ konvergiert
\item $\left( a_n\right)$ ist eine Cauchy Folge
\end{enumerate}
Für $\C$ haben wir noch dass sich die Körperstruktur mit Konvergenz gut verträgt. Nähmlich
\subsubsection*{Satz 3.32}
Seien $\left( z_n\right)$, $\left( w_n\right)$ zwei Konvergente Folgen in $\C$ mit $z=\lim z_n$, $w=\lim w_n$. Dann 
\begin{enumerate}[(i)]
\item $\overline{z_n}\to z_n$ und ${\left( {\left\| {{z_n}} \right\|} \right)_{n \ge 1}}$ konvergiert gegen $\norm{z}$
\item Die Folge $\left( z_n w_n\right)_{n\geq 1}$konvergiert gegen $zw$
\item Falls $w\not =0$ und $w_n\not=0$, $\forall n\geq 1$ so konvergiert ${\left( \frac{z_n}{w_n} \right)_{n \ge 1}}$ gegen $\frac{z}{w}$
\end{enumerate}

\section{Reihen}
Sei $\left( a_n\right)_{n\geq 1}$ eine Folge in $\R$ oder $\C$. Sei 
\[ S_n:=a_1+a_2+\dots +a_n=\sum\limits_{k = 1}^n {{a_k}} \]
die Folge der Partial Summen. Eine Reihe ist eine unendliche Summe\[{a_1} + {a_2} +  \ldots  = \sum\limits_{n = 1}^\infty {{a_n}} \] einer Folge $\left( a_n\right)_{n\geq 1}$

\begin{definition}{3.33}
Die Reihe $\sum\limits_{n = 1}^\infty a_n$ ist konvergent falls die Folge $\left( S_n\right)_{n\geq 1}$ konvergiert. In diesem Fall wird deren Limes mit $\sum\limits_{n = 1}^\infty a_n$ bezeichnet
\[\lim {S_n} = \lim \sum\limits_{k = 1}^n {{a_k}: = \sum\limits_{n = 1}^\infty  {{a_n}} } \]
\end{definition}
\subsubsection*{Beispiel 3.34}
\begin{enumerate}
\item Für $\abs{q}<1$ gilt 
\begin{align*}
{S_n}&= \sum\limits_{k = 0}^n {{q^k} = \frac{{1 - {q^{n + 1}}}}{{1 - q}}} \\
\lim {S_n}&= \mathop {\lim }\limits_{n \to \infty } \frac{{1 - {q^{n + 1}}}}{{1 - q}} = \frac{1}{{1 - q}}\text{  (da $\lim q^n=0$)}
\end{align*}
Somit konvergiert $\sum\limits_{k = 1}^\infty  {{q^k}} $ und deren Wert ist $\frac{1}{1-q}$
\item Die Harmonische Reihe $\sum\limits_{n = 1}^\infty  {\frac{1}{n}} $ ist nach Beispiel 3.23 (i)\todo{ADD reference} divergent.
\end{enumerate}
Für Reihen gibt es verschiedene praktische Konvergenz Kriterium. Das erste ergibt sich direkt aus dem Cauchy Kriterium (Satz 3.22)\todo{add reference + page number}

\subsubsection*{Satz 3.35 (Cauchy Kriterium)}
Die Reihe $\sum\limits_{n = 1}^\infty  {{a_n}} $ konvergiert genau dann wenn es für jedes $\varepsilon>0$ einer Index $N(\varepsilon)\geq 1$ gibt, so dass 
\[ \forall n\geq m\geq N(\varepsilon)\hspace{5mm}\left| \sum\limits_{k=m}^n a_k\right|<\varepsilon\]

\begin{beweis}{}
\todo[inline]{It says Übung, maybe incomplete? page 97 top}
Die Reihe $\sum\limits_1^\infty  {{a_j}} $ konvergiert genau dann, wenn 
\[\left| {\sum\limits_m^n {{a_j}} } \right| \to 0\hspace{10mm}\forall n \ge m\]
$\sum\limits_1^\infty  {{a_j}} $ konv. $\Leftrightarrow S_n=\sum\limits_1^n a_k$ konv. $\Leftrightarrow S_n$ Cauchy $\Rightarrow \forall\varepsilon>0, \exists N(\varepsilon)$ s.d. 
\[\forall n > m > N(\varepsilon)\hspace{10mm}\underbrace {\left| {{S_n} - {S_m}} \right|}_{\sum\limits_m^n {{a_k}} } < \varepsilon \]
\end{beweis}
\subsubsection*{Korollar}
$\sum\limits_{k = 1}^\infty  {{a_k}} $ konvergent $\Rightarrow\left| a_k\right|\to 0$

\begin{beweis}{}
Nehmen wir $m=n$ in Satz 3.5\todo{Add reference + page number}. Das ist ein Notwendiges Kriterium aber nicht hinreichendes Kriterium 
\end{beweis} 
\todo[inline]{Content between brackets  looks like personal notes, should these be copied? page 97 bottom}

\subsubsection*{Beispiel}
$\sum\limits_1^\infty  k $ ist nicht konvergent, weil $\mathop {\lim }\limits_{k \to \infty } k\not  = 0$ (notwendig).\\
$\sum\frac{1}{k}$ ist nicht konvergent obwohl $\lim\frac{1}{k}=0$ (nicht genügends)\\

Im Folgenden leiten wir aus Vergleich mit der geometrischen Reihe verschiedene Konvergenz Kriterium ab (Quotienten, Wurzelkriterium). Dies Stützt sich auf

\subsubsection*{Satz 3.36}
Seien $\sum\limits_1^\infty  {{a_k}} $, $\sum\limits_1^\infty  {{b_k}} $ Reihen wobei 
\begin{enumerate}
\item Es gibt $k_0$ so dass $\left| a_k\right| \leq b_k$, $\forall k > k_0$ 
\item $\sum b_k$ konvergiert
\end{enumerate}
Dann konvergiert $\sum\limits_{k = 1}^\infty  {{a_k}} $

\begin{beweis}{}
Sei $\varepsilon>0$ und $N(\varepsilon) > k_0$ so dass 
\[\sum\limits_{k = m}^n {{b_k}}  = \left| {\sum\limits_{k = m}^n {{b_k}} } \right| \le \varepsilon\hspace{10mm} \forall n,m > N(\varepsilon ) > {k_0}\]
Dann folgt aus 1. 
\[\left| {\sum\limits_{k = m}^n {{a_k}} } \right| \le \sum\limits_{k = m}^n {\left| {{a_k}} \right|}  \le \sum\limits_{k = m}^n {{b_k}}  \le \varepsilon \]
Der Satz folgt aus dem Cauchy Kriterium
\end{beweis}
\subsubsection*{Beispiel}
\begin{enumerate}
\item \[\sum\limits_{k = 1}^\infty  {\frac{1}{{k!}} = ?} \]
\[\frac{1}{{k!}} \le \frac{1}{{{2^{k - 1}}}}\hspace{5mm}\forall k \ge 1 \Rightarrow \sum\limits_1^\infty  {\frac{1}{{k!}} < } \sum\limits_1^\infty  {\frac{1}{{{2^{k - 1}}}}}  = 2\]
\item \[\sum\limits_{k = 1}^\infty  {\frac{1}{{{{\left( {k + 1} \right)}^2}}} = ?} \]
Zum erst zeigen wir dass $\sum\limits_1^\infty  {\frac{1}{{k\left( {k + 1} \right)}}} $ konvergiert. Da 
\begin{align*}
\frac{1}{{k\left( {k + 1} \right)}}&= \frac{1}{k} - \frac{1}{{k + 1}}\hspace{10mm}\forall k \ge 1\\
{S_n} = \sum\limits_1^\infty  {\frac{1}{{k\left( {k + 1} \right)}}} &= \left( {1 - \frac{1}{2}} \right) + \left( {\frac{1}{2} - \frac{1}{3}} \right) +  \ldots  + \left( {\frac{1}{{n - 1}} - \frac{1}{n}} \right)\\
&= 1 - \frac{1}{n}
\end{align*}
$\mathop {\lim }\limits_{n \to \infty } {S_n} = 1,{\text{ d.h. }}\sum\limits_1^\infty  {\frac{1}{{k\left( {k + 1} \right)}} = 1} {\text{ }}$.\\
$\forall k > 1$ Da 
\begin{align*}
{\left( {k + 1} \right)^2} > k\left( {k + 1} \right)\\
\frac{1}{{{{\left( {k + 1} \right)}^2}}} < \frac{1}{{k\left( {k + 1} \right)}}
\end{align*}
Daraus folgt 
\[\sum {\frac{1}{{{{\left( {k + 1} \right)}^2}}}}  \le \sum {\frac{1}{{k\left( {k + 1} \right)}}}  = 1\]
$\sum {\frac{1}{{{{\left( {k + 1} \right)}^2}}}} $ konvergiert.
\end{enumerate}

\subsubsection*{Satz 3.37 (Quotientenkriterium)}
Sei $a_k\not=0$, $\forall k\geq 1$ 

\begin{enumerate}[(i)]
\item Falls \[\mathop {\lim }\limits_{k \to \infty } \sup \left| {\frac{{{a_{k + 1}}}}{{{a_k}}}} \right| < 1\] so ist $\sum a_k$ konvergent
\item Falls \[\mathop {\lim }\limits_{k \to \infty } \inf \left| {\frac{{{a_{k + 1}}}}{{{a_k}}}} \right| > 1\] so ist $\sum a_k$ divergent
\end{enumerate}

\begin{beweis}{}
\begin{enumerate}[(i)]
\item Sei ${q_0}: = \lim \sup \left| {\frac{{{a_{k + 1}}}}{{{a_k}}}} \right|$ \[\mathop {\lim }\limits_{k \to \infty } \left( {\sup \left| {\frac{{{a_{k + 1}}}}{{{a_k}}}} \right|} \right) = \mathop {\lim }\limits_{n \to \infty } \left( {\mathop {\sup }\limits_{k \ge n} \left| {\frac{{{a_{k + 1}}}}{{{a_k}}}} \right|} \right)\] 
Wähle $q\in\R$ mit $q_0<q<1$. Dann gilt für genugend gross $n_0\in\N$ 
\begin{align*}
\forall n \ge {n_0}:&\left| {\mathop {\sup }\limits_{k \ge n} \left| {\frac{{{a_{k + 1}}}}{{{a_k}}}} \right| - {q_0}} \right| \le \underbrace {\left( {q - {q_0}} \right)}_\varepsilon\\
{\text{d.h. }}&\text{ }\mathop {\sup }\limits_{k \ge n} \left| {\frac{{{a_{k + 1}}}}{{{a_k}}}} \right| \le q\hspace{5mm}\forall n \ge {n_0}
\end{align*}
Insbesondere bei Wahl von $n=n_0$ 
\[\forall k \ge {n_0}\hspace{10mm}\left| {\frac{{{a_{k + 1}}}}{{{a_k}}}} \right| \le q\]
Es folgt für $k\geq n_0$ die Abschätzung
\begin{align*}
\left| {{a_k}} \right|&= \left| {\frac{{{a_k}}}{{{a_{k - 1}}}} \cdot \frac{{{a_{k - 1}}}}{{{a_{k - 2}}}} \cdot  \ldots  \cdot \frac{{{a_{{n_0} + 1}}}}{{{a_{{n_0}}}}} \cdot {a_{{n_0}}}} \right|\\
&\le {q^{k - {n_0}}}\left| {{a_{{n_0}}}} \right| = \underbrace {{q^n}\left| {{a_{{n_0}}}} \right|}_C{q^k} = C{q^k}
\end{align*}
Wir können nun $b_k=Cq^k$ setzen und Satz 3.36 \todo{Add refence + page number} anwenden. Da $\sum b_k$ konvergiert (da $\left| q\right| < 1$) $\sum a_k$ konvergiert.
\item Sei 
\[{q_0}: = \mathop {\lim }\limits_{k \to \infty } \inf \left| {\frac{{{a_{k + 1}}}}{{{a_k}}}} \right| = \mathop {\lim }\limits_{n \to \infty } \mathop {\inf }\limits_{k \ge n} \left| {\frac{{{a_{k + 1}}}}{{{a_k}}}} \right|\]
(falls es existiert). Wähle $q\in\R$ mit $q_0>q>1$. Dann existiert $n_0$ mit \[\forall k > {n_0}:\left| {\frac{{{a_{k + 1}}}}{{{a_k}}}} \right| > \mathop {\inf }\limits_{k > {n_0}} \left| {\frac{{{a_{k + 1}}}}{{{a_k}}}} \right| \ge q\]
\[\left( \begin{array}{l}
\hspace{4mm}\left| {\mathop {\inf }\limits_{k \ge n} \left| {\frac{{{a_{k + 1}}}}{{{a_k}}}} \right| - {q_0}} \right| < \underbrace {q - {q_0}}_\varepsilon \hspace{4mm}\forall k > {n_0}\\
 \Rightarrow  - {q_0} + q < \inf \left| {\frac{{{a_{k + 1}}}}{{{a_k}}}} \right| - {q_0}\hspace{5mm}\forall k > {n_0}
\end{array} \right)\]
Dann folgt analog wie in (i) dass 
\begin{align*}
\left| {{a_k}} \right|&= \left| {\frac{{{a_k}}}{{{a_{k - 1}}}} \cdot \frac{{{a_{k - 1}}}}{{{a_{k - 2}}}} \cdot  \ldots  \cdot \frac{{{a_{{n_0} + 1}}}}{{{a_{{n_0}}}}} \cdot {a_{{n_0}}}} \right|\\
&> {q^{k - {n_0}}}\left| {{a_{{n_0}}}} \right| > C{q^k}, \forall k>k_0
\end{align*}
Insbesondere ist $\left\{ {\left| {{a_k}} \right|:k \ge 1} \right\}$ nicht beschränkt $\Rightarrow$ $\lim a_k\not\to 0\Rightarrow\sum a_k$ divergent.\\
Falls $\lim \left| {\frac{{{a_{k + 1}}}}{{{a_k}}}} \right|$ existiert, dann \[\lim \inf \left| {\frac{{{a_{k + 1}}}}{{{a_k}}}} \right| = \lim \sup \left| {\frac{{{a_{k + 1}}}}{{{a_k}}}} \right| = \lim \left| {\frac{{{a_{k + 1}}}}{{{a_k}}}} \right| = L\]
\end{enumerate}
\end{beweis}

\subsubsection*{Satz 3.36'}
Sei $a_k\not=0$, $\forall k\in\N$ und sei $L = \mathop {\lim }\limits_{k \to \infty } \left| {\frac{{{a_{k + 1}}}}{{{a_k}}}} \right|$
\begin{enumerate}[(i)]
\item Falls $L<1$, so ist $\sum a_k$ konvergent
\item Falls $L>1$, so ist $\sum a_k$ divergent
\item Falls $L=1$, kann man daraus nichts ableiten 
\end{enumerate}
\subsubsection*{Beispiel}
\begin{enumerate}
\item $\sum\limits_1^\infty  {\frac{{n!}}{{{n^n}}}} $ konvergiert. Da
\begin{align*}
\left| {\frac{{(n + 1)!}}{{{{\left( {n + 1} \right)}^{n + 1}}}} \cdot \frac{{{n^n}}}{{n!}}} \right|&= \left( {n + 1} \right) \cdot \frac{{{n^n}}}{{{{\left( {n + 1} \right)}^{n + 1}}}} = {\left( {\frac{n}{{n + 1}}} \right)^n}\\
&={\left( {\frac{1}{{1 + \frac{1}{n}}}} \right)^n} \to \frac{1}{e} < 1
\end{align*} 
Insbesondere $\frac{n!}{n^n}\to 0$ falls $n\to\infty$, d.h. $n^n$ wächst schneller als $n!$ (schon gesehen) \todo{Maybe add where?? page 104 top}
\item Wir haben schon gesehen dass $\sum\frac{1}{n^2}$ konvergiert. Aber Quotientenkriterium gibt kein informations dass \[\left| {\frac{{{a_{n + 1}}}}{{{a_n}}}} \right| = \left| {\frac{{{{\left( {n + 1} \right)}^2}}}{{{n^2}}}} \right| \to 1\]
\item Wir haben auch schon gesehen dass $\sum\frac{1}{n}$ divergiert. In diesem fall auch \[\left| {\frac{{{a_{n + 1}}}}{{{a_n}}}} \right| = {\frac{{{{ {k + 1} }}}}{{{k}}}}  \to 1\], d.h. das Quotientenkriterium ist nicht anwendbar.
\end{enumerate}

%START WEEK 8






